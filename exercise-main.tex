% Preamble
%\documentclass[11pt]{report}
\documentclass[11pt]{article}
% Packages
\usepackage[a4paper, margin=2.0cm]{geometry}
\usepackage{amsmath}
\usepackage{amssymb}
\usepackage{amsthm}
\usepackage{mathtools}
\usepackage{enumitem}
\usepackage{hyperref}
\usepackage{graphicx}
\usepackage{subcaption}

%\setlength\parindent{0pt}

% Self-defined commands
\newcommand{\calE}{\mathcal{E}}
\newcommand{\calI}{\mathcal{I}}

\newcommand{\bbQ}{\mathbb{Q}}
\newcommand{\bbR}{\mathbb{R}}
\newcommand{\bbZ}{\mathbb{Z}}

\newcommand{\rmd}{\mathrm{d}}

\newcommand{\bfe}{\mathbf{e}}
\newcommand{\bff}{\mathbf{f}}
\newcommand{\bfh}{\mathbf{h}}
\newcommand{\bft}{\mathbf{t}}
\newcommand{\bfx}{\mathbf{x}}
\newcommand{\bfy}{\mathbf{y}}
\newcommand{\bfr}{\mathbf{R^1}}

\DeclareMathOperator*\lolim{\underline{lim}}
\DeclareMathOperator*\uplim{\overline{lim}}
\DeclareMathOperator{\dist}{dist}
\DeclareMathOperator{\vol}{vol}

\newcommand{\mea}[1]{\left\vert#1\right\vert}
\newcommand{\outmea}[1]{\left\vert#1\right\vert_\mathrm{e}}
\newcommand{\inmea}[1]{\left\vert#1\right\vert_\mathrm{i}}

\newtheorem*{lemma}{Lemma}

\theoremstyle{definition}
\newtheorem{innerexercise}{Exercise}
\newenvironment{exercise}[1]
{\pushQED{\qed}\renewcommand\theinnerexercise{#1}\innerexercise}
{\popQED\endinnerexercise}

% Document
\begin{document}

  \title{\textbf{Measure and Integral}\\ Homework 4\footnote{
    Muhan: 8.11--13, 9.6--9.8.
    Zhixuan: 8.15--17, 9.2, 9.3.
    Heqing: 8.2, 8.8, 8.28, 8.32, 9.4, 9.5.}}
  \author{Luo Muhan, Li Zhixuan, Huang Heqing}
  \maketitle

  % Exercise part
  %=====================================================================
%  \chapter{Preliminaries}
%  \setcounter{chapter}{1}
%
%  \chapter{Functions of Bounded Variation and the Riemann-Stieltjes Integral}
%  
\begin{exercise}{2.14}
  This is the example following (2.28).
  Take $[a,b] = [-1, 1], c=0$, and
  \[
    f(x) = \left\{
    \begin{aligned}
      0 & \quad \text{if} \quad -1 \le x < 0 \\
      1 & \quad \text{if} \quad 0 \le x \le 1,
    \end{aligned}
    \right.
    \quad
    \phi(x) = \left\{
    \begin{aligned}
      0 & \quad \text{if} \quad -1 \le x \le 0 \\
      1 & \quad \text{if} \quad 0 < x \le 1.
    \end{aligned}
    \right.
  \]
  $R_\Gamma(f\rmd \phi; [-1, 1])$ can be 0 or 1 for any $\Gamma$ straddling $0$,
  so $\int_{-1}^1 f\rmd \phi$ does not exist.
  Since $\phi \equiv 0$ on $[-1, 0]$,
  $R_\Gamma(f \rmd \phi; [-1, 0])$ equals to zero for any $\Gamma$
  and hence $\int_{-1}^0 f \rmd \phi = 0$.
  Finally, if $\Gamma = \{x_j\}$ is a partion of $[0, 1]$, then
  \[
    R_\Gamma(f \rmd \phi; [0, 1]) = f(\xi_1) \left( \phi(x_1) - \phi(x_0) \right)
    = 1 \cdot (1-0) = 1.
  \]
  Hence $\int_0^1 f \rmd \phi = 1$.
\end{exercise}

\begin{exercise}{2.15}
  Let $\Gamma = \{x_j\}_1^N$ be a partition of $[a,b]$. Then
  \[
    S_\Gamma = \sum \left\vert \psi(x_j) - \psi(x_{j-1}) \right\vert
    = \sum \left\vert \int_{x_{j-1}}^{x_j} f \rmd \phi \right\vert
    \le \left(\sup |f|\right) \sum V(\phi; [x_{j-1}, x_j])
    \le \left(\sup |f|\right) V(\phi; [a,b]),
  \]
  where the first inequality follows from Theorem 2.24.
  This holds for all $\Gamma$ and thus $\psi$ is of bounded variation.
  If $g$ is continuous on $[a,b]$,
  both $\int_a^b g \rmd \psi$ and $\int_a^b fg \rmd \phi$ exist
  by Theorem 2.24.
  Consider the Riemann-Stieltjes sum for $g \rmd \psi$, we have
  \begin{align*}
    & \sum g(\xi_j) \left( \psi(x_j) - \psi(x_{j-1}) \right) \\
    = & \sum g(\xi_j) f(\eta_j) \left( \phi(x_j) - \phi(x_{j-1}) \right) \\
    = & \sum g(\xi_j) f(\xi_j) \left( \phi(x_j) - \phi(x_{j-1}) \right)
    + \sum g(\xi_j) \left( f(\eta_j) - f(\xi_j) \right)
    \left( \phi(x_j) - \phi(x_{j-1}) \right) \\
    \coloneqq & I_1 + I_2,
  \end{align*}
  where the first equality follows from Theorem 2.27 (Mean-Value Theorem)
  with each $\eta_j \in [x_{j-1}, x_j]$.
  The term $I_1$ is the Riemann-Stieltjes sum for $gf \rmd \phi$
  and tends to $\int_a^b fg \rmd \phi$ as $|\Gamma| \rightarrow 0$.
  Using the uniform continuity of $f$ on $[a,b]$,
  we can select $|\Gamma|$ so small that
  \[
    \left\vert f(\eta_j) - f(\xi_j) \right\vert <
    \frac{\epsilon} {\left( \sup g \right) V(\phi)}
  \]
  for every $1 \le j \le N$.
  It follows that $|I_2| < \epsilon$
  and consequently $\lim_{|\Gamma| \rightarrow 0} I_2 = 0$.
  The proof of $\int_a^b g \rmd \psi = \int_a^b fg \rmd \phi$ is thus completed.

\end{exercise}

\begin{exercise}{2.16}
  In view of Corollary 2.7 (Jordan's Theorem),
  we assume that $\phi$ is monotone increasing.
  Suppose $\sup f = M$ and $\{y_j\}_1^L$ are the jump discontinuities.
  By the continuity of $\phi$ at $\{y_j\}$,
  for each $j$ there exists a closed interval $I_j$ containing $y_j$ such that
  \begin{equation}
    \left| \phi(y) - \phi(z) \right| < \frac{\epsilon}{4 M L}
    \label{eq:estOfPhi}
  \end{equation}
  for any $y, z \in I_j$.
  By the uniform continuity of $f$ on $[a,b] \setminus \bigcup I_j^\circ$,
  select $\delta'$ so small that
  \begin{equation}
    \left| f(y) - f(z) \right| < \frac{\epsilon}{2 V(\phi)}
    \label{eq:estOfF}
  \end{equation}
  for $y,z \in [a,b] \setminus \bigcup I_j^\circ$ and $|y-z| < \delta'$.
  Now suppose $\Gamma = \{x_k\}$ is a partition of $[a,b]$ satisfying
  $\left| \Gamma \right| < \min \left\{ \delta', |I_1|, \cdots, |I_L| \right\}$.
  Divide the closed intervals in $\Gamma$ into two subcollections:
  Let  $I \in \Gamma_1$
  if $I$ has nonempty intersection with $\{y_j\}$
  and let $I \in \Gamma_2$ otherwise.
  We estimate the difference between the upper and lower Riemann-Stieltjes sums as follows.
  First by \eqref{eq:estOfPhi} we have
  \[
    \left| \sum_{[x_{k-1}, x_k] \in I_1}
    \left( M_k - m_k \right) \left( \phi(x_k) - \phi(x_{k-1}) \right) \right|
    \le 2M \cdot L \cdot \frac{\epsilon}{4ML}
    \le \frac{\epsilon}{2}
  \]
  since there are at most $L$ intervals in the subcollection $\Gamma_1$.
  Second by \eqref{eq:estOfF} we have
  \[
    \left| \sum_{[x_{k-1}, x_k] \in I_2}
    \left( M_k - m_k \right) \left( \phi(x_k) - \phi(x_{k-1}) \right) \right|
    \le \frac{\epsilon}{2V(\phi)} \cdot V(\phi)
    \le \frac{\epsilon}{2}.
  \]
  We have shown $\lim_{|\Gamma| \rightarrow 0} (U_\Gamma - L_\Gamma) = 0$.
  It follows from $L_\Gamma \le R_\Gamma \le U_\Gamma$ that
  the integral $\int_a^b f \rmd \phi = \lim R_\Gamma$ exists.
\end{exercise}

\begin{exercise}{2.17}
  In view of Corollary 2.7 (Jordan's Theorem),
  we assume that $\phi$ is monotone increasing.
  Set $S_n = \int_{-n}^n f \rmd \phi$.
  We show that the sequence $\{S_n\}$ is Cauchy.
  Let $\phi(-\infty) = M^-$ and $\phi(+\infty) = M^+$.
  Since $\phi$ is monotone increasing,
  there exists $N_1 > 0$ so large that
  $\phi(x) > M^+ - \sqrt{\epsilon} / 2$ for all $x > N_1$
  and $\phi(x) < M^- + \sqrt{\epsilon} / 2$ for all $x < -N_1$.
  Since $f$ is continuous and $f(\infty) = 0$,
  there exists $N_2 > 0$ so large that
  $|f(x)| < \sqrt{\epsilon}$ for all $|x| > N_2$.
  Now if $n \ge m > \max\{N_1, N_2\}$, we have
  \[
    \begin{aligned}
      \left| S_n - S_m \right| =
      & \left| \int_{-n}^{-m} f \rmd \phi + \int_m^n f \rmd \phi \right| \\
      \le & \left( \sup_{|x| \ge m} |f| \right)
      \left( V(\phi;[-n,-m]) + V(\phi;[m,n]) \right) \\
      \le & \sqrt{\epsilon}
      \left( \phi(-m) - \phi(-\infty) + \phi(+\infty) - \phi(m) \right) \\
      < & \sqrt{\epsilon}
      \left( \frac{\sqrt{\epsilon}}{2} + \frac{\sqrt{\epsilon}}{2} \right) \\
      = &\epsilon,
    \end{aligned}
  \]
  where the first inequality follows from Theorem 2.24.
  Thus the improper integral $\int_{-\infty}^{+\infty} f \rmd \phi = \lim S_n$ exists.
\end{exercise}

\begin{exercise}{2.31}
  For the first part,
  note that the Riemann-Stieltjes sum
  \[
    R_\Gamma(\rmd f; [a,b]) = \sum f(x_j) - f(x_{j-1}) = f(b) - f(a)
  \]
  is independent of the partition $\Gamma$.
  For the second part, we have by the Mean-Value Theorem
  \[
    \sum f(x_j) - f(x_{j-1}) = \sum f'(\xi_j) \left( x_j - x_{j-1} \right)
  \]
  with $\xi_j \in (x_{j-1}, x_j)$.
  Since $f'$ is Riemann integrable,
  by taking the limit $|\Gamma| \rightarrow 0$ of both sides we obtain
  $\int_a^b \rmd f = \int_a^b f' \rmd x$.
\end{exercise}

\begin{exercise}{2.32}
  Since $af(a)$ and $bf(b)$ are finite,
  it follows from Theorem 2.21 (integration by parts) that
  the (Riemann) integrability of $f \rmd x$
  is equivalent to that of $x \rmd f$.
  But $x$ is continuous and $f$ is of bounded variation on $[a,b]$,
  the integrability follows from Theorem 2.24.
\end{exercise}

%
%  \chapter{Lebesgue Measure and Outer Measure}
%  
\begin{exercise}{3.11}
  Suppose $E$ is measurable and $\mea{E} < + \infty$.
  Select an open set $G$ such that $G \supset E$ and $\mea{G-E} < \epsilon$.
  By Theorem 1.11, there exists a countable collection $\{I_k\}$
  of nonoverlapping intervals such that $G = \bigcup I_k$
  and therefore $\mea{G} = \sum \mea{I_k}$.
  Choose $N$ so large that $\sum_{N+1}^{\infty} \mea{I_k} < \epsilon$.
  Now take $S = \bigcup_1^N I_k, N_1 = \bigcup_{N+1}^\infty I_k$
  and $N_2 = G - E$.
  Then $E = (S \cup N_1) - N_2$, $\mea{N_1} < \epsilon$ and $\mea{N_2} < \epsilon$
  as desired.

  Conversely, suppose $E = (S \cup N_1) - N_2$,
  $\outmea{N_1} < \epsilon$ and $\outmea{N_2} < \epsilon$.
  We may assume $N_2 \subset (S \cup N_1)$ by taking appropriate intersection.
  Select an open set $G$ with $G \supset S$ and $\mea{G - S} < \epsilon$.
  Then select an open set $G_1$ with $G_1 \supset N_1$
  and $\mea{G_1} < \outmea{N_1} + \epsilon < 2 \epsilon$.
  By subadditivity of outer measure, we have
  \[
    \begin{aligned}
      \outmea{(G \cup G_1) - E} &= \outmea{(G \cup G_1) - (S \cup N_1) \cup N_2} \\
      &\le \outmea{(G \cup G_1) - (S \cup N_1)} + \outmea{N_2} \\
      &< \mea{G-S} + \outmea{G_1 - N_1} + \epsilon \\
      &< \epsilon + \outmea{G_1} + \epsilon \\
      &< 4\epsilon.
    \end{aligned}
  \]
  We have shown that $E$ can be approximated by open sets from outside.
  Thus $E$ is measurable.
\end{exercise}

\begin{exercise}{3.12}
  First let us fix $E_1$ to be a finite open interval.

  \begin{enumerate}[noitemsep]
  \item
  Since the measure of an interval is equal to its volume,
  the assertion $\mea{E_1 \times E_2} = \mea{E_1} \mea{E_2}$ holds
  for $E_2$ being an open interval.

  \item
  Suppose $E_2$ is an open set.
  By Theorem 1.10,
  we can write $E_2 = \bigcup I_k$
  where $\{I_k\}$ is a collection of disjoint open intervals.
  Then by what has been proved,
  \[
    E_1 \times E_2 = E_1 \times \left( \bigcup I_k \right)
    = \bigcup \left(E_1 \times I_k\right)
  \]
  is a countable union of disjoint measurable sets in $\bbR^2$
  and hence measurable.
  Moreover,
  \[
    \mea{E_1 \times E_2} = \sum \mea{E_1 \times I_k}
    = \mea{E_1} \sum \mea{I_k} = \mea{E_1} \mea{E_2}.
  \]

  \item
  If $\mea{E_2} = 0$, then we can find an open set $G \supset E_2$
  with $\mea{G} < \epsilon$. Then
  $\outmea{E_1 \times E_2} \le \mea{E_1 \times G} < \epsilon \mea{E_1}$
  for arbitrary $\epsilon > 0$.
  Hence $\mea{E_1 \times E_2} = 0$.

  \item
  Suppose $E_2 = \bigcap G_k$ is a $G_\delta$ set with finite measure.
  We may assume $\mea{G_1} < \infty$ and $\{G_k\}$ is decreasing.
  Then by what has been proved,
  \[
    E_1 \times E_2 = E_1 \times \left( \bigcap G_k \right)
    = \bigcap \left( E_1 \times G_k \right)
  \]
  is a countable decreasing intersection of measurable sets
  and hence measurable.
  Moreover,
  \[
    \mea{E_1 \times E_2}
    = \lim \mea{E_1 \times G_k}
    = \mea{E_1} \lim \mea{G_k} = \mea{E_1} \mea{E_2}.
  \]

  \item
  Suppose $E_2$ has finite measure.
  Write $E_2 = G - Z$ where $G$ is a $G_{\delta}$ set with finite measure
  and $Z$ has measure zero. Then by what has been proved,
  \[
    E_1 \times E_2 = E_1 \times \left( G-Z \right)
    = E_1 \times G - E_1 \times Z
  \]
  is the difference of two measurable sets and hence measurable.
  It also follows easily that
  $\mea{E_1 \times E_2} = \mea{E_1}\mea{G} - \mea{E_1}\mea{Z} = \mea{E_1}\mea{E_2}$.

  \item
  Suppose $E_2$ has infinite measure.
  We can write $E_2 = \bigcup E_k$
  where the collection $\{E_k\}$ is increasing and each $\mea{E_k} < \infty$.
  Then $E_1 \times E_2$ is a countable increasing union of measurable sets
  and hence measurable.
  Moreover, $\mea{E_1 \times E_2} = \mea{E_1} \lim {E_k} = \infty$.
  \end{enumerate}

  We have proved that the assertion $\mea{E_1 \times E_2} = \mea{E_1} \mea{E_2}$
  holds for $E_1$ being a finite open interval and $E_2$ an arbitrary measurable set.
  By interchanging the roles of $E_1$ and $E_2$
  and going through the above steps again,
  we can show that the assertion holds for arbitrary measurable $E_1$ and $E_2$.
\end{exercise}

\begin{exercise}{3.13}
  First recall that
  \[
    \outmea{E} = \inf \left\{ \mea{G} : G \text{ is open and } G \subset E\right\}.
  \]
  Since $\mea{F} \le \mea{G}$ for all closed $F \subset E$
  and all open $G \supset E$,
  taking infimum and supremum we obtain
  $\inmea{E} \le \outmea{E}$.
  This proves part (i).

  Suppose $E$ has finite measure.
  Lemma 3.22 states that for an arbitrary $\epsilon > 0$
  there exists a closed set $F \subset E$ with
  $\mea{E} - \mea{F} = \mea{E-F} < \epsilon$.
  It follows easily that $\inmea{E} = \mea{E} = \outmea{E}$.
  Conversely, suppose $\inmea{E} = \outmea{E} < \infty$.
  For each $k \in \bbZ^+$,
  select an open set $G_k$ and a closed set $F_k$ such that
  \[
    F_k \subset E \subset G_k, \quad \mea{G_k} - \mea{F_k} < 1/k.
  \]
  Let $G = \bigcap G_k$.
  Then $G$ is a $G_\delta$ (and hence measurable) set.
  Moreover, it follows from $G - E \subset G_k - F_k$ that
  \[
    \outmea{G-E} \le \mea{G_k - F_k} = \mea{G_k} - \mea{F_k} < 1/k
  \]
  for every $k \in \bbZ^+$.
  Hence $G-E$ has measure zero.
  Finally, being the difference of two measurable sets,
  $E = G - (G-E)$ is measurable.
  This completes the proof of part (ii).
\end{exercise}

\begin{exercise}{3.14}
  Let $E_1$ be a nonmeasurable subset of $[-2, -1]$.
  Let $E_2 = [0, +\infty)$ and $E = E_1 \cup E_2$.
  Clearly $E$ is nonmeasurable.
  Moreover, we have
  \[
    \begin{aligned}
      \outmea{E} \ge \outmea{E_2} = \infty \quad &\Rightarrow \quad \outmea{E} = \infty, \\
      \inmea{E} \ge \mea{[0, n]} = n \text{ for every } n \in \bbZ^+
      \quad &\Rightarrow \quad \inmea{E} = \infty.
    \end{aligned}
  \]
  Hence part (ii) of Exercise 13 is false if $\outmea{E} = \infty$.
\end{exercise}

\begin{exercise}{3.16}
  Recall that a parallelepiped is
  \[
    P = \left\{ \sum_1^n t_k \bfe_k : 0 \le t_k \le 1 \right\}.
  \]
  It is easy to show that $P$ is a compact, and hence measurable set.

  First we show $v(P) \le \mea{P}$.
  Suppose $\{I_k\}$ is a collection of intervals that covers $P$.
  We can enlarge each $I_k$ to some open interval $J_k$ such that
  $I_k$ is contained in $J_k^\circ$ and $v(J_k) < (1+\epsilon) v(I_k)$.
  By the Heine-Borel property,
  there exists a finite cover $\{J_k\}_1^N$ of $P$.
  It follows that
  \[
    v(P) \le \sum_1^N v(J_k) \le (1+\epsilon) \sum_1^N v(I_k)
    \le (1 + \epsilon) \sum_1^\infty v(I_k).
  \]
  Since this holds for all covers $\{I_k\}$ of $P$,
  taking infimum of the right-hand-side we obtain
  $v(P) \le (1 + \epsilon) \mea{P}$.
  But $\epsilon$ is arbitrary,
  so $v(P) \le \mea{P}$.

  To show $\mea{P} \le v(P)$,
  we first claim that for each $\epsilon > 0$
  there exists an open set $G \supset P$
  such that $\mea{G} \le v(P) + \epsilon$.
  If $\{\bfe_k\}$ is linearly independent,
  then $v(P) > 0$ and we can take
  \[
    G = \left\{ \sum_1^n t_k \bfe_k : -\delta < t_k < 1 + \delta \right\}.
  \]
  Then by p.\ 8 in Section 1.3 we know
  %the volume formula for a parallelepiped states that
  $v(G) = (1 + 2\delta)^n v(P) < v(P) + \epsilon$
  for sufficiently small $\delta$.
  Since $G$ is open,
  by Theorem 1.11 $G$ can be written as a countable union
  of the nonoverlapping intervals $\{I_j\}$.
  Then we have $\sum_1^N v(I_j) \le v(G)$ for every $N$;
  taking limit we obtain
  \[
    \mea{G} = \sum v(I_j) \le v(G) \le v(P) + \epsilon.
  \]

  If $\{\bfe_k\}$ is not linearly independent,
  let $U$ be the proper subspace of $\bbR^n$ spanned by $\{\bfe_k\}$
  and let $V$ be the orthogonal complement of $U$.
  %with orthonormal basis. $\{\bff_k\}_1^m$.
  Since $P$ is a compact subset of $U$,
  we can find in the subspace $U$
  a bounded open rotated interval $B_1$ that contains $P$.
  If $B_2$ is an open rotated interval centered at the origin
  in the subspace of $V$,
  then $P$ is contained in the open rotated interval $G = B_1 \times B_2$
  and $\mea{G} < v(P) + \epsilon = \epsilon$ for sufficiently small $B_2$.
  This proves the claim.

  Finally, it follows from the claim that for each $\epsilon > 0$
  we have $\mea{P} \le \mea{G} \le v(P) + \epsilon$ for suitably chosen $G \supset P$.
  Thus $\mea{P} \le v(P)$ and the proof of $\mea{P} = v(P)$ is completed.
\end{exercise}

\begin{exercise}{3.17}
  Let $C$ be the Cantor set and $G = [0, 1] - C$.
  Recall that the Cantor-Lebesgue function $f$ is continuous
  and maps $[0, 1]$ onto $[0, 1]$.
%  By definition we see that
  Observe that
  \[
    f(G) = \left\{ \frac{k}{2^n} : n \in \bbZ^+, 1 \le k \le 2^n-1 \right\}
  \]
  is a countable subset of $[0, 1]$ and therefore has measure zero.
  Let $E$ be a nonmeasurable subset of $[0, 1] - f(G)$.
  Since $E \cap f(G) = \emptyset$,
  we must have $f^{-1}(E) \subset C$.
  Since $\mea{C} = 0$,
  we see that $f^{-1}(E)$ has measure zero.
  But by construction $E = f (f^{-1}(E))$ is nonmeasurable.
\end{exercise}

\begin{exercise}{3.18}
  Suppose $E \subset \bbR^n$ and $\bfh \in \bbR^n$.
%  We show that $\outmea{E_{\bfh}} \le \outmea{E}$.
  Suppose $\{I_k\}$ is a cover of $E$ by intervals and
  $\sum v(I_k) < \outmea{E} + \epsilon$.
  Then $\{I_{k, \bfh}\}$ is a cover of $E_{\bfh}$ and
  \[
    \outmea{E_{\bfh}} \le \sum v(I_{k, \bfh}) = \sum v(I_k) < \outmea{E} + \epsilon,
  \]
  since the volume of an interval is translation-invariant.
  But $\epsilon$ is arbitrary, so $\outmea{E_{\bfh}} \le \outmea{E}$.
  By symmetry we have $\outmea{E} \le \outmea{E_{\bfh}}$.

  If $E$ is measurable, for each $\epsilon > 0$
  there exists an open set $G$ such that $G \supset E$ and $\mea{G-E} < \epsilon$.
  But $G_{\bfh}$ is open,
  $G_{\bfh} \supset E_{\bfh}$
  and $\outmea{G_{\bfh} - E_{\bfh}} = \mea{G - E} < \epsilon$
  by what has been proved.
  Therefore $E_{\bfh}$ is measurable.
\end{exercise}

To complete Exercise 19,
we extend Lemma 3.37 to multidimensions.

\begin{lemma}
  Let $E \subset \bbR^n$ be a measurable set with $\mea{E} > 0$.
  Then the set of differences
  $\{d : d = x-y, x \in E, y \in E\}$
  contains an interval centered at the origin.
\end{lemma}
\begin{proof}
  Given $\epsilon > 0$, there exists an open set $G$
  such that $G \supset E$ and $\mea{G} \le (1+\epsilon) \mea{E}$.
  By Theorem 1.11, we can write $G = \bigcup I_k$
  where $\{I_k\}$ are nonoverlapping intervals
  and each $I_k$ has equal sides.
  Set $E_k = E \cap I_k$.
  Then $\mea{G} = \sum \mea{I_k}$ and $\mea{E} = \sum \mea{E_k}$,
  and there must exist at least one $\calI \in \{I_k\}$ and one $\calE \in \{E_k\}$
  such that $\mea{\calI} \le (1+\epsilon) \mea{\calE}$.
  Suppose $\calI$ has side length $l > 0$.
  We claim that $\calE \bigcap \calE_{\bfh} \neq \emptyset$
  for all sufficiently small $|\bfh|$.

  If $\calE \bigcap \calE_{\bfh} = \emptyset$ for some $\bfh$
  with $d = |\bfh| > 0$ and $d \le \sqrt{n} l$,
  then we must have $\mea{\calI \cup \calI_{\bfh}} \ge 2 \mea{\calE}$
  since the left-hand-side contains both $\calE$ and $\calE_{\bfh}$.
  Observe that %the volume of the union of the intervals
  \[
    \mea{\calI \cup \calI_{\bfh}} \le l^n + l^n - \left( l - \frac{d}{\sqrt{n}} \right)^n
    = \left[ 2 - \left( 1 - \frac{d}{\sqrt{n} l} \right)^n \right] l^n,
  \]
  where the maximum is attained when
  $\bfh = \left(d / \sqrt{n}, \cdots, d / \sqrt{n}\right)$.
  Combining the estimate for $\mea{\calE}$, we obtain
  \[
    \frac{2}{1 + \epsilon} \le  2 - \left( 1 - \gamma \right)^n,
  \]
  where $\gamma = d / \sqrt{n} l \in (0, 1]$.
  However, this inequality must be violated
  by sufficiently small $\epsilon$ and $\gamma$.
  Therefore, there exists a sufficiently small $d > 0$
  such that $\calE \cap \calE_{\bfh} \neq \emptyset$
  for all $|\bfh| < d$.
  This completes the proof of the Lemma.
\end{proof}

\begin{exercise}{3.19}
  In $\bbR^n$ we define an equivalence relation by saying
  $x \sim y$ if $x - y \in \bbQ^n$;
  that is, $x \sim y$ if every component of $x-y$ is rational.
  Using Axiom of Choice,
  we construct a set $E$ that consists of exactly one element
  from each equivalence class.
  If the set of differences of $E$ contains an interval centered at the origin,
  it must contain a nonzero number in $\bbQ^n$,
  which is impossible by the construction of $E$.
  By the preceding Lemma,
  either $E$ is nonmeasurable or $E$ has measure zero.

  Denote by $E_q$ the translate of $E$ by a rational number $q$.
  Then $E_q \cap E_r \neq \emptyset$ if $q$ and $r$ are distinct rational numbers.
  Since $\bbR^n$ is the countable disjoint union of all $E_q$
  where $q$ ranges over $\bbQ^n$,
  $\bbR^n$ would have measure zero if $E$ did.
  Thus $E$ is nonmeasurable.
\end{exercise}



  % Homework part
  %=====================================================================
%  
\begin{exercise}{1.15}
  By Exercise 1.1 (r),
  we show that $\inf_{\Gamma} U_\Gamma = \sup_{\Gamma} L_\Gamma = A$
  if and only if for each $\epsilon > 0$,
  there exists a partition $\Gamma$ such that $0 \le U_\Gamma - L_\Gamma < \epsilon$.

  ($\Longleftarrow$) We have
  \[
    0 \le \inf U_\Gamma - \sup L_\Gamma \le U_\Gamma - L_\Gamma < \epsilon
  \]
  for each $\epsilon > 0$.
  It follows immediately that $\inf U_\Gamma = \sup L_\Gamma$.

  ($\Longrightarrow$) For each $\epsilon > 0$,
  select a partition $\Gamma_1$ such that $U_{\Gamma_1} < A + \epsilon/2$
  and also a partition $\Gamma_2$ such that $L_{\Gamma_2} > A - \epsilon/2$.
  Let $\Gamma = \Gamma_1 \cup \Gamma_2$ be the common refinement of $\Gamma_1$ and $\Gamma_2$.
  It follows that
  \[
    0 \le U_{\Gamma} - L_{\Gamma} \le U_{\Gamma_1} - L_{\Gamma_2}
    < \left( A + \frac{\epsilon}{2} \right) - \left( A - \frac{\epsilon}{2} \right) = \epsilon.
    \qedhere
  \]
\end{exercise}

\begin{exercise}{1.20}
  For the first part, if $E_q = \left\{ q \right\}^C$, then
  \[
    \bbR - \bbQ = \bigcap_{q \in \bbQ} E_q
  \]
  is a $G_\delta$ set since each $E_q$ is open (and also dense).
  For the second part,
  suppose $\bbQ = \bigcap_n G_n$ is a $G_\delta$ set.
  Each $G_n$ is dense in $\bbR$ since $\bbQ \in G_n$ and $\bbQ$ is.
  It follows that
  \[
    \emptyset = \left( \bbR - \bbQ \right) \cap \bbQ
    = \left( \bigcap_{q \in \bbQ} E_q \right) \bigcap \left( \bigcap_n G_n \right)
  \]
  is a countable intersection of open dense subsets of $\bbR$,
  which is dense again by Exercise 1.19.
  But an empty set cannot be dense in $\bbR$.
\end{exercise}

\begin{exercise}{2.5}
  First observe that $f(x) \le |f(b)| + M$ for every $x \in (a, b]$.
  It follows that $L = \max \left\{ |f(b)| + M, f(a) \right\}$ is a bound of $f$.
  Suppose $\Gamma = \left\{ x_j \right\}_0^k$ is a partition of $[a, b]$.
  Then we have
  \[
    S_{\Gamma} = \left| f(x_1) - f(a) \right| +
    \sum_{j = 1}^{k-1} \left| f(x_{j+1}) - f(x_j) \right|
    \le 2L + V[f; x_1,b]
    \le 2L + M.
  \]
  Taking supremum we see that $V[f;a,b] \le 2L + M < +\infty$
  and $f$ is of bounded variation.

%  However, it does not necessarily follow that $V[f; a,b] \le M$.
  Assume further that $f$ is right-hand continuous at $a$.
  For each $\epsilon > 0$, there exists a $\xi$ such that
  $a < \xi < x_1$ and $|f(\xi) - f(a)| < \epsilon$.
  If $\Gamma'$ is the partition formed by inserting $\xi$ into $\Gamma$,
  we have
  \[
    S_\Gamma \le S_{\Gamma'} \le \left| f(\xi) - f(a) \right| + V[f; \xi, b]
    < \epsilon + M.
  \]
  Since $\epsilon$ is arbitrary, we have $S_{\Gamma} \le M$ for every $\Gamma$.
  Taking supremum, we obtain $V[f;a,b] \le M$.
\end{exercise}

\begin{exercise}{2.15}
  Let $\Gamma = \{x_j\}_0^k$ be a partition of $[a,b]$.
  Then
  \[
    S_\Gamma = \sum \left\vert \psi(x_j) - \psi(x_{j-1}) \right\vert
    = \sum \left\vert \int_{x_{j-1}}^{x_j} f \rmd \phi \right\vert
    \le \left(\sup |f|\right) \sum V[\phi; x_{j-1}, x_j]
    \le \left(\sup |f|\right) V[\phi; a,b],
  \]
  where the first inequality follows from Theorem 2.24.
  This holds for all $\Gamma$ and therefore $\psi$ is of bounded variation.
  If $g$ is continuous on $[a,b]$,
  both $\int_a^b g \rmd \psi$ and $\int_a^b fg \rmd \phi$ exist
  by Theorem 2.24.
  Consider the Riemann-Stieltjes sum for $g \rmd \psi$, we have
  \begin{align*}
    & \sum g(\xi_j) \left( \psi(x_j) - \psi(x_{j-1}) \right) \\
    = & \sum g(\xi_j) f(\eta_j) \left( \phi(x_j) - \phi(x_{j-1}) \right) \\
    = & \sum g(\xi_j) f(\xi_j) \left( \phi(x_j) - \phi(x_{j-1}) \right)
    + \sum g(\xi_j) \left( f(\eta_j) - f(\xi_j) \right)
    \left( \phi(x_j) - \phi(x_{j-1}) \right) \\
    = & I_1 + I_2,
  \end{align*}
  where the first equality follows from Theorem 2.27 (Mean-Value Theorem)
  with each $\eta_j \in [x_{j-1}, x_j]$.
  The term $I_1$ is the Riemann-Stieltjes sum for $gf \rmd \phi$
  and tends to $\int_a^b fg \rmd \phi$ as $|\Gamma| \rightarrow 0$.
  Using the uniform continuity of $f$ on $[a,b]$,
  we can select $|\Gamma|$ so small that
  \[
    \left\vert f(\eta_j) - f(\xi_j) \right\vert <
    \frac{\epsilon} {\left( \sup g \right) V[\phi; a,b]}
  \]
  for every $0 \le j < k$.
  It follows that $|I_2| < \epsilon$
  and consequently $\lim_{|\Gamma| \rightarrow 0} I_2 = 0$.
  The proof of $\int_a^b g \rmd \psi = \int_a^b fg \rmd \phi$ is thus completed.
\end{exercise}

\begin{exercise}{2.20}
  Clearly $S \coloneqq \lim_{|\Gamma| \rightarrow 0} S_\Gamma[f;a,b] \le V[f;a,b]$
  if the limit exists.
  To show the reverse inequality,
  first select a partition $\Gamma'$ such that
  $S_{\Gamma'} > V[f;a,b] - \epsilon/2$.
  Then select a $\delta > 0$ such that
  $|S_\Gamma - S| < \epsilon/2$ whenever $|\Gamma| < \delta$.
  Let $\Gamma^\ast$ be the refinement of $\Gamma'$
  subject to $|\Gamma^\ast| < \delta$.
  Then we have $S_{\Gamma'} \le S_{\Gamma^\ast}$, and
  \[
    V[f;a,b] - S < V - S_{\Gamma^\ast} + \epsilon/2
    \le V - S_{\Gamma'} + \epsilon/2
    < \epsilon/2 + \epsilon/2 = \epsilon.
  \]
  Since $\epsilon > 0$ is arbitrary,
  we have $V[f;a,b] \le S$.
\end{exercise}

\begin{exercise}{3.1}
  Suppose $b > 1$ is an integer and $0 \le x < 1$.
  We determine the integral sequence $c_k$ as follows.
  Let $c_1 = \lfloor bx \rfloor$, i.e.\
  the greatest integer less than or equal to $bx$.
  Then $0 \le c_1 < b$ and $0 \le bx - c_1 < 1$.
  Having determined $c_1, \cdots, c_{k-1}$,
  we determine $c_k$ by choosing
  \[
    c_k = \lfloor b^k x - \sum_{j=1}^{k-1} c_j b^{k-j} \rfloor.
  \]
  Since by the induction hypothesis $0 \le b^{k-1} x - \sum_1^{k-1} c_j b^{k-1-j} < 1$,
  we have $0 \le c_k < b$ and also
  \begin{equation}
    0 \le b^k x - \sum_1^{k} c_j b^{k-j} < 1.
    \label{eq:base_b_sandwich}
  \end{equation}
  This completes the induction step and
  hence the determination of the sequence $c_k$.
  Dividing by $b^{-k}$ in~\eqref{eq:base_b_sandwich}
  and letting $k \rightarrow \infty$,
  we obtain $x = \sum_1^{\infty} c_j b^{-j}$.
  In the exceptional case of $x = 1$,
  we shall choose $c_j = b-1$ for all $j \ge 1$.

  Suppose $\{c_j\}$ and $\{d_j\}$ are integral sequence
  such that $0 \le c_j, d_j < b$ and
  $\sum_{1}^{\infty} (c_j - d_j) b^{-j} = 0$.
  Let $k$ be the smallest index such that $c_k \neq d_k$.
  If no such $k$ exists, then $\{c_j\}$ are $\{d_j\}$ are identical sequences.
  Otherwise, we have
  \[
    b^{-k} \le \left| \left( c_k - d_k \right) b^{-k} \right|
    = \left| \sum_{j=k+1}^{\infty} \left( d_j - c_j \right) b^{-j} \right|
    \le \sum_{j=k+1}^{\infty} (b-1) b^{-j}
    = b^{-k}
  \]
  since $c_k \neq d_k$ and $|d_j - c_j| \le b-1$ for all $j \ge k+1$.
  It follows that the inequalities are in fact equalities,
  and this is the case if and only if
  $c_k - d_k = 1$ and $d_j - c_j = b-1 \Leftrightarrow d_j = b-1, c_j = 0$
  for all $j \ge k+1$ (or vice versa).
  It also follows that
  \[
    x = \sum_{j=1}^k c_j b^{-j} = c b^{-k},
  \]
  where $1 \le c = \sum_1^k c_j b^{k-j} \le b^k-1$.
\end{exercise}

\begin{exercise}{3.9}
  For each $\epsilon > 0$,
  select $n \in \bbZ^+$ so large that $\sum_n^{\infty} \outmea{E_k} < \epsilon$.
  Then $\limsup E_k \subset \cup_{k=n}^{\infty} E_k$,
  and by countable subadditivity we have
  \[
    \outmea{\limsup E_k} \le \outmea{ \cup_{k=n}^{\infty} E_k}
    \le \sum_{k=n}^{\infty} \outmea{E_k} < \epsilon.
  \]
  Since $\epsilon$ is arbitrary,
  we see that $\limsup E_k$ has measure zero,
  and so does $\liminf E_k$ since it is a subset of the former.
\end{exercise}

  \setcounter{section}{7}
%  \section{Lebesgue Measurable Functions}

\begin{exercise}{4.3}
  If $F$ is measurable as defined, then the sets
  \begin{align*}
    &F^{-1}((a,+\infty)\times\mathbb{R})=\{f>a\}
  \end{align*}
  and
  \begin{align*}
    &F^{-1}(\mathbb{R}\times(b,+\infty))=\{g>b\}
  \end{align*}
  are measurable for any $a,b\in\mathbb{R}.$
  Therefore $f$ and $g$ are measurable.
  Conversely, let $G$ be any open set in $\mathbb{R}^2$,
  then there exist partly open intervals $I_i$ and $J_i$ such that
  \[G=\bigcup_{i=1}^\infty I_i\times J_i.\]
  Then \[F^{-1}(G)=\bigcup_{i=1}^\infty F^{-1}(I_i\times J_i)=\bigcup_{i=1}^\infty f^{-1}(I_i)\cap g^{-1}(J_i).\]
  Since $f$ and $g$ are measurable,
  we have $f^{-1}(I_i)$ and $g^{-1}(J_i)$ are measurable sets.
  Thus $F^{-1}(G)$ is a measurable set and $F$ is measurable.
\end{exercise}

\begin{exercise}{4.5}
  We consider the equivalence relation on $[0,1]$
  defined by $x\sim y\Leftrightarrow x-y\in\mathbb{Q}$.
  Let $E$ be a set consisting exactly one element from each equivalence class
  and in particular let $0\in E$.
  Thus no nonzero rational numbers belong to $E$ and we also know that $E$ is nonmeasurable.
  Let $F$ be the Cantor-Lebesgue function and $Z=F^{-1}(E)$.
  By definition of $F$, elements outside the Cantor set are mapped to nonzero rational values by $F$,
  so we must have $Z\subset C$, the Cantor set.
  Thus $|Z|\leq |C|=0.$
  Define $\phi=\chi_Z$, the characteristic function of $Z$.
  Since $Z$ is a measurable set,
  $\phi$ is a measurable function.
  Now we define \[f(y)=\inf\{x\in[0:1]:x\in F^{-1}(y)\}. \]
  Then since $F$ is continuous and monotone,
  we have $F(f(y))=y$ and $\{f>a\}$ is an interval.
  We deduce that $f$ is also measurable.
  Now for any $y\in E$, we have \[y\in E\Rightarrow f(y)\in F^{-1}(y)\subset F^{-1}(E)=Z.\]
  On the other hand if $f(y)\in Z$, we have $y=F(f(y))\in F(Z)=F(F^{-1}(E))=E$.
  The last equality comes from the fact that $F$ is surjective from $[0,1]$ to $[0,1]$.
  Therefore we have \[f(y)\in Z\Leftrightarrow y\in E.\]
  Now \[(\phi\circ f)^{-1}(\{1\})=\{y:f(y)\in Z\}=E.\]
  Therefore $\phi\circ f$ is not measurable.

  Now if $g(x)=x+F(x)$, we notice that $[0,1]\setminus C$ is a disjoint union of open intervals $\{I_i\}$
  and $F$ is constant on each of the intervals.
  Since $g$ is strictly monotone, it follows that $g(I_i)$ are still disjoint intervals with the same measures as $I_i$.
  Thus \[|g([0,1]\setminus C)|=\sum_i|g(I_i)|=\sum_i|I_i|=|[0,1]\setminus C|=1.\]
  Therefore since $g$ is strictly monotone,
  we have \[|g(C)|=|[0,2]-g([0,1]\setminus C)|=2-1=1.\]
  By Corollary 3.39 there exists a nonmeasurable set $E$ in $g(C)$.
  Let $Z=g^{-1}(E)$.
  Then $Z$ is a subset of $C$ and is a measurable set of measure zero.
  Define $\phi=\chi_Z$ and $f=g^{-1}$, then $\phi$ is measurable and $f$ is continuous.
  However, we have
  \begin{align*}
  (\phi\circ f)^{-1}(\{1\})=\{y\in[0,2]:f(y)\in Z\}=\{y:y\in g(Z)\}=g(Z)=E.
  \end{align*}
  Therefore $\phi\circ f$ is not measurable.
\end{exercise}

\noindent\textbf{Exercise 4.12. }

\begin{proof} (Constructive proof) Suppose $f$ is continuous at almost every point of an interval $[a,b]$ and so $f$ is finite a.e.\ on $[a,b]$.
Let $\{\Gamma_k\}$ be a sequence of partitions of $[a,b]$ with norms $|\Gamma_k|$ tending to zero.
For each $k$, define a simple measurable function $l_k$ as follows:
if $x_1^{(k)}<x_2^{(k)}<\cdots$ are the partitioning points of $\Gamma_k$,
let $l_k(x)$ be defined in each semiopen interval $[x_i^{(k)},x_{i+1}^{(k)})$ as
$\max\{\inf_{[x_i^{(k)},x_{i+1}^{(k)}]}f,-k\}$ which is finite.
Suppose $f$ is continuous at $x\in[a,b)$, then for a given $\epsilon>0$,
there is a neighborhood $[a_x,b_x]$ of $x$ such that $|f(x)-f(y)|\leq \epsilon$
whenever $y\in [a_x,b_x]$.
Then we can choose a $K_1$ sufficiently large such that $f(y)> -K_1$ for each $y\in [a_x,b_x]$.
Since $|\Gamma_k|\to 0$, there is a $K_2>K_1$ such that for any $k\geq K_2$,
$x\in [x_{i_k}^{(k)},x_{{i_k}+1}^{(k)})\subset [a_x,b_x]$ for some $i_k$.
Then $|f(x)-l_k(x)|=|f(x)-\inf_{[x_{i_k}^{(k)},x_{i_{k}+1}^{(k)}]}f|\leq\epsilon$.
This proves that $l_k$ converges a.e.\ to $f$ and
thus that $f$ is measurable on $[a,b]$ by Theorem 4.12.
\end{proof}

Now we generalize this to functions defined in $\mathbf{R^n}$ by using a non-constructive way.

\begin{proof}(Non-constructive proof)
  Suppose $f$ is continuous a.e.\ on a measurable set $E\subset \mathbf{R^n}$,
  then we can write $E$ as a disjoint union $E=E_1\cup E_2$,
  where $f$ is continuous on $E_1$ and $|E_2|=0$.
  Note that the fact $f$ is continuous a.e.\ on $E$ implies
  $\{\bfx\in E:f(x)=+\infty\}$ is a set of measure zero and hence is measurable.
  To prove $f$ is measurable on $E$, by Corollary 4.2,
  we only need to show $\{\bfx\in E: a\leq f(\bfx)<+\infty\}$ is measurable for every finite $a$.
  Note that \[\{\bfx\in E: a\leq f(\bfx)<+\infty\}=\{\bfx\in E_1: a\leq f(\bfx)<+\infty\}\cup\{\bfx\in E_2: a\leq f(\bfx)<+\infty\}\eqqcolon A\cup B,\]
  where $B$ is a set of measure zero and hence is measurable.
  Let $\bfx_0$ be a limit point of $A$ that lies in $E_1$.
  Then there exist $\bfx_k\in E_1$ such that $\bfx_k\to \bfx_0$ and $a\leq f(\bfx_k)<+\infty$.
  Since $f$ is continuous at $\bfx_0\in E_1$,
  we have $a\leq \lim_{k\to\infty}f(\bfx_k)=f(\bfx_0)<+\infty$
  and then $\bfx_0\in A$.
  By Theorem 1.8, it follows that $A$ is relatively closed with respect to $E_1$ and then $A=E_1\cap F$ for some closed $F$.
  Hence $A$ is measurable and so is $\{\bfx\in E: a\leq f(\bfx)<+\infty\}$.
\end{proof}

\begin{exercise}{4.15}
  Define
  \[
    E_n = \left\{ x \in E : \lvert f_k(x) \rvert \le n
    \text{ for all $k$} \right\}
    = \bigcap_{k=1}^{\infty}
    \left\{ x \in E : \lvert f_k(x) \rvert \le n \right\}.
  \]
  Note that $E_n$ is measurable
  since $f_k$ is measurable
  and so is $\left\{ x \in E : \lvert f_k(x) \rvert \le n \right\}$.
  For each $x \in E$,
  we have $x \in E_n$ for all $n \ge M_x$,
  so $E = \bigcup E_n$.
  Since $E_n$ is increasing,
  we have $\lim \mea{E_n} = \mea{E} < +\infty$.
  Therefore, we can select $M \in \bbZ^+$ so large that
  $\mea{E - E_M} = \mea{E} - \mea{E_M} < \epsilon/2$,
  and in turn select a closed set $F \subset E_M$ such that
  $\mea{E_M - F} < \epsilon/2$.
  It follows that $\mea{E - F} = \mea{E - E_M} + \mea{E_M - F} < \epsilon$,
  and for each $x \in F$ we have
  $\lvert f_k(x) \rvert \le M$ for all $k$.
\end{exercise}

\begin{exercise}{4.16}
  Recall that $f_k \xrightarrow{m} f$ if for every $\epsilon > 0$,
  \begin{equation}
    \lim_{k \rightarrow \infty} \mea{\left\{ \lvert f - f_k \rvert > \epsilon \right\}} = 0.
    \label{eq:defOfCvgInMeasure}
  \end{equation}
  Assume $f_k \xrightarrow{m} f$.
  Given $\epsilon > 0$,
  there exists, according to~\eqref{eq:defOfCvgInMeasure}, a $K$ such that
  $\mea{\left\{ \lvert f - f_k \rvert > \epsilon \right\}} < \epsilon$
  for all $k > K$.

  Conversely, assume for each $\epsilon > 0$ there exists a $K$ such that
  $\mea{\left\{ \lvert f - f_k \rvert > \epsilon \right\}} < \epsilon$
  for all $k > K$.
  Then for each $\epsilon > 0$ and each $\epsilon_1 > 0$,
  we can take $\delta = \min \left\{ \epsilon, \epsilon_1 \right\} > 0$,
  and by assumption there exists a $K$ such that
  $\mea{\left\{ \lvert f - f_k \rvert > \delta \right\}} < \delta$
  for all $k > K$.
  Then
  \[
    \mea{\left\{ \lvert f - f_k \rvert > \epsilon \right\}}
    \le \mea{\left\{ \lvert f - f_k \rvert > \delta \right\}}
    < \delta \le \epsilon_1
  \]
  for all $k > K$, which shows exactly $f_k \xrightarrow{m} f$.

  An analogous Cauchy criterion is that
  $\{f_k\}$ is Cauchy in measure if and only if
  for each $\epsilon > 0$,
  there exists a $K$ such that
  \[
    \mea{\left\{ \lvert f_j - f_k \rvert > \epsilon \right\}} < \epsilon
  \]
  for all $j, k > K$.
\end{exercise}

\begin{lemma}
  If $g$ is finite a.e.\ on $E$ and $|E|<+\infty$,
  then for each $\epsilon > 0$
  there exist a subset $F$ of $E$ and a sufficiently large $M$ such that
  $\mea{E - F} < \epsilon$ and $\lvert g \rvert \le M$ on $F$.
\end{lemma}
\begin{proof}
  This is a special case of Exercise 4.15.
\end{proof}

\begin{exercise}{4.17}
  Suppose $f_k \xrightarrow{m} f$ and $g_k \xrightarrow{m} g$ on $E$.
  First we show $f_k + g_k \xrightarrow{m} f+g$.
  Given $\epsilon > 0$,
  we can select $K$ so large that
  \[
    \mea{\left\{ \lvert f - f_k \rvert > \frac{\epsilon}{2} \right\}} < \frac{\epsilon}{2},
    \quad
    \mea{\left\{ \lvert g - g_k \rvert > \frac{\epsilon}{2} \right\}} < \frac{\epsilon}{2},
  \]
  for all $k \ge K$.
  Since $\lvert (f_k + g_k) - (f+g) \rvert \le
  \lvert f_k - f \rvert + \lvert g_k - g \rvert$,
  we have
  \[
    \mea{\left\{ \lvert (f_k + g_k) - (f+g) \rvert > \epsilon \right\}} \le
    \mea{\left\{ \lvert f_k - f \rvert > \frac{\epsilon}{2} \right\}}
    + \mea{\left\{ \lvert g_k - g \rvert > \frac{\epsilon}{2} \right\}}
    < \frac{\epsilon}{2} + \frac{\epsilon}{2} = \epsilon
  \]
  for all $k \ge K$, which shows that $f_k + g_k \xrightarrow{m} f + g$.

  Suppose further that $\mea{E} < +\infty$.
  For each $\epsilon > 0$,
  we first invoke the preceding Lemma to select $M$ such that
  \[
    \mea{\left\{ \lvert f \rvert > M \right\}} < \epsilon/6, \quad
    \mea{\left\{ \lvert g \rvert > M \right\}} < \epsilon/6.
  \]
  Let $\delta = \min\left\{ \epsilon/3, \epsilon/3M, 1 \right\}$.
  By convergence in measure we can select $K$ so large that
  \[
    \mea{\left\{ \lvert f_k - f \rvert > \delta \right\}} < \epsilon/6, \quad
    \mea{\left\{ \lvert g_k - g \rvert > \delta \right\}} < \epsilon/6,
  \]
  for all $k \ge K$.
  Then we have
  \[
    \mea{\left\{ \lvert f_k - f)(g_k - g) \rvert > \epsilon/3 \right\} }
    \le \mea{\left\{ \lvert f_k - f \rvert > \epsilon/3 \right\}}
    + \mea{\left\{ \lvert g_k - g \rvert > 1 \right\}}
    < \epsilon/6 + \epsilon/6 = \epsilon/3.
  \]
  Moreover, we have
  \[
    \mea{\left\{ \lvert f (g_k - g) \rvert > \epsilon/3 \right\}}
    \le \mea{\left\{ \lvert f \rvert > M \right\}}
    + \mea{\left\{ \lvert g_k - g \rvert > \frac{\epsilon}{3M} \right\}}
    < \epsilon/6 + \epsilon/6 = \epsilon/3,
  \]
  and the same argument suggests
  \[
    \mea{\left\{ \lvert g (f_k - f) \rvert > \epsilon/3 \right\}} < \epsilon/3.
  \]
  Writing $f_k g_k - fg = (f_k - f)(g_k - g) + f(g_k - g) + g(f_k - f)$,
  we obtain
  \[
    \begin{aligned}
      \mea{\left\{ \lvert f_k g_k - fg \rvert \right\} > \epsilon}
      \le &\mea{\left\{ \lvert f_k - f)(g_k - g) \rvert > \epsilon/3 \right\} } \\
      + &\mea{\left\{ \lvert f (g_k - g) \rvert > \epsilon/3 \right\}} \\
      + &\mea{\left\{ \lvert g (f_k - f) \rvert > \epsilon/3 \right\}} \\
      < &\epsilon
    \end{aligned}
  \]
  for all $k \ge K$, which shows that $f_k g_k \xrightarrow{m} fg$.

  Finally suppose $g_k \rightarrow g$ on $E$ and $g \neq 0$ a.e.
  Since $\mea{E} < +\infty$, $1/g_k \rightarrow 1/g$ and $1/g$ is finite a.e.,
  Theorem 4.21 suggests that $1/g_k \xrightarrow{m} 1/g$.
  It follows immediately from the last case that $f_k/g_k \xrightarrow{m} f/g$.
\end{exercise}

\begin{exercise}{4.18}
  For the first part, fix $a \in (-\infty, +\infty)$.
  For each $x \in E$,
  it follows from $f_k \nearrow f$ that
  $f(x) > a$ if and only if $f_k(x) > a$ for some $k$.
  That is, $\left\{ f > a \right\} = \bigcup_{k=1}^{\infty} \left\{ f_k > a \right\}$.
  Since $\left\{ f_k > a \right\}$ is increasing,
  we see that $\omega_{f_k}(a) = \mea{\left\{ f_k > a \right\}} \nearrow
  \mea{\left\{ f > a \right\}} = \omega_f(a)$ as $k \rightarrow \infty$.

  Now suppose $f_k \xrightarrow{m} f$.
  For each $\epsilon > 0$, we have
  \[
    \omega_f(a + \epsilon) = \mea{\left\{ f > a + \epsilon \right\}} \le
    \mea{\left\{ f_k > a \right\}} + \mea{\left\{ \lvert f_k - f \rvert > \epsilon \right\}}
    = \omega_{f_k}(a) + \mea{\left\{ \lvert f_k - f \rvert > \epsilon \right\}}.
  \]
  But $f_k \xrightarrow{m} f$ implies that
  $\mea{\left\{ \lvert f_k - f \rvert > \epsilon \right\}} \rightarrow 0$
  as $k \rightarrow \infty$;
  taking limit inferior of the last inequality we obtain
  \[
    \omega_f(a + \epsilon) \le \liminf_{k \rightarrow \infty} \omega_{f_k}(a).
  \]
  Since $\epsilon > 0$ is arbitrary,
  by letting $\epsilon \rightarrow 0^+$ we see that
  $\omega_f(a) \le \liminf \omega_{f_k}(a)$
  provided $\omega_f$ is continuous at $a$.
  A similar argument shows that $\limsup \omega_{f_k}(a) \le \omega_f(a)$.
  We conclude that $\lim \omega_{f_k}(a) = \omega_f(a)$
  provided $\omega_f$ is continuous at $a$.
\end{exercise}

%  \section{The Lebesgue Integral}

\begin{exercise}{5.4}
  For each $k=1,2,\cdots$, the continuous function $x^k$ is measurable,
  so is $x^kf(x)$ (Theorem 4.10).
  Since $[x^kf(x)]^\pm=x^kf^\pm(x)\leq f^\pm(x)$ on $(0,1)$,
  the existence of the integral $\int_0^1x^kf(x)\rmd x$ follows from Theorem 5.5 (i).
  Note that
  \[\left|\int_0^1 x^kf(x)\rmd x\right|\leq\int_0^1\left|x^kf(x)\right|\rmd x\leq\int_0^1|f(x)|\rmd x<\infty\]
  as $f\in L(0,1)$.
  Hence $x^kf(x)\in L(0,1)$ for $k=1,2,\cdots$.

  Since $f\in L(0,1)$, it follows that $f$ is finite a.e.\ in $(0,1)$ (Theorem 5.22)
  and then $x^kf(x)\to 0$ a.e.\ in $(0,1)$.
  Moreover, $|x^kf(x)|\leq |f(x)|$ in $(0,1)$ for all $k$.
  By Lebesgue's dominated convergence theorem (Theorem 5.36),
  we have $\int_0^1x^kf(x)\rmd x\to0$.
\end{exercise}

\begin{exercise}{5.5}
  Let $\{f_k\}$ be a sequence of measurable functions on $E$ such that
  $f_k\to f$ a.e.\ in $E$ (so $f$ is measurable by Theorem 4.12).
  Suppose $|E|<+\infty$ and there is a finite constant $M$
  such that $|f_k|\leq M$ a.e.\ in $E$ (so $f_k\in L(E)$).
  We want to prove $\int_Ef_k\to \int_E f$ by using Egorov's theorem.

  We note that $|f|\leq M$ a.e.\ on $E$ and hence it is also finite a.e.
  Since $|E|<+\infty$, it follows that $f\in L(E)$.
  Given $\epsilon>0$, by Egorov's theorem,
  there is a closed subset $F$ of $E$ such that
  $|E-F|<\epsilon/(4M)$ and $\{f_k\}$ converges uniformly to $f$ on $F$.
  If $|E|=0$, then the result follows from Theorem 5.25.
  If $|E|\neq 0$, then by the uniform convergence of $\{f_k\}$ in $F$,
  we have $|f_k(x)-f(x)|\leq \epsilon/(2|E|)$ for all $x\in F$ if $k$ is sufficiently large.
  From Theorem 5.28 and (5.20), we obtain
  \begin{equation*}
    \begin{aligned}
      \left|\int_E f_k-\int_Ef\right|&=	\left|\int_E (f_k-f)\right|\leq \int_E|f_k-f|\\&=\int_{E-F}|f_k-f|+\int_F|f_k-f|\\&\leq\int_{E-F}(|f_k|+|f|)+\int_F\frac{\epsilon}{2|E|}\\
      &\leq 2M|E-F|+\frac{\epsilon}{2|E|}|F|\\&\leq\frac{\epsilon}{2}+\frac{\epsilon}{2}=\epsilon
    \end{aligned}
  \end{equation*}
  provided $k$ is sufficiently large.
  Since $\epsilon>0$ is arbitrary, we have $\int_Ef_k\to \int_E f$.
\end{exercise}

\begin{exercise}{5.6}
  Given $x\in [0,1]$, let $\{x_n\}$ be any sequence in $[0,1]$ converging to $x$ and $x_n\neq x$.
  We define
  \[h_n(y)=\frac{f(x_n,y)-f(x,y)}{x_n-x},\quad n=1,2,\cdots.\]
  Since for each $x$, $f(x,y)$ is a measurable function of $y$,
  so are $h_n(y)$ by Theorem 4.9 and Theorem 4.10.
  Note that
  \[\frac{\partial}{\partial x}f(x,y)=\lim_{n\to\infty}h_n(y)\] exists
  and then the measurability of $(\partial f(x,y)/\partial x)$ w.r.t $y$ follows from Theorem 4.12.

  Since $(\partial f(x,y)/\partial x)$ is a bounded function of $(x,y)$,
  the mean value theorem implies that
  \[|h_n(y)|\leq \sup_{x\in[0,1]}\left|\frac{\partial}{\partial x}f(x,y)\right|\leq M, \quad\forall y\in[0,1],\]
  for some constant $M$.
  So the bounded convergence theorem (Corollary 5.37) can be invoked to give
  \[\frac{\rmd}{\rmd x}\int_0^1f(x,y)\rmd y=\lim_{n\to\infty}\int_0^1h_n(y)\rmd y=\int_0^1\lim_{n\to\infty}h_n(y)\rmd y=\int_0^1\frac{\partial}{\partial x}f(x,y)\rmd y.
  \qedhere\]
\end{exercise}

\begin{exercise}{5.9}
  Given $\epsilon>0$, we have
  \[|\left\{\bfx\in E:|f(\bfx)-f_k(\bfx)|>\epsilon\right\}|\leq\frac{1}{\epsilon^p}\int_{\{\bfx\in E:|f(\bfx)-f_k(\bfx)|>\epsilon\}}|f-f_k|^p\leq\frac{1}{\epsilon^p}\int_E|f-f_k|^p \to 0 \]
  as $k\to\infty$.
  This proves that $f_k\overset{m}{\longrightarrow} f$ on $E$ and thus
  that there is a subsequence $f_{k_j}\to f$ a.e.\ in $E$ by Theorem 4.22.
\end{exercise}

\begin{exercise}{5.10}
  Suppose $p > 0$ and $\int_E \lvert f_k - f \rvert^p \rightarrow 0$.
  By Exercise 5.9, we know that $f_k \xrightarrow{m} f$ on $E$.
%  First we claim that $f_k \xrightarrow{m} f$ on $E$ (Exercise 5.9).
%  Given $\epsilon > 0$, Chebyshev's inequality states that
%  \[
%    \mea{\left\{ \lvert f - f_k \rvert > \epsilon \right\}}
%    \le \frac{1}{\epsilon^p} \int_E \lvert f_k - f \rvert^p.
%  \]
%  Letting $k \rightarrow \infty$
%  we see that $\mea{\left\{ \lvert f - f_k \rvert > \epsilon \right\}} \rightarrow 0$
%  as $k \rightarrow \infty$,
%  which is exactly $f_k \xrightarrow{m} f$.
  By Theorem 4.22, there exists a subsequence $\{f_{k_j}\} \rightarrow f$ a.e.\ in $E$.
  If $\int_E \lvert f_k \rvert^p \le M$ for all $k$,
  we conclude by Fatou's Lemma that
  \[
    \int_E \lvert f \rvert^p = \int_E \lim \lvert f_{k_j} \rvert^p
    \le \liminf \int_E \lvert f_{k_j} \rvert^p \le M.
    \qedhere
  \]
\end{exercise}

\begin{exercise}{5.18}  The result of Exercise 5.16 suggests that if $0 < p < \infty$ and $f \ge 0$, then
  \[
    \int_E f^p = p \int_0^{\infty} \alpha^{p-1} \omega(\alpha) \rmd \alpha,
  \]
  regardless of the finiteness of either $\mea{E}$ or $\lVert f \rVert_p$.
  By definition of improper Riemann integral, we have
  \[
    \int_0^\infty \alpha^{p-1} \omega(\alpha) \rmd \alpha =
    \sum_{k = -\infty}^{+\infty}
    \int_{2^k}^{2^{k+1}} \alpha^{p-1} \omega(\alpha) \rmd \alpha.
%    \eqqcolon \sum_{k = -\infty}^{+\infty} I_k.
  \]
  Since $\omega$ is decreasing and $\alpha^{p-1} > 0$,
  we have
  \[
    I_k \coloneqq \int_{2^k}^{2^{k+1}} \alpha^{p-1} \omega(\alpha) \rmd \alpha
    \le \omega(2^k) \int_{2^k}^{2^{k+1}} \alpha^{p-1} \rmd \alpha
    = \frac{2^p-1}{p} 2^{kp} \omega(2^k).
  \]
  On the other hand,
  \[
    I_k \ge \omega(2^{k+1}) \int_{2^k}^{2^{k+1}} \alpha^{p-1} \rmd \alpha
    = \frac{2^p-1}{p 2^p} 2^{(k+1)p} \omega(2^{k+1}).
  \]
  Combining the above inequalities, we find
  \[
    \frac{2^p-1}{p 2^p}
    \sum 2^{(k+1)p} \omega(2^{k+1})
    \le \sum I_k
    \le \frac{2^p-1}{p} \sum 2^{kp} \omega(2^k).
  \]
  Note that the summation terms on the left-hand-side
  can be changed to $2^{kp} \omega(2^k)$.
  Summarizing, there exist constants $c_1, c_2 > 0$ such that
  $c_1 \sum 2^{kp} \omega(2^k) \le \int f^p \le c_2 \sum 2^{kp} \omega(2^k)$.
%  and consequently all the terms in the last inequalities
%  are simultaneously finite or infinite.
  We conclude that $f \in L^p$ if and only if $\sum 2^{kp} \omega(2^k) < + \infty$.
\end{exercise}


\begin{exercise}{5.21}
  If $f$ is not zero almost everywhere.
  Then at least one of the sets $\{f>0\}$ and $\{f<0\}$ has strictly positive measure.
  Suppose we have $|\{f>0\}|>0$.
  Since \[\{f>\frac{1}{n}\}\nearrow\{f>0\},\]
  we have $\lim_{n\rightarrow\infty}|\{f>1/n\}|=|\{f>0\}|$.
  Therefore for some $n_0$ sufficiently large,
  $A=\{f>1/n_0\}$ has strictly positive measure.
  In this case we can deduce that\[\int_A f\geq\frac{|A|}{n_0}>0, \] contradiction.
\end{exercise}

\begin{exercise}{5.22}
  We have a sequence of measurable functions $\{|f_k-f|\}$ on $E$
  such that $|f_k-f|\rightarrow 0$ a.e.\ in $E$.
  Now suppose on $E\setminus Z_k$, we have $|f_k|\leq \phi$ where $|Z_k|=0$.
  Let \[Z=\bigcup_k Z_k\cup\{x\in E:f_k(x)\not\rightarrow f(x)\},\]
  then $|Z|=0$ and on $E\setminus Z$,
  we have $|f|\leq \phi$.
  Therefore we deduce that $|f_k-f|\leq 2\phi$ a.e.\ in $E$.
  Since $2\phi\in L(E)$, from Lebesgue's Dominated Convergence Theorem,
  we have $\int_E|f_k-f|\rightarrow 0.$
\end{exercise}

\begin{exercise}{5.23}
  Let $g_k=|f_k-f|$, then $g_k\rightarrow 0$ a.e.\ in $E$
  and $|g_k|\leq \phi_k+\phi$ a.e.\ in $E$
  and $\int_E(\phi_k+\phi)\rightarrow 2\int_E\phi.$
  Therefore $\phi_k+\phi-g_k$ is nonnegative a.e.\ in $E$.
  By Fatou's Lemma, we have
  \[2\int_E\phi=\int_E\varliminf(\phi_k+\phi-g_k)\leq\varliminf\int_E\phi_k+\phi-g_k=2\int_E\phi-\varlimsup\int_Eg_k.\]
  Therefore \[0\leq\varliminf\int_E g_k\leq\varlimsup\int_Eg_k\leq 0.\]
  Thus $\lim\int_Eg_k=0.$
\end{exercise}

\begin{exercise}{5.24}
(a) Since $f\in L^p(E)$,
we have \[\alpha^p\omega_{|f|}(\alpha)\leq\int_{\{|f|>\alpha\}}|f|^p\leq\int_E|f|^p<\infty.\]
Thus \[\omega_{|f|}(\alpha)\leq\frac{\int_E|f|^p}{\alpha^p},\]
which implies $f$ belongs to weak $L^p(E)$.
Now let $E=(0,+\infty)$ and $f(x)=1/x$ defined on $E$,
then $\int_E f=\infty$.
But for any $\alpha>0$, we have $\omega_{|f|}(\alpha)=1/\alpha$.
Thus $f$ belongs to weak $L^1(E)$ but not $L^1(E)$.

(b) By definition we have nonnegative constants $A$ and $A'$ such that
$\omega_{|f|}(\alpha)\leq A/\alpha$ and $\omega_{|f|}(\alpha)\leq A'/\alpha^r$.
Thus $\omega_{|f|}$ is finite on $(0,\infty)$ and by results in Exercise 5.16 we have
\begin{align*}
  \int_E|f|^p&=p\int_0^\infty\alpha^{p-1}\omega_{|f|}(\alpha)d\alpha\\
  &=p\int_0^1\alpha^{p-1}\omega_{|f|}(\alpha)d\alpha+p\int_1^\infty\alpha^{p-1}\omega_{|f|}(\alpha)d\alpha\\
  &\leq p\int_0^1\frac{A}{\alpha^{2-p}}d\alpha+p\int_1^\infty \frac{A'}{\alpha^{1+r-p}}d\alpha.
\end{align*}
Now since $2-p<1$ and $1+r-p>1$, the right hand side integrals are finite,
which implies $f$ belongs to $L^p(E)$.

(c) Suppose $|f| \le M$ for some $M>0$.
Thus $\omega_{|f|}(\alpha)=0$ when $\alpha\geq M$.
We still have $\omega_{|f|}$ is finite on $(0,\infty)$ and therefore
\[\int_E|f|^p=p\int_0^M\alpha^{p-1}\omega_{|f|}(\alpha)d\alpha\leq p\int_0^M\frac{A}{\alpha^{2-p}}d\alpha<\infty\]
since $2-p<1.$
\end{exercise}

%  \section{Repeated Integration}
\begin{exercise}{6.3}
Let $F(x,y)=f(x)-f(y)$ be integrable over the square $[0,1]\times [0,1]$. By Fubini's theorem, it follows that for almost every $y\in [0,1]$, $F(x,y)$ is integrable on $[0,1]$ as a function of $x$. Since $f$ is finite a.e. on $[0,1]$, there exists $y_0\in[0,1]$ such that both $F(x,y_0)$ and $f(y_0)$ (which is a finite constant) are integrable with respect to $x$ on $[0,1]$. Thus the sum $f(x)=F(x,y_0)+f(y_0)$ is integrable on $[0,1]$ as a function of $x$.
\end{exercise}
\begin{exercise}{6.4}
Let $F(x,t)=|f(x+t)-f(-x+t)|$, then $\int_0^1F(x,t)\rmd t$ is measurable as a function of $x$ by Tonelli's Theorem; and $\int_0^1\int^1_0F(x,t)\rmd t\rmd x\leq c$. Making the change of variables $\xi=x+t, \eta=-x+t$, we have
\begin{equation*}
	\begin{aligned}
	\int_0^1\int^1_0F(x,t)\rmd t\rmd x&=\frac{1}{2}\int_0^1\int_{-\xi}^\xi|f(\xi)-f(\eta)|\rmd \eta\rmd \xi+\frac{1}{2}\int_1^2\int_{\xi-2}^{-\xi+2}|f(\xi)-f(\eta)|\rmd \eta \rmd \xi\\
	&\geq\frac{1}{2}\int_0^1\int_{0}^\xi|f(\xi)-f(\eta)|\rmd \eta \rmd \xi + \frac{1}{2}\int_1^2\int_{\xi-2}^{0}|f(\xi)-f(\eta)|\rmd \eta \rmd \xi\\
&	=\frac{1}{2}\int_0^1\int_{0}^\xi|f(\xi)-f(\eta)|\rmd \eta \rmd \xi + \frac{1}{2}\int_0^1\int_{\xi}^{1}|f(\xi+1)-f(\eta-1)|\rmd \eta \rmd \xi\\
	&=\frac{1}{2}\int_0^1\int_0^1|f(\xi)-f(\eta)|\rmd\eta\rmd\xi,
	\end{aligned}
\end{equation*}
where the last equality is provided by the periodicity of $f$. Thus $f(\xi)-f(\eta)$ is integrable over the square $[0,1]\times [0,1]$ and then $f\in L[0,1]$ by Exercise 6.3.
\end{exercise}
\begin{exercise}{6.5}
  (a) By Tonelli's theorem and the definition of Lebesgue integral, we have
  \[
    \int_Ef=\mea{R(f,E)}=\iint_{R(f,E)}\rmd\bfx \rmd y=\int_0^\infty\bigg[\int_{\mathbb{R}^n}\chi_{R(f,E)}\rmd\bfx\bigg]\rmd y.
  \]
  Now
  \[
    \int_{\mathbb{R}^n}\chi_{R(f,E)}\rmd\bfx=\mea{\{\bfx\in E:f(\bfx)\geq y\}}
  \] 
  and when $\omega$ is continuous at $y$, we have $\omega(y)=\mea{\{\bfx\in E:f(\bfx)\geq y\}}$. Since $\omega$ is monotone, it has countable number of points of discontinuity. We deduce that $\omega(y)=\mea{\{\bfx\in E:f(\bfx)\geq y\}}$ almost everywhere on $(0,\infty)$. Thus
  \[
    \int_Ef=\int_0^\infty\omega(y)\rmd y.
  \]
  (b) By (a) we have \[\int_Ef^p=\int_0^\infty\omega_{f^p}(y)\rmd y.\]Now $\omega_{f^p}(y)=\mea{\{\bfx\in E:f^p(\bfx)>y\}}=\mea{\{\bfx\in E:f(\bfx)>\sqrt[p]{y}\}}=\omega_f(\sqrt[p]{y})$ for $y\geq 0$. Since $\omega$ is improper Riemann integrable, we can use change of variables in improper Riemann integral to get
  \[
    \int_Ef^p=\int_0^\infty\omega_{f^p}(y)\rmd y=\int_0^\infty\omega_f(\sqrt[p]{y})\rmd y=p\int_0^\infty\omega(t)t^{p-1}\rmd t
    \qedhere
  \]
\end{exercise}
\begin{exercise}{6.6}
	Suppose $f$ and $g$ belong to $L(\bfr)$, then $f\ast g\in L(\bfr)$. Thus the Fourier transform $\widehat{f\ast g}$ is well defined and the application of Fubini’s theorem is justified; and then
	\begin{equation*}
		\begin{aligned}
		\widehat{(f\ast g)}(x)&=\frac{1}{2\pi}\int_\bfr (\widehat{f\ast g})(t)e^{-ixt}\rmd t\\
		&=\frac{1}{2\pi}\int_\bfr \left[\int_{\bfr}f(t-y)g(y)\rmd y\right]e^{-ixt}\rmd t\\
		&=\frac{1}{2\pi}\int_\bfr g(y)e^{-ixy}\left[\int_{\bfr}f(t-y)e^{-ix(t-y)}\rmd t\right]\rmd y\\
		&=2\pi \widehat{f}(x)\widehat{g}(x).
		\end{aligned}
	\end{equation*}
\end{exercise}

\begin{exercise}{6.10}
  Denote by $B^n_r$ the ball centered at the origin
  with radius $r$ in $\bbR^{n}$.
  First we claim that $\vol(B^n_r) = r^n v_n$.
  Indeed, the linear transformation $T : \bfx \mapsto r\bfx$
  maps the unit ball to $B^n_r$ bijectively.
  Since $|\det{T}| = r^n$, we have
  \[
    \vol(B^n_r) = \int_{B^n_r} \rmd \bfx
    = \int_{B^n_1} |\det{T}| \ \rmd \bfx
    = r^n v_n.
  \]

  Now we prove the formula.
  First we have $B^1_1 = [-1, 1]$ and $v_1 = 2$.
  For $n \ge 2$, we have
  \[
    \begin{aligned}
      v_n &= \int_{B^n_1} \rmd \bfx \\
      &= \int_{-1}^1 \left(\idotsint_{x_2^2 + \cdots + x_n^2 \le 1 - x_1^2}
      \rmd x_2 \cdots \rmd x_n\right) \rmd x_1 \quad
      &&(\text{Tonelli's Theorem}) \\
      &= \int_{-1}^1
      \left( \idotsint_{B^{n-1}_{(1-x_1^2)^{1/2}}} \rmd x_2 \cdots \rmd x_n \right)
      \rmd x_1 \\
      &= \int_{-1}^1 \left( 1 - x_1^2 \right)^{(n-1)/2} v_{n-1} \rmd x_1 \\
      &= 2 v_{n-1} \int_0^1 (1-t^2)^{(n-1)/2} \rmd t.
      \quad &&(t \mapsto \left( 1-t^2 \right)^{(n-1)/2} \text{ is even})
    \end{aligned}
  \]
  By setting $w = t^2$, we find $\rmd t = \frac{1}{2} w^{-1/2} \rmd w$ and
  \[
    \int_0^1 \left( 1-t^2 \right)^{(n-1)/2} \rmd t
    = \int_0^1 \frac{1}{2} \left( 1-w \right)^{(n-1)/2} w^{-1/2} \rmd w
    = \frac{1}{2} B(\frac{n+1}{2}, \frac{1}{2})
    = \frac{1}{2} \frac {\Gamma(\frac{n+1}{2}) \Gamma(\frac{1}{2})} {\Gamma(\frac{n}{2}+1)}.
    \qedhere
  \]
\end{exercise}


\begin{exercise}{6.11}
  When $n=1$, we have
  \[
    \bigg(\int_{\mathbb{R}}e^{-x^2}\rmd x\bigg)^2=\iint_{\mathbb{R}^2}e^{-(x^2+y^2)}\rmd x\rmd y=\int_0^{2\pi}\int_0^\infty e^{-r^2}r\rmd r\rmd\theta=\pi.
  \]
  Thus $\int_{\mathbb{R}}e^{-x^2}\rmd x=\pi^{1/2}$. Now for $n>1$, since $e^{-|\bfx|^2}$ is nonnegative and measurable on $\bbR^n$, we can use Tonelli's theorem for $n-1$ times and get
  \[
    \int_{\bbR^n}e^{-|\bfx|^2}\rmd\bfx=\int_{\bbR^n}e^{-x_1^2}e^{-x_2^2}\cdots e^{-x_n^2}\rmd x_1\rmd x_2\cdots \rmd x_n=\bigg(\int_{\mathbb{R}}e^{-x^2}\rmd x\bigg)^n=\pi^{n/2}.
    \qedhere
  \]
\end{exercise}
\begin{exercise}{6.13}
Let $F(x,y)=\frac{1}{2h}f(y)\chi_{[x-h,x+h]}(y)$ for fixed $h>0$, and note that $\chi_{[x-h,x+h]}(y)=\chi_{[y-h,y+h]}(x)$ for all $x,y\in\bfr$, then $\int_\bfr|F(x,y)|\rmd x=|f(y)|\in L(\rmd y)$. Thus $F(x,y)\in L(\rmd x\rmd y)$ and Fubini's Theorem gives 
\begin{equation*}
	\begin{aligned}
	\int_{-\infty}^{\infty}\left(\frac{1}{2h}\int_{x-h}^{x+h}f(y)\rmd y\right)\rmd x&=	\int_{-\infty}^{\infty}\left(	\int_{-\infty}^{\infty} F(x,y)\rmd y\right)\rmd x\\&=\int_{-\infty}^{\infty}\left(	\int_{-\infty}^{\infty} F(x,y)\rmd x\right)\rmd y\\
	&=	\int_{-\infty}^{\infty}\frac{1}{2h}f(y)\left(	\int_{-\infty}^{\infty} \chi_{[y-h,y+h]}(x)\rmd x\right)\rmd y\\
	&=\int_{-\infty}^{\infty}f(x)\rmd x.
	\end{aligned}
\end{equation*}
\end{exercise}

%  \section{Differentiation}
\begin{exercise}{7.1}
  By assumption, $\mea{\{f\neq 0\}}>0$ and since $\{|f|>1/k\}\nearrow\{|f|\neq 0\}$, there exists some $k_0$ such that $E=\{|f|>1/k_0\}$ has positive measure. Moreover, we may assume $E$ is bounded since otherwise we may consider $E\cap B(0,N)$ for some $N$ big enough. Now for any $\bfx$ with $|\bfx|\geq 1$, we consider the smallest cube centered at $\bfx$ which contains $E$ and denote it by $Q_\bfx$. We have
  \begin{align*}
      f^*(\bfx)&=\sup\frac{1}{|Q|}\int_Q|f(\bfy)|\rmd\bfy\geq \frac{1}{|Q_\bfx|}\int_{Q_\bfx\cap E}|f(\bfy)|\rmd\bfy\\
      &\geq \frac{\mea{Q_\bfx\cap E}}{k_0\mea{Q_\bfx}}=\frac{|E|}{k_0|Q_\bfx|}.
  \end{align*}
  Now we try to estimate the measure of $Q_\bfx$. Suppose $E\subset B(0,M)$ for some $M>0$. Let 
  \[
    r=\sup_{\bfy\in E}|\bfy-\bfx|.
  \]
  Then $r\leq |\bfx|+M$ and 
  \[
    |Q_\bfx|\leq (3r)^n\leq 3^n(|\bfx|+M)^n=|\bfx|^n3^n(1+\frac{M}{|\bfx|})^n\leq 3^n(1+M)^n|\bfx|^n.
  \]
  Therefore we have 
  \[
    f^*(\bfx)\geq \frac{|E|}{k_0(3+3M)^n}|\bfx|^{-n}
  \]
  for all $|\bfx|\geq 1.$
\end{exercise}

\begin{exercise}{7.2}
  By definition we have
  \[
    (f*\phi_\epsilon)(\bfx)-f(\bfx)=\int[f(\bfx-\bfy)-f(\bfx)]\phi_\epsilon(\bfy)\rmd\bfy
  \]
  Since $\phi_\epsilon(\bfx)=0$ when $|\bfx|\geq\epsilon$ and suppose $\phi(\bfx)$ is bounded by $M$, we can further have
  \begin{align*}
      |(f*\phi_\epsilon)(\bfx)-f(\bfx)|&=\bigg|\int_{|\bfy|<\epsilon}[f(\bfx-\bfy)-f(\bfx)]\phi_\epsilon(\bfy)\rmd\bfy\bigg|\\
      &=\bigg|\int_{|\bfy-\bfx|<\epsilon}[f(\bfy)-f(\bfx)]\phi_\epsilon(\bfx-\bfy)\rmd\bfy\bigg|\\
      &\leq \frac{M}{\epsilon^n}\int_{|\bfy-\bfx|<\epsilon}|f(\bfy)-f(\bfx)|\rmd\bfy\\
      &=\frac{M'}{\mea{B(\bfx,\epsilon)}}\int_{B(\bfx,\epsilon)}|f(\bfy)-f(\bfx)|\rmd\bfy
  \end{align*}
  which tends to zero when $\epsilon\rightarrow 0$ since $\bfx$ is a Lebesgue point.
 \end{exercise}
 
 
\begin{exercise}{7.4}
  Given $\epsilon > 0$, since $\mea{E_1} > 0$,
  there exists an open set $G_1$ such that
  $E_1 \subset G_1$ and $\mea{G_1} \le (1 + \epsilon) \mea{E_1}$.
  Write $G_1 = \bigcup I^{(1)}_k$ as a countable union of
  disjoint partly open dyadic intervals,
  and set $E^{(1)}_k = E_1 \cap I^{(1)}_k$.
  Clearly we have
  $\mea{G_1} = \sum \mea{I^{(1)}_k}$ and
  also $\mea{E_1} = \sum \mea{E^{(1)}_k}$,
  so there exists a $k_1$ such that
  \begin{equation}
    \mea{I^{(1)}_{k_1}} \le (1 + \epsilon) \mea{E^{(1)}_{k_1}}.
    \label{eq:measureEstimateForE1}
  \end{equation}
  Let $I^{(1)} \coloneqq I^{(1)}_{k_1}$ and $E^{(1)} \coloneqq E^{(1)}_{k_1}$.
  Repeating this construction for $E_2$, we have
  \[
    \mea{I^{(2)}} \le (1 + \epsilon) \mea{E^{(2)}}
  \]
  for some dyadic partly open interval $I^{(2)}$
  and some $E^{(2)} = E_2 \cap I^{(2)}$.

  If the length of $I^{(1)}$ is greater than that of $I^{(2)}$,
  we can subdivide $I^{(1)}$ to obtain
  subintervals which have the same length as $I^{(2)}$.
  Moreover,
  there must exist a subinterval, still denoted by $I^{(1)}$,
  such that \eqref{eq:measureEstimateForE1} holds.
  The opposite case is handled similarly.

  Suppose $I^{(1)} + q = I^{(2)}$ for some $q \in \bbR$.
  Note that for small $d > 0$,
  $E^{(1)} + (q+d)$ and $E^{(2)}$ are contained in
  an interval of length $|I^{(2)}| + |d|$.
  In case that they are disjoint, we have
  \[
    \frac{2}{1 + \epsilon} \mea{I^{(2)}} \le
    \mea{E^{(1)}} + \mea{E^{(2)}} \le \mea{I^{(2)}} + |d|.
  \]
  But this must be false for $\epsilon = \frac{1}{4}$
  and all $|d| < \frac{1}{2} |I^{(2)}|$.
  Consequently, $E^{(1)} + (q+d)$ has nonempty intersection with $E^{(2)}$
  for all $|d| < \frac{1}{2}|I^{(2)}|$.
  We conclude that $E_2 - E_1$ must contain an open interval
  of length $|I^{(2)}|$ centered at $q$.
\end{exercise}

\begin{exercise}{7.5}
  Since $g$ is absolutely continuous,
  it is of bounded variation.
  It follows that all $f, g, h$ are of bounded variation,
  and the Riemann-Stieltjes integrals
  $\int \phi \rmd f, \int \phi \rmd g$ and $\int \phi \rmd h$ exist.
  By linearity of Riemann-Stieltjes integral, we have
  \[
    \int_a^b \phi \rmd f = \int_a^b \phi \rmd g + \int_a^b \phi \rmd h.
  \]
  Next, we find
  \[
    \int_a^b \phi \rmd g = \int_a^b \phi g' \rmd x
    = \int_a^b \phi f' \rmd x,
  \]
  where the first equality follows from integration by parts
  and the second from the fact that $f' = g' + h' = g'$ a.e.
  We conclude that
  \[
    \int_a^b \phi \rmd f = \int_a^b \phi f' \rmd x + \int_a^b \phi \rmd h.
    \qedhere
  \]
\end{exercise}

\begin{exercise}{7.6}
(Quick proof)
Note that $\phi'(x) = \alpha x^{\alpha-1} > 0$ on $(0, \infty)$. 
Moreover, for any $0 < y < \infty$, 
we have
\[
  \int_0^y \phi' \rmd x
  = \lim_{\delta \rightarrow 0} \int_{\delta}^y \phi' \rmd x
  = \lim_{\delta \rightarrow 0} \left( \phi(y) - \phi(\delta) \right)
  = \phi(y) - \phi(0), 
\]
where we interpret the second integral 
in either Lebesgue or Riemann sense. 
Since $\phi(y) - \phi(0)$ is an indefinite integral of
the integrable function $\phi'$,
$\phi$ is absolutely continuous on $[0, y]$
and also on every bounded subinterval of $[0, \infty)$. 

(Long proof)
  Let $\phi(x) = x^{\alpha}$ with $\alpha > 0$.
  Suppose $\epsilon > 0$ and $0 < M < +\infty$ are given.
  Since $\phi$ is increasing and right-continuous at $0$,
  there exists a $\delta_0 > 0$ such that
  $\phi(\delta_0) - \phi(0) < \epsilon/2$.
  Since $\phi$ is $C^1$ on $[\delta_0, M]$,
  we have
  \[
    L = \sup_{x \in [\delta_0, M]} \left| \phi'(x) \right| < \infty.
  \]
  Now choose $\delta = \min \left\{ \delta_0, {\epsilon}/{2 L} \right\}$,
  and let $\left\{ [a_k, b_k] \right\}$ be a collection of
  nonoverlapping intervals contained in $[0, M]$ subject to
  \[
    \sum \left(b_k - a_k\right) < \delta.
  \]
%  We may further assume that each $[a_k, b_k]$ does not contain
%  $\delta_0$ in its interior,
%  for otherwise we can split the interval into two
%  and the sum of lengths of intervals remains less than $\delta$.
  First assume that no $[a_k, b_k]$ contains $\delta_0$ in its interior.
  On one hand we have
  \[
    \sum_{b_k \le \delta_0} \left| \phi(b_k) - \phi(a_k) \right|
    \le \phi(\delta_0) - \phi(0) < \epsilon/2
  \]
  since $\phi$ is increasing.
  On the other hand we use Mean-Value Theorem to deduce
  \[
    \sum_{a_k \ge \delta_0} \vert \phi(b_k) - \phi(a_k) \vert
    \le L \sum_{a_k \ge \delta_0} \left( b_k - a_k \right)
    < L \delta \le \epsilon/2.
  \]
  Thus we have
  \[
    \sum \left| \phi(b_k) - \phi(a_k) \right|
    = \sum_{b_k \le \delta_0} \left| \phi(b_k) - \phi(a_k) \right|
    + \sum_{a_k \ge \delta_0} \left| \phi(b_k) - \phi(a_k) \right|
    < \epsilon.
  \]
  If $\delta_0$ is contained in some $(a_{k_0}, b_{k_0})$,
  we apply the above argument to the collection of intervals
  with $[a_{k_0}, b_{k_0}]$ replaced by
  $[a_{k_0}, \delta_0]$ and $[\delta_0, b_{k_0}]$.
  Then $\sum \left| \phi(b_k) - \phi(a_k) \right| < \epsilon$
  remains valid due to the triangular inequality
  $|\phi(b_{k_0}) - \phi(a_{k_0})| \le
  |\phi(b_{k_0}) - \phi(\delta_0)| + |\phi(\delta_0) - \phi(a_{k_0})|$.
  We have shown that $\phi$ is absolutely continuous on $[0, M]$
  for every $0 < M < +\infty$, 
  and hence on every bounded subinterval of $[0, \infty)$. 
\end{exercise}

\begin{exercise}{7.7}
  Necessity follows easily from the inequality
  \[
    \left| \sum_{i=1}^m \left[ f(b_i) - f(a_i) \right] \right|
    \le \sum_{i=1}^m \left| f(b_i) - f(a_i) \right|.
  \]
  for every finite collection $\{[a_i, b_i]\}_1^m$. 
  We prove the sufficiency.
  Given $\epsilon > 0$,
  let $\delta$ be the value corresponding to $\epsilon/2$
  as dictated by the condition.
  Suppose $\left\{ [a_i, b_i] \right\}$ is a collection of
  nonoverlapping subintervals of $[a, b]$
  with $\sum \left( b_i - a_i \right) < \delta$.
  Define
  \[
    \begin{aligned}
      I^+_n \coloneqq &\left\{ 1 \le i \le n : f(b_i) - f(a_i) \ge 0 \right\}, \\
      I^-_n \coloneqq &\left\{ 1 \le i \le n : f(b_i) - f(a_i) < 0 \right\}.
    \end{aligned}
  \]
  Since $\sum_{i \in I^{\pm}_n} \left( b_i - a_i \right) < \delta$,
  we have
  \[
    \left| \sum_{i \in I^\pm_n} \left( f(b_i) - f(a_i) \right) \right|
    < \epsilon/2.
  \]
  It follows that
  \[
    \begin{aligned}
      \sum_{i=1}^n \left| f(b_i) - f(a_i) \right|
      &=\sum_{i \in I^+_n} \left( f(b_i) - f(a_i) \right)
      - \sum_{i \in I^-_n} \left( f(b_i) - f(a_i) \right) \\
      &\le \left| \sum_{i \in I^+_n} \left[ f(b_i) - f(a_i) \right] \right|
      + \left| \sum_{i \in I^-_n} \left[ f(b_i) - f(a_i) \right] \right| \\
      &< \epsilon.
    \end{aligned}
  \]
  But this holds for any $n \ge 1$. 
  Letting $n \rightarrow \infty$, 
  we find $\sum \left| f(b_i) - f(a_i) \right| \le \epsilon$, 
  so $f$ is absolutely continuous.
\end{exercise}

 
 \begin{exercise}{7.10}
   (a) First we notice that since $f$ is continuous, for any interval $[a_i,b_i]$, $f([a_i,b_i])$ is still an interval. More precisely, we have
   \[
     f([a_i,b_i])=[\min_{[a_i,b_i]}f,\max_{[a_i,b_i]}f]=[f(x_i),f(y_i)]
   \]
   for some $x_i,y_i\in[a_i,b_i]$. Therefore
   \begin{align*}
       \mea{f([a_i,b_i])}=\mea{f(y_i)-f(x_i)}\leq V[f;a_i,b_i]=V(b_i)-V(a_i).
   \end{align*}
   Now since $f$ is absolutely continuous, $V$ is also absolutely continuous. For any given $\epsilon$, there exists $\delta>0$ such that for any collection $\{[a_i,b_i]\}$ of nonoverlapping subintervals of $[a,b]$ with $\sum(b_i-a_i)<\delta$, we have $\sum V(b_i)-V(a_i)<\epsilon.$ Now since $\mea{Z}=0$, we can cover $Z$ by nonoverlapping subintervals $\{[a_i,b_i]\}$ such that $\sum(b_i-a_i)<\delta$. Then $f(Z)\subset\bigcup_i f([a_i,b_i])$ and 
   \[
     \mea{f(Z)}\leq\sum\mea{f([a_i,b_i])}\leq\sum V(b_i)-V(a_i)<\epsilon.
   \]
   Since $\epsilon$ is arbitrary, we have $\mea{f(Z)}=0.$ 
   
   
   Now let $E$ be any measurable subset of $[a,b]$, there exist some closed sets  $\{F_n\}_{n=1}^\infty$ and a set of measure zero $Z$ such that 
   \[
     E=\bigcup_nF_n\cup Z.
   \]
   Therefore $F_n$ is compact for each $n$ and $f(F_n)$ is also compact since $f$ is continuous. We can now write $f(E)$ as union of countable compact sets with a set of measure zero as follows:
   \[
     f(E)=\bigcup_nf(F_n)\cup f(Z)
   \]
   which implies $f(E)$ is measurable.
   
   
   (b) Let $g(x)=x+F(x)$ where $F(x)$ is the Cantor-Lebesgue function and let $f(x)=g^{-1}(x):[0,2]\rightarrow [0,1]$. Since $g$ is a strictly increasing continuous function on $[0,1]$, $f$ is well-defined and also strictly increasing continuous. Moreover, for any distinct $x,x'$ with $x>x'$, we have $|g(x)-g(x')|=x-x'+F(x)-F(x')\geq |x-x'|$ since $F(x)$ is monotone increasing. Therefore for any $y, y'\in[0,2]$, we have $|f(y)-f(y')|\leq |g(f(y))-g(f(y'))|=|y-y'|$. Thus $f$ is Lipschitz. Let $C$ be the Cantor set on $[0,1]$. Then $\mea{C}=0$ and we claim that $f^{-1}(C)=g(C)$ has measure 1. We notice that $[0,1]\setminus C$ is a disjoint union of open intervals $\{I_i\}$ and $F$ is constant on each of the intervals. Since $g$ is strictly monotone, it follows that $g(I_i)$ are still disjoint intervals with the same measures as $I_i$. Thus 
   \[
     |g([0,1]\setminus C)|=\sum|g(I_i)|=\sum|I_i|=|[0,1]\setminus C|=1.
   \]
   Therefore since $g$ is strictly increasing, we have 
   \[
     |g(C)|=|[0,2]-g([0,1]\setminus C)|=2-1=1.
     \qedhere
   \]
 \end{exercise}
 
 \begin{exercise}{7.11}
   First we consider the case when $f=\chi_G$ where $G$ is an open subset of $[a,b]$. Then we may write $G=\bigcup_{i=1}^n (a_i,b_i)$ where $(a_i,b_i)$ are disjoint open intervals. We can also assume $g^{-1}(a_i,b_i)=(\alpha_i,\beta_i)$, then $(\alpha_i,\beta_i)$ are also disjoint. Now $f\circ g=\chi_{g^{-1}(G)}=\sum_i\chi_{(\alpha_i,\beta_i)}$, which is clearly measurable. We have
   \[
     \int_\alpha^\beta f(g(t))g'(t)\rmd t=\sum_i\int_{\alpha_i}^{\beta_i}g'(t)\rmd t=\sum_i (b_i-a_i)=\mea{G}=\int_{a}^{b}f(x)\rmd x.
   \]
   In particular if $G$ is an open subset, we have
   \[
     \mea{G}=\int_{g^{-1}(G)}g'(t)\rmd t.
   \]
   
   Next we consider the general case when $f=\chi_E$ for a measurable set $E$. Then $f(g(t))g'(t)=\chi_{g^{-1}(E)}(t)g'(t)=\chi_{g^{-1}(E)\cap\{g'>0\}}(t)g'(t)$. To prove this is a measurable function, it suffices to prove $g^{-1}(E)\cap\{g'>0\}$ is a measurable set. Now suppose 
   \[
     E=\bigcap_iG_i-Z
   \]
   where $G_i$ are open and $Z$ is of measure zero. Then we deduce that
   \begin{align*}
       g^{-1}(E)\cap\{g'>0\}&=\big(\bigcap_ig^{-1}(G_i)-g^{-1}(Z)\big)\cap\{g'>0\}\\
       &=\bigcap_i\big(g^{-1}(G_i)\cap\{g'>0\}\big)-\big(g^{-1}(Z)\cup\{g'=0\}\big).
   \end{align*}
   Since $g^{-1}(Z)\cup\{g'=0\}=\big(g^{-1}(Z)\cap\{g'>0\}\big)\cup\{g'=0\}$, it suffices to prove $g^{-1}(Z)\cap\{g'>0\}$ is measurable. In fact we prove it to be of measure zero. Since $Z$ has measure zero, for any $k$ we can find an open set $W_k$ such that $Z\subset W_k$ and $\mea{W_k}<1/k$. Let $G'=\bigcap_k g^{-1}(W_k)\cap\{g'>0\}$, which is measurable and contains $g^{-1}(Z)\cap\{g'>0\}$. It suffices to prove $G'$ has zero measure. Since $g'$ is nonnegative and measurable, for any $n$, we have
   \[
     0\leq\int_{G'}g'(t)\rmd t\leq \int_{g^{-1}(W_n)}g'(t)\rmd t=\mea{W_n}<\frac{1}{n}.
   \]
   Thus we must have $\int_{G'}g'(t)dt=0$. Now if $G'$ has positive measure, since $g'$ is strictly positive on it, we can find a subset of positive measure such that $g'>\epsilon$ on it for some $\epsilon>0$. Then the integral will be greater than zero, contradiction. Now we calculate the original integral. In our setting $E=\bigcap_iG_i-Z$, we may assume that $G_i$ is monotone decreasing, otherwise we can set $G_i'=\bigcap_{k=1}^iG_k$. Then $g^{-1}(G_i)$ is also decreasing and $\mea{E}=\lim_i\mea{G_i}$. Now we have
   \begin{align*}
       \int_\alpha^\beta f(g(t))g'(t)\rmd t&=\int_{g^{-1}(E)}g'(t)\rmd t=\int_{\bigcap_ig^{-1}(G_i)}g'(t)\rmd t-\int_{g^{-1}(Z)}g'(t)\rmd t\\
       &=\lim_i\int_{g^{-1}(G_i)}g'(t)\rmd t-\int_{g^{-1}(Z)\cap\{g'>0\}}g'(t)\rmd t\\
       &=\lim_i\mea{G_i}=\mea{E}=\int_a^bf(x)\rmd x
   \end{align*}
   where we use the fact that $g'$ is nonnegative and integrable.
   
   
   By linearity, the conclusion holds for any simple functions. Now if $f$ is nonnegative, we can find a sequence of nonnegative increasing simple functions $\{f_k\}$ which converges to $f$ pointwise. Then $f(g(t))g'(t)$ is measurable as the limit of $f_k(g(t))g'(t)$. By dominated convergence theorem, we have
   \[
     \int_a^bf(x)\rmd x=\lim_k\int_a^bf_k(x)\rmd x=\lim_k\int_\alpha^\beta f_k(g(t))g'(t)\rmd t=\int_\alpha^\beta f(g(t))g'(t)\rmd t
   \]
   Finally we can use $f=f^+-f^-$ to get the conclusion for all integrable functions $f$.
 \end{exercise}
 
 
\begin{exercise}{7.14}
	Suppose $\phi$ is convex on $(a,b)$. Then the continuity of $\phi$ is proved in Theorem 7.40 and 
	\begin{equation}\label{714}
			\phi\left(\frac{x_1+x_2}{2}\right)\leq \frac{\phi(x_1)+\phi(x_2)}{2},\quad\forall x_1,x_2\in(a,b)
	\end{equation}
	follows from the definition of convexity.
	
	Conversely, suppose $\phi$ is continuous and (\ref{714}) holds. We first show, by induction on $n$, that the inequality 
	\begin{equation}\label{7142}
		\phi\left(\frac{x_1+x_2+\cdots+x_{2^n}}{2^n}\right)\leq \frac{1}{2^n}\left(\phi(x_1)+\phi(x_2)+\cdots+\phi(x_{2^n})\right)
	\end{equation}
	holds for all positive integer $n$ and any $x_1, x_2,\cdots, x_{2^n}$ in $(a,b)$. The inequality being shown for $n$, we pass to $n+1$: 
	\begin{equation*}
		\begin{aligned}
		\phi\left(\frac{x_1+x_2+\cdots+x_{2^{n+1}}}{2^{n+1}}\right)=&	\phi\left(\frac{1}{2}\left(\frac{x_1+x_2+\cdots+x_{2^{n}}}{2^{n}}+\frac{x_{2^{n}+1}+x_{2^{n}+2}+\cdots+x_{2^{n+1}}}{2^{n}}\right)\right)\\
	\leq&\frac{1}{2}\left(\phi\left(\frac{x_1+x_2+\cdots+x_{2^n}}{2^n}\right)+	\phi\left(\frac{x_{2^{n}+1}+x_{2^{n}+2}+\cdots+x_{2^{n+1}}}{2^n}\right)\right)\\
	\leq&\frac{1}{2}\left(\frac{1}{2^n}\sum_{i=1}^{2^n}\phi(x_i)+\frac{1}{2^n}\sum_{i={2^n+1}}^{2^{n+1}}\phi(x_i)\right)\\
	=&\frac{1}{2^{n+1}}\left(\phi(x_1)+\phi(x_2)+\cdots+\phi(x_{2^{n+1}})\right).
		\end{aligned}
	\end{equation*}
	This proves (\ref{7142}). For $m\in\{0,1,\cdots,2^n\}$, let $x_i=x$ $( i\leq m)$ and $x_i=y$ $( i\geq m+1)$, where $x,y\in(a,b)$. By (\ref{7142}), we have
	\begin{equation}\label{7143}
	\phi\left(\frac{m}{2^n}x+(1-\frac{m}{2^n})y\right)\leq\frac{m}{2^n}\phi\left(x\right)+\left(1-\frac{m}{2^n}\right)\phi(y).
	\end{equation} 
	Finally, we note that for any $\theta\in[0,1]$,
	$2^n\theta-1<{\left\lfloor 2^{n} \theta\right\rfloor}\leq2^n\theta$ and $\left\lfloor 2^{n} \theta\right\rfloor\in\{0,1,\cdots,2^n\}$. (Here $\left\lfloor x\right\rfloor$ denotes the floor function, i.e. the greatest integer less than or equal to $x$.) Thus $\left\lfloor 2^{n} \theta\right\rfloor/2^n\to\theta$ as $n\to\infty$.
It follows from (\ref{7143}) that
$$\phi\left(\frac{\left\lfloor 2^{n} \theta\right\rfloor}{2^n}x+(1-\frac{\left\lfloor 2^{n} \theta\right\rfloor}{2^n})y\right)\leq\frac{\left\lfloor 2^{n} \theta\right\rfloor}{2^n}\phi\left(x\right)+\left(1-\frac{\left\lfloor 2^{n} \theta\right\rfloor}{2^n}\right)\phi(y).$$
Since $\phi$ is continuous, by taking limit $n\to\infty$ in the inequality above, we have  
	\begin{equation*}
	\phi(\theta x+(1-\theta)y)\leq \theta \phi(x)+(1-\theta)\phi(y),
	\end{equation*}
which proves $\phi$ is convex.
\end{exercise}
\begin{exercise}{7.17}
	Since $f_k$ and $f$ are integrable on $(0,1)$, the indefinite intergals $\phi_k(E)=\int_Ef_k$ and $\phi(E)=\int_E|f|$ are absolutely continuous.
	
	Suppose $\int_0^1|f-f_k|\to0$, then for given $\epsilon>0$, there exists an integer $K>0$ such $\int_0^1|f-f_k|<\epsilon/2$ whenever $k>K$. Since
 $\phi_k$ $(k=1,\cdots,K)$ and $\phi$ are absolutely continuous, there exists $\delta>0$, such that if $E$ satisfies $|E|<\delta$, then $|\phi_k(E)|<\epsilon$ for all $k=1,\cdots,K$ and
 $\phi(E)<\epsilon/2$. (Note that such $\delta$ exists as $\{\phi_1,\cdots,\phi_K,\phi\}$ is a finite collection of absolutely continuous set functions.) For $k>K$, we have
 $$|\phi_k(E)|\leq\int_E|f_k|\leq\int_0^1|f-f_k|+\phi(E)<\frac{\epsilon}{2}+\frac{\epsilon}{2}=\epsilon$$
 provided $|E|<\delta$. Therefore, $|\phi_k(E)|<\epsilon$ for all $k$ whenever $|E|<\delta$, i.e. $\{\phi_k\}$ is uniformly absolutely continuous.
 
 Conversely, suppose $\{\phi_k\}$ is uniformly absolutely continuous. Note that 
 \begin{equation*}
 	\begin{aligned}
 \int_E|f-f_k|&=\int_{E\cap\{f-f_k>0\}}(f-f_k)+\int_{{E}\cap\{f-f_k\leq0\}}(f_k-f)\\&\leq\phi(E)+|\phi_k(E\cap\{f-f_k>0\})|+|\phi_k(E\cap\{f-f_k\leq0\})|,
 	\end{aligned}
 \end{equation*}
 thus $\{\int_E|f-f_k|\}$ is also uniformly absolutely continuous. 
  Then given $\epsilon>0$, there exits $\delta>0$ such that $\int_E|f-f_k|<\epsilon/2$ for all $k$ provided $|E|<\delta$. Note that $f_k\to f$ a.e. on $(0,1)$, then $f_k\overset{m}{\longrightarrow} f$ on $(0,1)$. In particular, there exists $K>0$ such that $|\{x\in(0,1):|f(x)-f_k(x)|>\epsilon/2\}|<\delta$ whenever $k>K$. Thus 
 \begin{equation*}
 	\int_0^1|f-f_k|=\int_{\{|f-f_k|\leq\epsilon/2\}}|f-f_k|+\int_{\{|f-f_k|>\epsilon/2\}}|f-f_k|
 	\leq\frac{\epsilon}{2}+\frac{\epsilon}{2}=\epsilon
 \end{equation*}
 provided $k>K$.
 This proves $\int_0^1|f-f_k|\to 0$ since $\epsilon>0$ is arbitrary.
\end{exercise}
  \section{$L^p$ Classes}

\begin{exercise}{8.2}
(1)	The converse of H\"older's inequality is trivial if $f=0$ a.e.\ in $E$, so we assume $\Vert f\Vert_p>0$. 
	H\"older's inequality says that $\Vert f\Vert_p\geq \sup \int_Efg$,
	where the supremum is taken over all real-valued $g$ such that $\Vert g\Vert_{q^\prime}\leq 1$ and $\int_E fg$ exists. Thus we only need to show the opposite inequality. 
	
	In the case of $p=1$, we let $g=\mathrm{sign} f$, then $\Vert g\Vert_\infty=1$ and $\int_Efg=\int_E|f|=\Vert f\Vert_1$.
For $p=\infty$, given $\epsilon>0$, let $A=\{x\in E:|f(x)|>\|f\|_\infty-\epsilon\}$. Then $|A|>0$ and there exists $B\subset A$ with $0<|B|<\infty$. Let $g=|B|^{-1}\chi_B(\mathrm{sign} f)$; then $\|g\|_1=1$, so
$$\int_Efg=\frac{1}{|B|}\int_B|f|\geq\|f\|_\infty-\epsilon.$$
Since $\epsilon>0$ is arbitrary, $\underset{{\|g\|_1\leq1}}{\sup}\int_Efg\geq\|f\|_\infty$.

(2) For $1\leq p\leq\infty$, let $f$ be a real-valued measurable function such that
$fg\in L^1(E)$ for every $g\in L^{p^{\prime}}$, $1/p+1/p^{\prime}=1$. Suppose that $f\not\in L^p(E)$, i.e. $\|f\|_p=+\infty$, then for every $n$, there exists $g_n$ with $\|g_n\|_{p^{\prime}}\leq1$ such that $\int_E|f|g_n>n^3$. Let $g(x)=\sum_{n=1}^{\infty}|g_n(x)|/n^2$ and $S_N(x)=\sum_{n=1}^{N}|g_n(x)|/n^2$. By Minkowski's Inequality, $\|S_N\|_{p^{\prime}}\leq\sum_{n=1}^N\|g_n\|_{p^\prime}/n^2\leq\sum_{n=1}^\infty1/n^2$. Then monotone convergence theorem gives $\|g\|_{p^\prime}=\lim_N\|S_N\|_{p^\prime}\leq\sum_{n=1}^\infty1/n^2<+\infty$, i.e. $g\in L^{p^\prime}(E)$. However, for every $n$,  $$\int_E|f|g\geq\frac{1}{n^2}\int_E|f|g_n> n.$$
Hence $fg\not\in L^1(E)$ and this contradiction proves that $f\in  L^p(E)$. 
\end{exercise}
\begin{exercise}{8.8}\label{88}
  For $p=1$, Tonelli's theorem can be invoked to give 
  $$\int\int|f(\bfx,\bfy)|\rmd\bfx\rmd\bfy=\int\int|f(\bfx,\bfy)|\rmd\bfy\rmd\bfx.$$
  
  For $1<p<\infty$, let $p^\prime$ denote the exponent conjugate to $p$, i.e. $1/p+1/p^\prime=1$. Note that
  \begin{equation*}
      \begin{aligned}
        \int\left[\int|f(\bfx,\bfy)|\rmd\bfx\right]^p\rmd\bfy&=\int\left[\int|f(\bfz,\bfy)|\rmd\bfz\right]^{p-1}\left[\int|f(\bfx,\bfy)|\rmd\bfx\right]\rmd\bfy\\
        (\text{by Tonelli's theorem})&=\int\left[\int\left[\int|f(\bfz,\bfy)|\rmd\bfz\right]^{p-1}|f(\bfx,\bfy)|\rmd\bfy\right]\rmd\bfx\\
        (\text{by H\"older's inequality})&\leq\int\left[\left[\int\left[\int|f(\bfz,\bfy)|\rmd\bfz\right]^p\rmd\bfy\right]^{1/p^\prime}\left[\int|f(\bfx,\bfy)|^p\rmd\bfy\right]^{1/p}\right]\rmd\bfx\\
       &=\left[\int\left[\int|f(\bfx,\bfy)|\rmd\bfx\right]^{p}\rmd\bfy\right]^{1/p^\prime}\int\left[\int|f(\bfx,\bfy)|^p\rmd\bfy\right]^{1/p}\rmd \bfx.
      \end{aligned}
  \end{equation*}
  The generalized Minkowski's inequality follows by dividing both sides by $\left[\int\left[\int|f(\bfx,\bfy)|\rmd\bfx\right]^{p}\rmd\bfy\right]^{1/p^\prime}$. (Note that if $\int\left[\int|f(\bfx,\bfy)|\rmd\bfx\right]^p\rmd\bfy=0$, the result is obvious.)
\end{exercise}
\begin{remark}
The integral version of Minkowski's inequality can be written as
$$\left\Vert\|f(\bfx,\bfy)\|_{L^1(\rmd\bfx)}\right\Vert_{L^p(\rmd\bfy)}\leq\left\Vert\|f(\bfx,\bfy)\|_{L^p(\rmd\bfy)}\right\Vert_{L^1(\rmd\bfx)},$$
where $1\leq p<\infty$. 

Consider the measurable function $f(x,\bfy)$ defined on $[0,2]\times\bfrn$:
$$
f(x,\bfy)= \begin{cases}g(\bfy), & 0\leq x<1, \\ h(\bfy), & 1\leq x\leq2.\end{cases}
$$
Then the integral version of Minkowski's inequality implies the ordinary Minkowski's inequality $$\|g+h\|_p\leq\|g\|_p+\|h\|_p.$$
\end{remark}
\begin{exercise}{8.11}
  First since $\inorm{g_k}\leq M$, we have $\mea{\{g_k>M\}}=0$. Let $x\in E$ such that $g(x)>M$, then there exists $N$ such that $g_k(x)>M$ when $k>N$. Therefore $\{g>M\}\subset\liminf{\{g_k>M\}}$. This implies
  \[
    \mea{\{g>M\}}\leq\mea{\{\liminf{\{g_k>M\}}\}}\leq\liminf{\mea{\{g_k>M\}}}=0.
  \]
  Thus $\inorm{g}\leq M.$ Now by Minkowski's Inequality, we have
  \begin{align*}
    \pnorm{f_kg_k-fg}&=\pnorm{f(g_k-g)+g_k(f_k-f)}\\
    &\leq \pnorm{f(g_k-g)}+\pnorm{g_k(f_k-f)}\\
    &=\bigg(\int_E|f|^p|g_k-g|^p\bigg)^{1/p}+\bigg(\int_E|g_k|^p|f_k-f|^p\bigg)^{1/p}\\
    &\leq \bigg(\int_E|f|^p|g_k-g|^p\bigg)^{1/p}+M\pnorm{f_k-f}
  \end{align*}
  where we used Holder's Inequality in the last step. It suffices to prove the first term tends to zero when $k$ tends to infinity. Let $E_n=E\cap B(0,n)$. Then since $E=\lim E_n$ and $f\in L^p(E)$, we have
  \[
    \int_E|f|^p=\lim_{n\rightarrow\infty}\int_{E_n}|f|^p.
  \]
  For any $\epsilon>0$, there exists $n$ sufficiently large such that 
  \[
    \int_{E\setminus E_n}|f|^p<\frac{\epsilon}{2(2M)^p}.
  \]
  Now since $E_n$ has finite measure, we deduce that $g_k$ converges to $g$ in measure. We choose 
  \[
    \delta=\frac{\epsilon}{2\pnorm{f}^p}
  \]
  Then the measure of $E_n^k=\{x\in E_n:|g_k(x)-g(x)|^p>\delta\}$ tends to zero as $k$ tends to infinity. Therefore
  \begin{align*}
      \int_E|f|^p|g_k-g|^p&=\int_{E\setminus E_n}|f|^p|g_k-g|^p+\int_{E_n\setminus E_n^k}|f|^p|g_k-g|^p+\int_{E_n^k}|f|^p|g_k-g|^p\\
      &\leq (2M)^p\int_{E\setminus E_n}|f|^p+\delta\int_{E_n\setminus E_n^k}|f|^p+(2M)^p\int_{E_n^k}|f|^p\\
      &\leq\epsilon+(2M)^p\int_{E_n^k}|f|^p.
  \end{align*}
  Now since the set function $F(A)=\int_A|f|^p$ is absolutely continuous and $\lim_k\mea{E_n^k}=0$, we have
  \[
    \varlimsup_{k\rightarrow \infty}\int_E|f|^p|g_k-g|^p\leq\epsilon
  \]
  and since $\epsilon$ is arbitrary, the limit must exist and equal zero.
\end{exercise}

\begin{exercise}{8.12}
  When $1\leq p\leq\infty$, by Minkowski's Inequality, we have
  \[
    \pnorm{f}-\pnorm{f-f_k}\leq\pnorm{f_k}\leq\pnorm{f_k-f}+\pnorm{f}.
  \]
  By letting $k$ tends to infinity, we deduce that $\pnorm{f_k}\rightarrow\pnorm{f}.$
  
  When $0<p<1$, for any $a,\ b\geq 0$, we have the inequality
  \begin{align}
    (a+b)^p\leq a^p+b^p.
  \end{align}
  In fact if $a=b=0$, this is obvious. Otherwise suppose $a\neq 0$, then dividing by $a^p$, if suffices to prove $(1+t)^p\leq 1+t^p$ for $t\geq 0$. This is true since when $t=0$ we have the equality and the derivative of the right side majorizes that of the left for $t>0$. Therefore we also have
  \[
    \pnorm{f}^p-\pnorm{f-f_k}^p\leq\pnorm{f_k}^p\leq\pnorm{f_k-f}^p+\pnorm{f}^p.
  \]
  We deduce the same conclusion by letting $k$ tends to infinity.
  
  Conversely, we first notice that for $0<p<\infty$ and any $a,\ b\geq 0$ we have the following
  \begin{align}
    (a+b)^p\leq c(a^p+b^p)
  \end{align}
  where $c=\max\{2^{p-1},1\}$. In fact, when $0<p<1$, this is true by (1) above. When $p\geq 1$, we notice that the function $x^p$ is convex for $x\geq 0$ and therefore 
  \[
    (\frac{a+b}{2})^p\leq \frac{a^p+b^p}{2},
  \]
  which implies (2) is true for all $0<p<\infty.$ Now by using (2) we have
  \[
    |f-f_k|^p\leq (|f|+|f_k|)^p\leq c(|f|^p+|f_k|^p)
  \]
  Now let 
  \[
    g_k=c(|f|^p+|f_k|^p)-|f-f_k|^p\geq 0.
  \]
  Then $g_k\rightarrow 2c|f|^p$ a.e. and we can apply Fatou's Lemma on $g_k$:
  \[
    2c\int|f|^p\leq c\int|f|^p+c\varliminf_{k\rightarrow\infty}\int|f_k|^p-\varlimsup_{k\rightarrow\infty}\int|f-f_k|^p.
  \]
  Since $\pnorm{f_k}\rightarrow\pnorm{f}$, we deduce that
  \[
    0\leq\varliminf_{k\rightarrow\infty}\int|f-f_k|^p\leq\varlimsup_{k\rightarrow\infty}\int|f-f_k|^p\leq 0.
  \]
  Thus we must have $\pnorm{f-f_k}\rightarrow 0.$
  
  When $p=\infty$, let $f(x)\equiv1$ on $\bbR$ and $f_k(x)=1$ when $x\leq k$ and $f_k(x)=0$ when $x>k.$ Then $f_k\rightarrow f$ a.e. but we also have 
  \[
    \inorm{f}=\inorm{f_k}=\inorm{f-f_k}=1.
  \]
  Thus the conclusion does not hold when $p=\infty.$
\end{exercise}

\begin{exercise}{8.13}
  When $1<p<\infty$, we have $1<p'<\infty$. Let $E_n=E\cap B(0,n)$. Then since $E=\lim E_n$ and $g\in L^{p'}(E)$, we have
  \[
    \int_E|g|^{p'}=\lim_{n\rightarrow\infty}\int_{E_n}|g|^{p'}.
  \]
  For any $\epsilon>0$, there exists $n$ sufficiently large such that 
  \[
    \int_{E\setminus E_n}|g|^{p'}<\epsilon^{p'}.
  \]
  Therefore
  \begin{align*}
      \int_{E\setminus E_n}|(f_k-f)g|&\leq \bigg(\int_{E\setminus E_n}|f_k-f|^p\bigg)^{1/p}\bigg(\int_{E\setminus E_n}|g|^{p'}\bigg)^{1/{p'}}\\
      &\leq \pnorm{f_k-f}\epsilon\\
      &\leq (\pnorm{f_k}+\pnorm{f})\epsilon\\
      &\leq (M+\pnorm{f})\epsilon.
  \end{align*}
  Let $k$ tend to infinity and since $\epsilon$ is arbitrary we deduce that the limit of the left hand side is zero. Now we consider the integral on $E_n$. Since $E_n$ has finite measure, it follows that $f_k$ converges to $f$ in measure. Let 
  \[
    E_n^k=\{x\in E_n:|f_k(x)-f(x)|>\delta\}
  \]
  where 
  \[
    \delta=\frac{\epsilon}{\mea{E_n}^{1/p}\left\|g\right\|_{p'}}.
  \] Then $\lim_k\mea{E_n^k}=0$. By Holder's Inequality and Minkowski's Inequality, we have
  \begin{align}
      \int_{E_n}|(f_k-f)g|&=\int_{E_n^k}|(f_k-f)g|+\int_{E_n\setminus E_n^k}|(f_k-f)g|\notag\\
      &\leq \bigg(\int_{E_n^k}|f_k-f|^p\bigg)^{1/p}\bigg(\int_{E_n^k}|g|^{p'}\bigg)^{1/{p'}}+\bigg(\int_{E_n\setminus E_n^k}|f_k-f|^p\bigg)^{1/p}\bigg(\int_{E_n\setminus E_n^k}|g|^{p'}\bigg)^{1/{p'}}\notag\\
      &\leq \pnorm{f_k-f}\bigg(\int_{E_n^k}|g|^{p'}\bigg)^{1/{p'}}+\delta\mea{E_n}^{1/p}\left\|g\right\|_{p'}\notag\\
      &\leq (M+\pnorm{f})\bigg(\int_{E_n^k}|g|^{p'}\bigg)^{1/{p'}}+\epsilon.
  \end{align}
  Since the set function $F(A)=\int_A|g|^{p'}$ is absolutely continuous, we have
  \[
    \lim_{k\rightarrow\infty}\int_{E_n^k}|g|^{p'}=0.
  \]
  Thus we can let $k$ tends to infinity on both sides of (3) we get that 
  \[
    \varlimsup_{k\rightarrow\infty}\int_{E_n}|(f_k-f)g|\leq \epsilon.
  \]
  Since $\epsilon$ is arbitrary, it follows that $\lim_{k\rightarrow\infty}\int_{E_n}|(f_k-f)g|=0.$
  
  When $p=\infty$ and $p'=1$, the conclusion is still true using the same method as above except that we use the corresponding version of Holder's and Minkowski's Inequality for $p=\infty.$
  
  When $p=1$ and $p'=\infty$, we let $g\equiv 1$ and $f\equiv 0$ on $\bbR$ and $f_k(x)=k$ when $x\in[0,1/k]$ and $f_k(x)=0$ elsewhere. Then $f_k\rightarrow f$ everywhere and $\left\|f_k\right\|_1=1$ but $\left\|f\right\|_1=0$. Thus the conclusion does not hold for $p=1.$
\end{exercise}

\begin{exercise}{8.15}
  Let $i^2 = -1$, and we first show that the systems
  $\phi_k(x) = \frac{1}{\sqrt{2\pi}} e^{i k x}, k = 1, 2, \cdots$,
  are orthonormal in $L^2(0, 2\pi)$.
  We compute
  \[
    \left< \phi_k, \phi_l \right>
%    = \frac{1}{\pi} \int_0^{2\pi}
%    e^{2\pi i k x} \overline{e^{2\pi i l x}} \rmd x
    = \frac{1}{2\pi} \int_0^{2\pi} e^{i (k-l) x} \rmd x
    = \left\{
    \begin{aligned}
      & \frac{1}{2\pi} \int_0^{2\pi} \rmd x = 1, \quad &&\text{if }k=l, \\
      & \frac{1}{2 \pi i (k-l)} e^{i (k-l) x}\vert_{0}^{2\pi} = 0,
      \quad &&\text{if }k\neq l.
    \end{aligned}
    \right.
  \]
  Next, for real-valued $f \in L^2(0, 2\pi)$
  we compute the Fourier coefficients
  \[
    c_k = \left< f, \phi_k \right>
    = \frac{1}{\sqrt{2\pi}} \int_0^{2\pi} f \cdot e^{-i k x} \rmd x
    = \frac{1}{\sqrt{2\pi}} \left(
    \int_0^{2\pi} f(x) \cos(kx) \rmd x
    - i \int_0^{2\pi} f(x) \sin(kx) \rmd x
    \right).
  \]
  From Bessel's inequality $\sum |c_k|^2 \le \Vert f \Vert_2^2$
  we know $c_k \rightarrow 0$, and thus
  \begin{equation}
    \lim_{k \rightarrow \infty} \int_0^{2\pi} f(x) \cos(kx) \rmd x
    = \lim_{k \rightarrow \infty} \int_0^{2\pi} f(x) \sin(kx) \rmd x
    = 0.
    \label{eq:RiemannLebesgueLemma}
  \end{equation}

  Now assume $f \in L^1(0, 2\pi)$.
  For $n > 0$, define
  \[
    f_n(x) = \left\{
    \begin{aligned}
      & f(x), \quad &&\text{if }\left\vert f(x) \right\vert \le n, \\
      & 0, \quad &&\text{otherwise}.
    \end{aligned}
    \right.
  \]
  Then $|f_n| \nearrow |f|$,
  so by Dominated Convergence Theorem we have $\lim \int |f_n| = \int |f|$.
  For each $\epsilon > 0$,
  there exists a sufficiently large $N$ such that
  \[
    \int_0^{2\pi} \left| f - f_N \right|
    = \int_0^{2\pi} \left| f \right| - \left| f_N \right|
    < \epsilon/2.
  \]
  Note that $f_N \in L^2(0, 2\pi)$ since
  $\int \left| f_N \right|^2 \le \int \left| f_N \right| N
  \le N \left\Vert f \right\Vert_1 < +\infty$.
  By what has been proved,
  $| \int_0^{2\pi} f_N \cos(kx) \rmd x | < \epsilon/2$
  for all sufficiently large $k$.
  Combining the above inequalities, we obtain
  \[
    \left| \int_0^{2\pi} f \cos(kx) \rmd x \right|
    \le \left| \int_0^{2\pi} f_N \cos(kx) \rmd x \right|
    + \int_0^{2\pi} \left|f - f_N\right| \rmd x
    < \epsilon/2 + \epsilon/2 = \epsilon
  \]
  for all sufficiently large $k$.
  This establishes the first limit in \eqref{eq:RiemannLebesgueLemma}
  for $f \in L^1(0, 2\pi)$,
  and the second follows from exactly the same argument.
\end{exercise}

\begin{exercise}{8.16}
  Suppose $f_k \rightarrow f$ in $L^p$,
  i.e.\ $\Vert f_k - f \Vert_p \rightarrow 0$.
  For each $g \in L^{p'}$, we use H\"{o}lder's inequality to find
  \[
    \left| \int f_k g - \int f g \right|
    \le \int \left| (f_k - f) g \right|
    \le \left\| f_k - f \right\|_p \left\| g \right\|_{p'}
    \rightarrow 0,
  \]
  so $f_k \rightarrow f$ weakly in $L^p$.
  The converse is not true.
  Exercise 15 shows that $\cos(kx)$
  converges weakly to the zero function in $L^2(0, 2\pi)$.
  However, we have
  \[
    \int_0^{2\pi} \cos^2(kx) \rmd x
    = \int_0^{2\pi} \frac{1}{2} \left[ \cos(2kx) + 1 \right] \rmd x
    = \pi + \frac{1}{2} \int_0^{2\pi} \cos(2kx) \rmd x
    = \pi,
  \]
  so $\cos(kx)$ does not converge strongly to the zero function.
\end{exercise}

\begin{exercise}{8.17}
  Denote the real inner product by $\langle f, g \rangle = \int fg$.
  Since $f_k \rightarrow f$ weakly in $L^2$,
  in particular we have
  $\langle f_k, f \rangle \rightarrow \langle f, f \rangle = \| f \|_2^2$.
  We then compute
  \[
    \begin{aligned}
      \left\| f_k - f \right\|_2^2
      &= \langle f_k - f, f_k - f \rangle \\
      &= \left\| f_k \right\|_2^2 + \left\| f \right\|_2^2
      - 2 \langle f_k, f \rangle \\
      &\rightarrow \left\| f \right\|_2^2 + \left\| f \right\|_2^2
      - 2 \left\| f \right\|_2^2 \\
      &= 0.
    \end{aligned}
  \]
  Thus $f_k \rightarrow f$ in $L^2$ norm.
\end{exercise}
\begin{exercise}{8.28} We will show that $L_\phi(E)$ is a Banach space with norm $\|\cdot\|_{L_\phi(E)}$ in three steps below.

($B_1$) $L_\phi(E)$ is a linear space over $\mathbf{R}$:
If $f,g\in L_\phi(E)$ and $\alpha\in\mathbf{R}$, then $f+g$ and $\alpha f$ are measurable and finite a.e.\ in $E$. Since $\phi$ is increasing and $\phi(2t)\leq c\phi(t)$ for some constant $c>0$ independent of $t$, we have
\begin{equation*}
    \begin{aligned}
   &\phi(|f+g|)\leq\phi(|f|+|g|)\leq \phi(2\max\{|f|,|g|\})\leq c\phi(\max\{|f|,|g|\})\leq c\left[\phi(|f|)+\phi(|g|)\right]\in L(E),\\
   &\phi(|\alpha f|)=\phi(2|2^{-1}\alpha f|)\leq c\phi(|2^{-1}\alpha f|)\leq\cdots\leq c^k\phi(|2^{-k}\alpha f|)\leq c^k\phi(|f|)\in L(E)
    \end{aligned}
\end{equation*}
for some $k$ such that $|2^{-k}\alpha |\leq 1$. Thus $\phi(|f+g|), \phi(|\alpha f|)\in L(E)$ since $\phi$ is positive. This proves $f+g,\alpha f\in L_{\phi}(E)$.

($B_2$) $L_\phi(E)$ is a normed space: let us first prove that for every $f\in L_{\phi}(E)$,  $$0\leq\|f\|_{L_\phi(E)}=\inf\left\{\lambda>0:\int_E\phi\left(\frac{|f|}{\lambda}\right)\rmd\bfx\leq1\right\}<+\infty.$$
The fact that $\|f\|_{L_\phi(E)}\geq 0$ is obvious. Suppose $\|f\|_{L_\phi(E)}=\infty$ for some $f\in L_{\phi}(E)$ and let $\psi_n=\phi\left(\frac{|f|}{n}\right)$, then $\int_E\psi_n\rmd\bfx>1$ for any integer $n\geq1$. Since $f$ is finite a.e. in $E$, $\phi$ is continuous on $[0,\infty)$ and $\phi(0)=0$, we have $\psi_n\to 0$ a.e.\ in $E$. Moreover, $0\leq \psi_n\leq\phi(|f|)\in L(E)$ for all $n\geq 1$ since $\phi$ is increasing. By Lebesgue's dominated convergence theorem, it follows that $1<\int_E\psi_n\rmd\bfx\to 0$. This contradiction proves that $\|\cdot\|_{L_\phi(E)}<\infty$.
\begin{itemize}
    \item[(a)] It is obvious that $\|f\|_{L_\phi(E)}=0$ for $f=0$ a.e.\ on $E$ since $\phi(0)=0$. 
    Conversely, suppose $\|f\|_{L_\phi(E)}=0$ and $|f|>0$ on a subset $F\subset E$ with $|F|>0$. Then for $\lambda_n=1/n$, $n=1,2,\cdots$, we have $\int_F\phi\left(\frac{|f|}{\lambda_n}\right)\rmd\bfx\leq 1$ and $\lim_{n\to\infty}\phi\left(\frac{|f|}{\lambda_n}\right)=\infty$ on $F$. By Fatou's lemma, 
    $$\infty=\int_F\lim_{n\to\infty}\phi\left(\frac{|f|}{\lambda_n}\right)\rmd\bfx\leq\liminf_{n\to\infty}\int_F\phi\left(\frac{|f|}{\lambda_n}\right)\rmd\bfx\leq 1.$$
    This contradiction proves that $f=0$ a.e.\ on $E$ if  $\|f\|_{L_\phi(E)}=0$.
    \item[(b)] For $\alpha\in\mathbf{R}$ and $f\in L_\phi(E)$. It is obvious that $\|\alpha f\|_{L_\phi(E)}=|\alpha|\|f\|_{L_\phi(E)}$ provided $\alpha=0$. Now we assume that $\alpha\neq 0$. Then for any $\lambda>0$ with $\int_E\phi\left(\frac{|f|}{\lambda}\right)\rmd\bfx\leq1$, we have
    $$\int_E\phi\left(\frac{|\alpha f|}{|\alpha|\lambda}\right)\rmd\bfx\leq1,$$
 which implies $|\alpha|\lambda\geq\|\alpha f\|_{L_\phi(E)}$ and hence $|\alpha|\|f\|_{L_\phi(E)}\geq\|\alpha f\|_{L_\phi(E)}$. On the other hand, for any $\lambda>0$ with $\int_E\phi\left(\frac{|\alpha f|}{\lambda}\right)\rmd\bfx\leq1$, we have
 $$\int_E\phi\left(\frac{|f|}{\lambda/|\alpha|}\right)\rmd\bfx\leq1,$$
 which implies $\lambda/|\alpha|\geq\|f\|_{L_\phi(E)}$ and then $\|\alpha f\|_{L_\phi(E)}\geq|\alpha|\|f\|_{L_\phi(E)}$. This proves $\|\alpha f\|_{L_\phi(E)}=|\alpha|\|f\|_{L_\phi(E)}$.
    \item[(c)] Let $f,g \in L_\phi(E)$. Then for any $\lambda,\mu>0$ with $\int_E\phi\left(\frac{|f|}{\lambda}\right)\rmd\bfx\leq1$ and $\int_E\phi\left(\frac{|g|}{\mu}\right)\rmd\bfx\leq1$, the monotonicity and convexity of $\phi$ imply that
    $$\phi\left(\frac{|f+g|}{\lambda+\mu}\right)\leq \phi\left(\frac{|f|+|g|}{\lambda+\mu}\right)=\phi\left(\frac{\lambda}{\lambda+\mu}\frac{|f|}{\lambda}+\frac{\mu}{\lambda+\mu}\frac{|g|}{\mu}\right)\leq\frac{\lambda}{\lambda+\mu}\phi\left(\frac{|f|}{\lambda}\right)+\frac{\mu}{\lambda+\mu}\phi\left(\frac{|g|}{\mu}\right).$$
    Then $\int_E\phi\left(\frac{|f+g|}{\lambda+\mu}\right)\rmd\bfx\leq1$ and thus $\|f+g\|_{L_\phi(E)}\leq\lambda+\mu$. Since $\lambda$ and $\mu$ are arbitrary, we get $\|f+g\|_{L_\phi(E)}\leq\|f\|_{L_\phi(E)}+\|g\|_{L_\phi(E)}$.
\end{itemize}

($B_3$) $L_\phi(E)$ is complete with respect to norm $\|\cdot\|_{L_\phi(E)}$:
suppose $\{f_n\}$ is a Cauchy sequence in ${L_\phi(E)}$, then there is an increasing sequence of positive integers $n_k$ which forms a subsequence $\{f_{n_k}\}$ such that
$$\|f_n-f_{n_k}\|_{L_\phi(E)}\leq 2^{-k},\quad \forall n\geq n_k.$$

Consider the partial sums $S_m=\sum_{k=1}^{m}(f_{n_{k+1}}-f_{n_k})=f_{n_{m+1}}-f_{n_1}$ and $\widetilde{S}_m=\sum_{k=1}^{m}|f_{n_{k+1}}-f_{n_k}|$ which belong to $L_\phi(E)$ since $L_\phi(E)$ is a linear space. We denote by $\widetilde{S}=\lim_{m\to\infty}\widetilde{S}_m=\sum_{k=1}^{\infty}|f_{n_{k+1}}-f_{n_k}|$ which exists (but may be infinite) a.e.\ on $E$ since $\widetilde{S}_m$ is increasing. 
The fact that
$$\|\widetilde{S}_m\|_{L_\phi(E)}\leq\sum_{k=1}^{m}\|f_{n_{k+1}}-f_{n_k}\|_{L_\phi(E)}\leq\sum_{k=1}^m2^{-k}<1$$
implies $\int_E\phi(|\widetilde{S}_m|)\rmd\bfx\leq 1$ just by the definition of $\|\cdot\|_{L_\phi(E)}$. By the continuity of $\phi\geq0$ and Fatou's lemma, we have
$$\int_E\phi(|\widetilde{S}|)\rmd\bfx=\int_E\lim_{m\to\infty}\phi(|\widetilde{S}_m|)\rmd\bfx\leq\liminf_{m\to\infty}\int_E\phi(|\widetilde{S}_m|)\rmd\bfx\leq1,$$
which proves $\phi(|\widetilde{S}|)\in L(E)$ and hence $\widetilde{S}$ is finite a.e.\ in $E$ since $\lim_{t\to\infty}\phi(t)=\infty$. Thus $S=\lim_{m\to\infty}S_m=\sum_{k=1}^\infty(f_{n_{k+1}}-f_{n_k})$ exists and is finite a.e.\ in $E$. Moreover, since $\phi\geq0$ is increasing and $|S|\leq|\widetilde{S}|$ a.e., we have $\phi(|S|)\in L(E)$ and then $S\in L_\phi(E)$.

Now we can show that $S_m\to S$ in ${L_\phi(E)}$ as $m\to\infty$:
$$\|S_m-S\|_{L_\phi(E)}\leq\sum_{k=m+1}^{\infty}\|f_{n_{k+1}}-f_{n_k}\|_{L_\phi(E)}\leq \sum_{k=m+1}^{\infty}2^{-k}=2^{-m}\to 0.$$
Note that $f_{n_m}=S_{m-1}+f_{n_1}$, thus $f_{n_m}\to S+f_{n_1}\eqqcolon f$ in ${L_\phi(E)}$ as $m\to\infty$. Recall that $\{f_m\}$ is a Cauchy sequence in ${L_\phi(E)}$, then 
$$\|f_m-f\|_{L_\phi(E)}\leq\|f_m-f_{n_m}\|_{L_\phi(E)}+\|f_{n_m}-f\|_{L_\phi(E)}\to0$$
as $m\to\infty$. 
\end{exercise}
\begin{exercise}{8.32}[Convolution on the multiplicative group $(\mathbf{R}^+,\frac{\rmd y}{y})$] Given $x\in(0,\infty)$,
the change of variable $z = x / y$ yields that
$$F(x)=\int_0^\infty f\left(\frac{x}{y}\right)g(y)\frac{\rmd y}{y}=\int_0^\infty f(z)g\left(\frac{x}{z}\right)\frac{\rmd z}{z}.$$

For $p=\infty$, we have
    $F(x)\leq [g]_\infty\int_0^\infty f\left(z\right)\frac{\rmd z}{z}=[g]_\infty[f]_1$,
which proves $[F]_\infty\leq [f]_1[g]_\infty$.

For $1\leq p<\infty$, note that
\begin{equation*}
    \begin{aligned}
    [F]_p&=\left[\int_0^\infty\left(\int_0^\infty f(z)g\left(\frac{x}{z}\right)\frac{\rmd z}{z}\right)^p\frac{\rmd x}{x}\right]^{1/p}\\
    (\text{by Minkowski's inequality},\textbf{Exercise}\ \ref{88})&\leq\int_0^\infty\left[\int_0^\infty\left( f(z)g\left(\frac{x}{z}\right)\frac{1}{z}\frac{1}{x^{1/p}}\right)^p\rmd x\right]^{1/p}\rmd z\\
    &=\int_0^\infty\left[\int_0^\infty g\left(\frac{x}{z}\right)^p\frac{\rmd x}{x}\right]^{1/p}f(z)\frac{\rmd z}{z}\\
    &=\int_0^\infty\left[\int_0^\infty g\left(x\right)^p\frac{\rmd x}{x}\right]^{1/p}f(z)\frac{\rmd z}{z}\\
    &=[f]_1[g]_p.
    \end{aligned}
\end{equation*}
\end{exercise}


  \section{Approximations of the Identity and Maximal Functions}
\begin{lemma}[Generalized H\"{o}lder's Inquality]
  Let $1 \le p_i, r \le \infty$ and
  $\sum_{i=1}^k \frac{1}{p_i} = \frac{1}{r}$.
  Then
  \[
    \left\| f_1 \cdots f_k \right\|_r \le
    \left\| f_1 \right\|_{p_1} \cdots \left\| f_k \right\|_{p_k}.
  \]
\end{lemma}
\begin{proof}
  We first prove the case $k=2$.
  If $p_1 = \infty$, then $p_2 = r$,
  $|f_1 f_2| \le \|f_1\|_{\infty} |f_2|$ a.e.,
  and by homogeneity $\|f_1 f_2\|_r \le \|f_1\|_{\infty} \|f_2\|_{p_2}$.
  The case $p_2 = \infty$ is similar.
  If $p_1, p_2 < \infty$, then $r < \infty$ and $1 \le p_i/r < \infty$.
  Applying H\"{o}lder's inequality with exponents $p_i/r$,
  we obtain
  \[
    \left\| f_1 f_2 \right\|_r^r
    \le \int \left| f_1 f_2 \right|^r
    \le \left( \int \left| f_1 \right|^{p_1} \right)^{r/p_1}
    \left( \int \left| f_2 \right|^{p_2} \right)^{r/p_2}
    = \left\| f_1 \right\|_{p_1}^r \left\| f_2 \right\|_{p_2}^r.
  \]
  Taking the $r$th root and we are done with $k=2$.
  Now assume the conclusion holds for $k-1$ terms.
  Let $\sum_{i=2}^{k} \frac{1}{p_i} = \frac{1}{r'}$.
  Then we have
  \[
    \begin{aligned}
      \left\| f_1 \cdots f_k \right\|_r
      &\le \left\| f_1 \right\|_{p_1} \left\| f_2 \cdots f_k \right\|_{r'}
      \quad &&(\text{induction hypothesis for }
      \frac{1}{p_1} + \frac{1}{r'} = \frac{1}{r}) \\
      &\le \left\| f_1 \right\|_{p_1}
      \left\| f_2 \right\|_{p_2} \cdots \left\| f_k \right\|_{p_k}
      \quad &&(\text{induction hypothesis for }
      \sum_{i=2}^{k} \frac{1}{p_i} = \frac{1}{r'}) \\
      &= \left\| f_1 \right\|_{p_1} \cdots \left\| f_k \right\|_{p_k},
    \end{aligned}
  \]
  and the proof of the Lemma is completed.
\end{proof}


\begin{exercise}{9.2}
  (a)
  First assume $f, g \ge 0$.
  Then $f * g$ is measurable by Tonelli's Theorem.
  Consider the case $r = \infty$ where we have $\frac{1}{p} + \frac{1}{q} = 1$.
  Then we find
  \[
    \left| (f * g)(\bfx) \right| \le \int \left| f(\bft) g(\bfx-\bft) \right| \rmd t
    \le \left\| f \right\|_p \left\| g(\bfx-\cdot) \right\|_q
    = \left\| f \right\|_p \left\| g \right\|_q
  \]
  for each $\bfx \in \bbR^n$, so $\| f*g \|_{\infty} \le \|f\|_p \|g\|_q$.
  Next consider the case $r < \infty$,
  where $\frac{1}{p} + \frac{1}{q} = 1 + \frac{1}{r} > 1$
  and $p, q < \infty$.
  We write
  \[
    (f*g)(\bfx) = \int f(\bft) g(\bfx-\bft) \rmd \bft
    = \int f(\bft)^{\frac{p}{r}} g(\bfx-\bft)^{\frac{q}{r}}
    \cdot f(\bft)^{p(\frac1p - \frac1r)}
    \cdot g(\bfx - \bft)^{q(\frac1q - \frac1r)} \rmd \bft.
  \]
%  Applying the H\"{o}lder's inequality for three functions
  Applying the Lemma
  with exponents
  $\frac{1}{r} + (\frac{1}{p} - \frac{1}{r})
  + (\frac{1}{q} - \frac{1}{r}) = 1$,
  then raising to the $r$th power,
  we find
  \[
    \begin{aligned}
    (f*g)(\bfx)^r
    &\le
    \left( \int f(\bft)^p g(\bfx-\bft)^q \rmd \bft \right)
    \left( \int f(\bft)^p \rmd \bft \right)^{\frac rp - 1}
    \left( \int g(\bfx-\bft)^q \rmd \bft \right)^{\frac rq - 1} \\
    &= \left\| f \right\|_p^{r-p} \left\| g \right\|_{q}^{r-q}
      \left( \int f(\bft)^p g(\bfx-\bft)^q \rmd \bft \right).
    \end{aligned}
  \]
  Integrating the term in parentheses with respect to $\bfx$
  and applying Tonelli's Theorem, we obtain
  \[
    \int \int f(\bft)^p g(\bfx-\bft)^q \rmd \bft \rmd \bfx
    = \int f(\bft)^p \left( \int g(\bfx-\bft)^q \rmd \bfx \right) \rmd \bft
    = \left\| f \right\|_p^p \left\| g \right\|_q^q.
  \]
  It follows that
  \[
    \left\| f*g \right\|_r^r \le
    \left\| f \right\|_p^{r-p} \left\| g \right\|_{q}^{r-q}
    \left\| f \right\|_p^p \left\| g \right\|_q^q
    = \left\| f \right\|_p^r \left\| g \right\|_q^r.
  \]
  Taking the $r$th root we obtain the desired inequality
  $\left\| f*g \right\|_r \le \left\| f \right\|_p \left\| g \right\|_q$.

  For general $f \in L^p$ and $g \in L^q$,
  we show that $f * g$ is measurable.
  By the previous case we have $|f| * |g| \in L^r$.
  Therefore $|f| * |g| < \infty$ a.e.,
  $f(\bft) g(\bfx - \bft) \in L^1(\rmd \bft)$ for a.e.\ $\bfx$,
  and $f*g$ exists and is finite a.e.
  Next, define
  \[
    f_N = f \chi_{\{|\bfx| < N\}}, \quad
    g_N = g \chi_{\{|\bfx| < N\}}, \quad
    N = 1, 2, \cdots.
  \]
  Then $f_N, g_N \in L^1$,
  and $f_N * g_N$ is measurable by Fubini's Theorem.
  Note that
  \[
    \lim_{N \rightarrow \infty} f_N(\bft) g_N(\bfx - \bft)
    = f(\bft) g(\bfx - \bft)
    \text{ and }
    |f_N(\bft) g_N(\bfx - \bft)| \le |f(\bft) g(\bfx-\bft)|.
  \]
  By Dominated Convergence Theorem
  we have $\lim_{N \rightarrow \infty} (f_N * g_N) = f * g$ a.e.,
  so we know that $f * g$ is measurable.
  Finally, the inequality for general $f$ and $g$
  follows by noting that
  $\left| (f*g)(\bfx) \right| \le \left( |f|*|g| \right) (\bfx)$
  and applying the previous case to $|f|$ and $|g|$.

  (b) Define $f_{\lambda}(\bfx) = f(\lambda \bfx), \lambda > 0$.
  Then
  \[
    \left\| f_\lambda \right\|_p
    = \left( \int \left|f(\lambda \bfx)\right|^p \rmd \bfx \right)^{\frac 1p}
    = \left( \lambda^{-n} \int \left|f(\bfx)\right|^p \rmd \bfx \right)^{\frac 1p}
    = \lambda^{-\frac{n}{p}} \left\| f \right\|_p.
  \]
  Moreover, we have
  \[
    \begin{aligned}
      \left\| f_{\lambda}*g_{\lambda} \right\|_r
      &= \left( \int \left|
      \int f(\lambda\bft) g(\lambda\bfx-\lambda\bft) \rmd \bft
      \right|^r \rmd \bfx \right)^{\frac 1r} \\
      &= \left( \int \left|
      \lambda^{-n} \int f(\bft) g(\lambda\bfx-\bft) \rmd \bft
      \right|^r \rmd \bfx \right)^{\frac 1r} \\
      &= \lambda^{-n} \left( \lambda^{-n} \int \left|
      \int f(\bft) g(\bfx-\bft) \rmd \bft
      \right|^r \rmd \bfx \right)^{\frac 1r} \\
      &= \lambda^{-n - \frac nr} \left\| f*g \right\|_r.
    \end{aligned}
  \]
  Applying Young's Convolution Theorem to $f_{\lambda} * g_{\lambda}$,
  we obtain
  \[
    \lambda^{-n - \frac nr} \left\| f*g \right\|_r
    \le \lambda^{-\frac n p} \left\| f \right\|_p
    \lambda^{-\frac n q} \left\| g \right\|_q
    \Longleftrightarrow
    \left\| f*g \right\|_r \le
    \lambda^{n\left( 1 + \frac1r - \frac 1p - \frac 1q \right)}
    \left\| f \right\|_p \left\| g \right\|_q.
  \]
  If $n\left( 1 + \frac1r - \frac 1p - \frac 1q \right) > 0$
  (or $<0$, resp.),
  we can let $\lambda \rightarrow 0$
  (or $\rightarrow +\infty$, resp.)
  so that $\left\| f*g \right\|_r = 0$ for all $f \in L^p$ and $g \in L^q$.
  But this must be false, so we have
  \[
    \frac{1}{r} = \frac{1}{p} + \frac{1}{q} - 1.
    \qedhere
  \]
\end{exercise}

\begin{exercise}{9.3}
  (a) By H\"{o}lder's inequality, we have
  \[
    \left| (f*K)(\bfx) \right|
    \le \int \left| f(\bft) K(\bfx - \bft) \right| \rmd \bft
    \le \left\| f \right\|_p \left\| K \right\|_{p'},
  \]
  so $f*K$ is bounded in $\bbR^n$.
  If $1\le p' < \infty$ and $\bfx_1, \bfx_2 \in \bbR^n$,
  again by H\"{o}lder's inequality we have
  \[
    \begin{aligned}
      \left| (f*K)(\bfx_2) - (f*K)(\bfx_1) \right|
      &\le \int \left| f(\bft) \right| \cdot
      \left| K(\bfx_2 - \bft) - K(\bfx_1 - \bft)\right|
      \rmd \bft \\
      &\le \left\| f \right\|_p
      \left\| K(\bfx_2 - \cdot) - K(\bfx_1 - \cdot) \right\|_{p'}.
    \end{aligned}
  \]
  As $\bfx_2 \rightarrow \bfx_1$,
  the last norm term tends to zero
  due to continuity in $L^{p'}$ norm ($1 \le p' < \infty$).
  Thus  $f*K$ is continuous.
  If $p' = \infty$, then $p < \infty$,
  and the continuity follows from
  \[
    (f*K)(x) = \int f(\bft) K(\bfx - \bft) \rmd \bft
    = \int f(\bfx - \bft) K(\bft) \rmd \bft
    = (K*f)(x).
  \]

  (b)
%  Note that
%  \[
%    (\chi_I * \chi_J) (x) = \int \chi_I(t) \chi_J(x-t) \rmd t
%    = \mea{I \cap (x-J)}.
%  \]
  For $I = [a_1, b_1], J = [a_2, b_2]$,
  we sketch the graph of $\chi_I * \chi_J$ in Figure \ref{fig:graphOfChiIChiJ}.
  \begin{figure}[htbp]
    \centering
    \begin{subfigure}[b]{.4\linewidth}
      \includegraphics[height=0.4\linewidth]{pst/convGraph_1}
      \caption{Case $\mea{I} \ge \mea{J}$. }
    \end{subfigure}
    \begin{subfigure}[b]{.4\linewidth}
      \includegraphics[height=0.4\linewidth]{pst/convGraph_2}
      \caption{Case $\mea{I} < \mea{J}$. }
    \end{subfigure}%

    \begin{subfigure}[b]{.4\linewidth}
      \includegraphics[height=0.4\linewidth]{pst/convGraph_3}
      \caption{Case $I = J$. }
    \end{subfigure}%
    \caption{Graph of $\chi_I * \chi_J$. }
    \label{fig:graphOfChiIChiJ}
  \end{figure}
\end{exercise}
\begin{exercise}{9.4}\label{94}
  (a) \textbf{Claim}: For any positive integer $m$, the derivative of order $m$ of $h$ has the form $$h^{(m)}(x)=\sum_{i=1}^{N(m)}c_i(m)\frac{1}{x^i}e^{-1/x^2}\quad \text{for}\ x>0,$$
  where $N(m)$ is an integer and $c_i(m)$ are constants depend on $m$.
  
  We will prove the claim by induction on $m$. For $m=1$, we have $h^\prime(x)=2\frac{1}{x^3}e^{-1/x^2}$ for $x>0$. Assuming that the formula holds for $m$, we have
  $$h^{(m+1)}=\frac{\rmd}{\rmd x}\sum_{i=1}^{N(m)}c_i(m)\frac{1}{x^i}e^{-1/x^2}=\sum_{i=1}^{N(m)}c_i(m)\left(\frac{2}{x^{i+3}}-\frac{i}{x^{i+1}}\right)e^{-1/x^2}\quad \text{for}\ x>0.$$
  This proves the claim.
  
 Note that $h^{(m)}(x)=0$ for $x<0$ for any positive integer $m$. It is obvious that $h$ is continuous on $\mathbf{R}$ and is infinitely differentiable on $\mathbf{R}\backslash\{0\}$. Therefore, to prove $h\in C^\infty$, we only need to show that $h$ is infinitely differentiable at $x=0$. We will prove $h\in C^m$ and $h^{(m)}(0)=0$ for any positive integer $m$ by induction. For $m=1$, we have
  \begin{equation*}
      \begin{aligned}
         h_+^{\prime}(0)&=\lim_{x\to 0+}\frac{h(x)-h(0)}{x-0}=\lim_{x\to 0+}\frac{e^{-1/x^2}}{x}=\lim_{t\to+\infty}\frac{t}{e^{t^2}}=0,\\
             h_-^{\prime}(0)&=\lim_{x\to 0-}\frac{h(x)-h(0)}{x-0}=\lim_{x\to 0-}\frac{0}{x}=0,
      \end{aligned}
  \end{equation*}
  which imply that $h\in C^1$ and $h^\prime(0)=0$. Suppose that this statement is true for $m$. Then for $m+1$, we have 
  \begin{equation*}
      \begin{aligned}
         h_+^{(m+1)}(0)&=\lim_{x\to 0+}\frac{h^{(m)}(x)-h^{(m)}(0)}{x-0}= \lim_{x\to 0+}\frac{\sum_{i=1}^{N(m)}c_i(m)\frac{1}{x^i}e^{-1/x^2}}{x}=\sum_{i=1}^{N(m)}c_i(m)\lim_{t\to+\infty}\frac{t^{i+1}}{e^{t^2}}=0,\\
          h_-^{(m+1)}(0)&=\lim_{x\to 0-}\frac{h^{(m)}(x)-h^{(m)}(0)}{x-0}=\lim_{x\to 0-}\frac{0}{x}=0,
      \end{aligned}
  \end{equation*}
  which imply that $h\in C^{(m+1)}$ and $h^{(m+1)}(0)=0$. This proves $h\in C^\infty$.
  
  (b) For $a<b$, we have $$g(x)=h(x-a)h(b-x)=e^{-\left[\frac{1}{(x-a)^2}+\frac{1}{(b-x)^2}\right]}\chi_{(a,b)}.$$ Thus $\operatorname{supp} g=\overline{\{x:g(x)\neq 0\}}=[a,b]$. Since $h(x)\in C^\infty$, it follows that the compositions $h(x-a),h(b-x)\in C^\infty$ and then the product $g(x)=h(x-a)h(b-x)\in C^\infty$.
  
  (c) For $\bfx=(x_1,\cdots,x_n)\in\bfrn$, we denote the Euclidean norm by $|\bfx|=(x_1^2+\cdots+x_n^2)^{1/2}$ and define 
  $$  \eta(\bfx)= \begin{cases}e^{\frac{1}{|\bfx|^2-1}} & \text { if } |\mathbf{x}|<1,  \\ 0 & \text { if } |\mathbf{x}|\geq 1.\end{cases}$$
  Then $\eta\in C_0^\infty(\bfrn)$ with support $\{\bfx:|\bfx|\leq1\}$.
  Given a closed ball $\bar{B}(\bfy,\epsilon)=\{\bfx:|\bfx-\bfy|\leq\epsilon\}$ of radius $\epsilon$ centred at $\bfy$. Let $\eta_{\epsilon,\bfy}(\bfx)=\eta\left(\frac{\bfx-\bfy}{\epsilon}\right)$, then $\eta_{\epsilon,\bfy}(x)\in C_0^\infty(\bfrn)$ with support $\bar{B}(\bfy,\epsilon)$. 
  
  Given an interval $I=\{\bfx=(x_1,\cdots,x_n): a_k\leq x_k\leq b_k, k=1,\cdots,n\}$, we define $$\rho(\bfx)=\prod_{k=1}^ng(x_k)=\prod_{k=1}^nh(x_k-a_k)h(b_k-x_k),$$
  where $h$ and $g$ are as defined in (a) and (b) respectively. Then $\rho\in  C_0^\infty(\bfrn)$ with support $I$.
\end{exercise}
\begin{exercise}{9.5} We will show that such function $h$ exists in three steps below.

  Step 1: We can choose an open set $G_2$ such that $\overline{G}_1\subset G_2$, $\overline{G}_2\subset G$.
  
  In fact, the boundary $\partial G=\overline{G}\backslash G=\overline{G}\cap G^c$ is compact and nonempty since $G$ is open and bounded. Similarly, $\partial G_1$ is also compact and nonempty.
   Note that $\overline{G}_1\subset G$, thus $\partial G_1\cap\partial G=\varnothing$ and then $d(\partial G,\partial G_1)=\inf\{|\bfx-\bfy|:\bfx\in\partial G,\bfy\in\partial G_1\}>0$. Let $\delta=d(\partial G,\partial G_1)/3$ and $G_2=\{\bfx:|\bfx-\bfy|<\delta\ \text{for some}\ \bfy\in G_1\}$, then $G_2$ is open and $\overline{G}_1\subset G_2\subset\overline{G}_2\subset G$.
   
   Step 2: we can find a $K\in C^\infty$ with $\operatorname{supp} K=\overline{B}_{\delta}=\{\bfx:|\bfx|\leq\delta\}$ and $\int K=1$.
   
   Recall from \textbf{Exercise} \ref{94} (c) that 
   $$K(\bfx)=\frac{C}{\delta^n}\eta\left(\frac{\bfx}{\delta}\right),$$ where $C=\left(\int \eta\right)^{-1}$,
   satisfies our requirements.
   
   Step 3: Let $h=\chi_{G_2}\ast K$, where $G_2$ and $K$ are as defined in Step 1 and Step 2 respectively, then $h\in C_0^\infty$ such that $h=1$ in $G_1$ and $h=0$ outside $G$.
   
   Since $G_2$ is bounded and $K\in C_0^\infty$, the function $h$ is $C_0^\infty$ with 
   $$\operatorname{supp} h\subset \operatorname{supp}\chi_{G_2}+\operatorname{supp}K=\overline{G}_2+\overline{B}_{\delta}\subset G.$$
   (Note that the validity of the last inclusion depends on the selection of $\delta$ in Step 1.) This shows that $h=0$ outside $G$. For $\bfx\in G_1$ and $\bft\notin G_2$, by the definition of $G_2$, we have $|\bfx-\bft|\geq \delta$. Since $\operatorname{supp}K=\overline{B}_\delta$, we have $K(\bfx-\bft)=0$. Therefore, for any $\bfx\in G_1$, it follows that $\chi_{G_2}(\bft)K(\bfx-\bft)=K(\bfx-\bft)$ and then 
   $$h(\bfx)=\int_{\bfrn}\chi_{G_2}(\bft)K(\bfx-\bft)\rmd\bft=\int_{\bfrn} K(\bfx-\bft)\rmd\bft=\int_{\bfrn} K(\bft)\rmd\bft=1.$$
\end{exercise}
\begin{exercise}{9.6}
  Suppose $|K|\leq M$ for some $M>0$. Then for any $\bfx\in\bbR^n$, we have
  \begin{align*}
      |f*K(\bfx)|&=\bigg|\int_{\bbR^n}f(\bft)K(\bfx-\bft)\rmd\bft\bigg|\\
      &\leq\int_{\bbR^n}|f(\bft)K(\bfx-\bft)|\rmd\bft\\
      &\leq M\int_{\bbR^n}|f(\bft)|\rmd\bft<\infty,
  \end{align*}
  which implies $f*K$ is bounded.
  
  Now for any given $\epsilon>0$, since $K$ is uniformly continuous, there exists $\delta>0$ such that as long as $|\bfx-\bfx'|<\delta$ we have $|K(\bfx)-K(\bfx')|<\epsilon.$ Then for such $\bfx$ and $\bfx'$ we have
  \begin{align*}
      |f*K(\bfx)-f*K(\bfx')|&=\bigg|\int_{\bbR^n}f(\bft)(K(\bfx-\bft)-K(\bfx'-\bft))\rmd\bft\bigg|\\
      &\leq \int_{\bbR^n}|f(\bft)(K(\bfx-\bft)-K(\bfx'-\bft))|\rmd\bft\\
      &\leq \epsilon\int_{\bbR^n}|f(\bft)|\rmd\bft.
  \end{align*}
  Thus $f*K$ is uniformly continuous.
\end{exercise}

\begin{exercise}{9.7}
  Let $g(x,y,t)=f(t)P_y(x-t)$, then 
  \[
  \frac{\partial g}{\partial x}(x,y,t)=f(t)\frac{-2y(x-t)}{[y^2+(x-t)^2]^2}.
  \]
  When $|x|\leq N$, we then have 
  \[
    \bigg|\frac{\partial g}{\partial x}(x,y,t)\bigg|\leq |f(t)|\frac{2y(t+N)}{y^4+(|t|-N)^4}.
  \]
  Denote the right hand side function by $h(y,t)$. By Holder's Inequality
  \[
    \int_\bbR h(y,t)\rmd t\leq \pnorm{f}\bigg(\int_\bbR\bigg[\frac{2y(t+N)}{y^4+(|t|-N)^4}\bigg]^{p'}\rmd t\bigg)^{1/{p'}}<\infty.
  \]
  Thus $h(y,t)\in L^1(\bbR)$. Now for fixed $x$ and $y$, choose an arbitrary sequence $\{h_n\}$ converging to zero and we may assume $|h_i|\leq 1$ for all $i$. We may also assume $|x|\leq N-1$ for some $N$ sufficiently large. Then by mean value theorem
  \[
    \frac{g(x+h_n,y,t)-g(x,y,t)}{h_n}=\frac{\partial g}{\partial x}(x+\xi_n,y,t)
  \]
  for some $\xi_n$ between 0 and $h_n$. Hence by earlier result we have
  \[
    \bigg|\frac{g(x+h_n,y,t)-g(x,y,t)}{h_n}\bigg|\leq h(y,t).
  \]
  Now by dominated convergence theorem, we have
  \begin{align*}
      \lim_{n\rightarrow\infty}\frac{f(x+h_n,y,t)-f(x,y,t)}{h_n}&=\lim_{n\rightarrow\infty}\int_\bbR\frac{g(x+h_n,y,t)-g(x,y,t)}{h_n}\rmd t\\
      &=\int_\bbR \frac{\partial g}{\partial x}(x,y,t)\rmd t=\int_\bbR f(t)\frac{\partial P_y}{\partial x}(x-t)\rmd t.
  \end{align*}
  Therefore
  \[
    \frac{\partial f}{\partial x}(x,y)=\int_\bbR f(t)\frac{\partial P_y}{\partial x}(x-t)\rmd t.
  \]
  Using similar methods, we can calculate that 
  \[
    (\frac{\partial^2}{\partial x^2}+\frac{\partial^2}{\partial y^2})f(x,y)=\int_\bbR f(t)(\frac{\partial^2}{\partial x^2}+\frac{\partial^2}{\partial y^2})P_y(x-t)\rmd t
  \]
  and since $P_y(x)$ is harmonic when $y>0$, we deduce that $f(x,y)$ is harmonic when $y>0.$
\end{exercise}

\begin{exercise}{9.8}
  For any $s> 0$, since $K(s,t)=s^{-1}K(1,t/s)$, we have
  \[
    (Tf)(s)=\int_0^\infty f(t)s^{-1}K(1,t/s)\rmd t=\int_0^\infty f(st)K(1,t)\rmd t.
  \]
  Using results in Exercise 8.8, it follows that
  \begin{align*}
     \pnorm{Tf}&=\bigg[\int_0^\infty\bigg[\int_0^\infty f(st)K(1,t)\rmd t\bigg]^p\rmd s\bigg]^{1/p}\\
     &\leq \int_0^\infty\bigg[\int_0^\infty f(st)^p K(1,t)^p\rmd s\bigg]^{1/p}\rmd t\\
     &=\int_0^\infty\bigg[\int_0^\infty f(s)^p t^{-1}K(1,t)^p\rmd s\bigg]^{1/p}\rmd t\\
     &=\bigg[\int_0^\infty f(s)^p\rmd s\bigg]^{1/p}\bigg[\int_0^\infty t^{-1/p}K(1,t)\rmd t\bigg]\\
     &\leq\gamma\pnorm{f}.
     \qedhere
  \end{align*}
\end{exercise}

\begin{exercise}{9.13}
  For each $x \in (0, 1)$,
  if we let $I_k(x)$ be the unique interval
  among $I_{k,j}$ containing $x$,
  then $\{I_k(x)\}$ shrinks regularly to $x$.
  It follows from Lebesgue Differentiation Theorem
  that $f_k \rightarrow f$ a.e.
  By Exercise 8.12, to show $f_k \rightarrow f$ in $L^p(0,1)$,
  it suffices to show $\|f_k\|_p \rightarrow \|f\|_p$.
  To this end, we compute
  \[
    \begin{aligned}
      \int_0^1 \left|f_k\right|^p
      &= \sum_{j=1}^{2^k} \int_{I_{k,j}} \left|f_k\right|^p \\
      &= \sum_{j=1}^{2^k} 2^{-k} \left| 2^k \int_{I_{k,j}} f \right|^p \\
      &\le 2^{k(p-1)} \sum_{j=1}^{2^k}
      \left( \int_{I_{k,j}} \left|f\right|^p \right)
      \left( \int_{I_{k,j}} 1 \right)^{p/p'}
      \quad &&(\text{H\"{o}lder's inequality}) \\
      &= 2^{k(p-1)} 2^{-k(p-1)}
      \sum_{j=1}^{2^k} \int_{I_{k,j}} \left|f\right|^p
      \quad &&(p/p' = p-1) \\
      &= \int_0^1 \left| f \right|^p.
    \end{aligned}
  \]
  Taking limit superior, we obtain
  $\limsup \left\| f_k \right\|_p \le \left\| f \right\|_p$.
  However by Fatou's Lemma, we have
  $\left\| f \right\|_p \le \liminf \left\| f_k \right\|_p$.
  Combining the inequalities, we find $\lim \|f_k\|_p = \|f\|_p$
  and the proof is completed.
\end{exercise}

\end{document}