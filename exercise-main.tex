% Preamble
%\documentclass[11pt]{report}
\documentclass[11pt]{article}
% Packages
\usepackage[a4paper, margin=2.0cm]{geometry}
\usepackage{amsmath}
\usepackage{amssymb}
\usepackage{amsthm}
\usepackage{mathtools}
\usepackage{enumitem}
\usepackage{hyperref}

%\setlength\parindent{0pt}

% Self-defined commands
\newcommand{\calE}{\mathcal{E}}
\newcommand{\calI}{\mathcal{I}}

\newcommand{\bbQ}{\mathbb{Q}}
\newcommand{\bbR}{\mathbb{R}}
\newcommand{\bbZ}{\mathbb{Z}}

\newcommand{\rmd}{\mathrm{d}}

\newcommand{\bfe}{\mathbf{e}}
\newcommand{\bff}{\mathbf{f}}
\newcommand{\bfh}{\mathbf{h}}
\newcommand{\bfx}{\mathbf{x}}

\DeclareMathOperator*\lolim{\underline{lim}}
\DeclareMathOperator*\uplim{\overline{lim}}
\DeclareMathOperator{\dist}{dist}
\DeclareMathOperator{\vol}{vol}
%\DeclareMathOperator{\det}{det}

\newcommand{\mea}[1]{\left\vert#1\right\vert}
\newcommand{\outmea}[1]{\left\vert#1\right\vert_\mathrm{e}}
\newcommand{\inmea}[1]{\left\vert#1\right\vert_\mathrm{i}}

\newtheorem*{lemma}{Lemma}

\theoremstyle{definition}
\newtheorem{innerexercise}{Exercise}
\newenvironment{exercise}[1]
{\pushQED{\qed}\renewcommand\theinnerexercise{#1}\innerexercise}
{\popQED\endinnerexercise}

% Document
\begin{document}

  \title{\textbf{Measure and Integral}\\ Homework 3\footnote{
    Muhan: 4.3, 4.5, 5.21--24.
    Zhixuan: 4.15--18, 5.10, 5.18.
    Heqing: 4.12, 5.4--6, 5.9. }}
  \author{Luo Muhan, Li Zhixuan, Huang Heqing}
  \maketitle

  % Exercise part
  %=====================================================================
%  \chapter{Preliminaries}
%  \setcounter{chapter}{1}
%
%  \chapter{Functions of Bounded Variation and the Riemann-Stieltjes Integral}
%  
\begin{exercise}{2.14}
  This is the example following (2.28).
  Take $[a,b] = [-1, 1], c=0$, and
  \[
    f(x) = \left\{
    \begin{aligned}
      0 & \quad \text{if} \quad -1 \le x < 0 \\
      1 & \quad \text{if} \quad 0 \le x \le 1,
    \end{aligned}
    \right.
    \quad
    \phi(x) = \left\{
    \begin{aligned}
      0 & \quad \text{if} \quad -1 \le x \le 0 \\
      1 & \quad \text{if} \quad 0 < x \le 1.
    \end{aligned}
    \right.
  \]
  $R_\Gamma(f\rmd \phi; [-1, 1])$ can be 0 or 1 for any $\Gamma$ straddling $0$,
  so $\int_{-1}^1 f\rmd \phi$ does not exist.
  Since $\phi \equiv 0$ on $[-1, 0]$,
  $R_\Gamma(f \rmd \phi; [-1, 0])$ equals to zero for any $\Gamma$
  and hence $\int_{-1}^0 f \rmd \phi = 0$.
  Finally, if $\Gamma = \{x_j\}$ is a partion of $[0, 1]$, then
  \[
    R_\Gamma(f \rmd \phi; [0, 1]) = f(\xi_1) \left( \phi(x_1) - \phi(x_0) \right)
    = 1 \cdot (1-0) = 1.
  \]
  Hence $\int_0^1 f \rmd \phi = 1$.
\end{exercise}

\begin{exercise}{2.15}
  Let $\Gamma = \{x_j\}_1^N$ be a partition of $[a,b]$. Then
  \[
    S_\Gamma = \sum \left\vert \psi(x_j) - \psi(x_{j-1}) \right\vert
    = \sum \left\vert \int_{x_{j-1}}^{x_j} f \rmd \phi \right\vert
    \le \left(\sup |f|\right) \sum V(\phi; [x_{j-1}, x_j])
    \le \left(\sup |f|\right) V(\phi; [a,b]),
  \]
  where the first inequality follows from Theorem 2.24.
  This holds for all $\Gamma$ and thus $\psi$ is of bounded variation.
  If $g$ is continuous on $[a,b]$,
  both $\int_a^b g \rmd \psi$ and $\int_a^b fg \rmd \phi$ exist
  by Theorem 2.24.
  Consider the Riemann-Stieltjes sum for $g \rmd \psi$, we have
  \begin{align*}
    & \sum g(\xi_j) \left( \psi(x_j) - \psi(x_{j-1}) \right) \\
    = & \sum g(\xi_j) f(\eta_j) \left( \phi(x_j) - \phi(x_{j-1}) \right) \\
    = & \sum g(\xi_j) f(\xi_j) \left( \phi(x_j) - \phi(x_{j-1}) \right)
    + \sum g(\xi_j) \left( f(\eta_j) - f(\xi_j) \right)
    \left( \phi(x_j) - \phi(x_{j-1}) \right) \\
    \coloneqq & I_1 + I_2,
  \end{align*}
  where the first equality follows from Theorem 2.27 (Mean-Value Theorem)
  with each $\eta_j \in [x_{j-1}, x_j]$.
  The term $I_1$ is the Riemann-Stieltjes sum for $gf \rmd \phi$
  and tends to $\int_a^b fg \rmd \phi$ as $|\Gamma| \rightarrow 0$.
  Using the uniform continuity of $f$ on $[a,b]$,
  we can select $|\Gamma|$ so small that
  \[
    \left\vert f(\eta_j) - f(\xi_j) \right\vert <
    \frac{\epsilon} {\left( \sup g \right) V(\phi)}
  \]
  for every $1 \le j \le N$.
  It follows that $|I_2| < \epsilon$
  and consequently $\lim_{|\Gamma| \rightarrow 0} I_2 = 0$.
  The proof of $\int_a^b g \rmd \psi = \int_a^b fg \rmd \phi$ is thus completed.

\end{exercise}

\begin{exercise}{2.16}
  In view of Corollary 2.7 (Jordan's Theorem),
  we assume that $\phi$ is monotone increasing.
  Suppose $\sup f = M$ and $\{y_j\}_1^L$ are the jump discontinuities.
  By the continuity of $\phi$ at $\{y_j\}$,
  for each $j$ there exists a closed interval $I_j$ containing $y_j$ such that
  \begin{equation}
    \left| \phi(y) - \phi(z) \right| < \frac{\epsilon}{4 M L}
    \label{eq:estOfPhi}
  \end{equation}
  for any $y, z \in I_j$.
  By the uniform continuity of $f$ on $[a,b] \setminus \bigcup I_j^\circ$,
  select $\delta'$ so small that
  \begin{equation}
    \left| f(y) - f(z) \right| < \frac{\epsilon}{2 V(\phi)}
    \label{eq:estOfF}
  \end{equation}
  for $y,z \in [a,b] \setminus \bigcup I_j^\circ$ and $|y-z| < \delta'$.
  Now suppose $\Gamma = \{x_k\}$ is a partition of $[a,b]$ satisfying
  $\left| \Gamma \right| < \min \left\{ \delta', |I_1|, \cdots, |I_L| \right\}$.
  Divide the closed intervals in $\Gamma$ into two subcollections:
  Let  $I \in \Gamma_1$
  if $I$ has nonempty intersection with $\{y_j\}$
  and let $I \in \Gamma_2$ otherwise.
  We estimate the difference between the upper and lower Riemann-Stieltjes sums as follows.
  First by \eqref{eq:estOfPhi} we have
  \[
    \left| \sum_{[x_{k-1}, x_k] \in I_1}
    \left( M_k - m_k \right) \left( \phi(x_k) - \phi(x_{k-1}) \right) \right|
    \le 2M \cdot L \cdot \frac{\epsilon}{4ML}
    \le \frac{\epsilon}{2}
  \]
  since there are at most $L$ intervals in the subcollection $\Gamma_1$.
  Second by \eqref{eq:estOfF} we have
  \[
    \left| \sum_{[x_{k-1}, x_k] \in I_2}
    \left( M_k - m_k \right) \left( \phi(x_k) - \phi(x_{k-1}) \right) \right|
    \le \frac{\epsilon}{2V(\phi)} \cdot V(\phi)
    \le \frac{\epsilon}{2}.
  \]
  We have shown $\lim_{|\Gamma| \rightarrow 0} (U_\Gamma - L_\Gamma) = 0$.
  It follows from $L_\Gamma \le R_\Gamma \le U_\Gamma$ that
  the integral $\int_a^b f \rmd \phi = \lim R_\Gamma$ exists.
\end{exercise}

\begin{exercise}{2.17}
  In view of Corollary 2.7 (Jordan's Theorem),
  we assume that $\phi$ is monotone increasing.
  Set $S_n = \int_{-n}^n f \rmd \phi$.
  We show that the sequence $\{S_n\}$ is Cauchy.
  Let $\phi(-\infty) = M^-$ and $\phi(+\infty) = M^+$.
  Since $\phi$ is monotone increasing,
  there exists $N_1 > 0$ so large that
  $\phi(x) > M^+ - \sqrt{\epsilon} / 2$ for all $x > N_1$
  and $\phi(x) < M^- + \sqrt{\epsilon} / 2$ for all $x < -N_1$.
  Since $f$ is continuous and $f(\infty) = 0$,
  there exists $N_2 > 0$ so large that
  $|f(x)| < \sqrt{\epsilon}$ for all $|x| > N_2$.
  Now if $n \ge m > \max\{N_1, N_2\}$, we have
  \[
    \begin{aligned}
      \left| S_n - S_m \right| =
      & \left| \int_{-n}^{-m} f \rmd \phi + \int_m^n f \rmd \phi \right| \\
      \le & \left( \sup_{|x| \ge m} |f| \right)
      \left( V(\phi;[-n,-m]) + V(\phi;[m,n]) \right) \\
      \le & \sqrt{\epsilon}
      \left( \phi(-m) - \phi(-\infty) + \phi(+\infty) - \phi(m) \right) \\
      < & \sqrt{\epsilon}
      \left( \frac{\sqrt{\epsilon}}{2} + \frac{\sqrt{\epsilon}}{2} \right) \\
      = &\epsilon,
    \end{aligned}
  \]
  where the first inequality follows from Theorem 2.24.
  Thus the improper integral $\int_{-\infty}^{+\infty} f \rmd \phi = \lim S_n$ exists.
\end{exercise}

\begin{exercise}{2.31}
  For the first part,
  note that the Riemann-Stieltjes sum
  \[
    R_\Gamma(\rmd f; [a,b]) = \sum f(x_j) - f(x_{j-1}) = f(b) - f(a)
  \]
  is independent of the partition $\Gamma$.
  For the second part, we have by the Mean-Value Theorem
  \[
    \sum f(x_j) - f(x_{j-1}) = \sum f'(\xi_j) \left( x_j - x_{j-1} \right)
  \]
  with $\xi_j \in (x_{j-1}, x_j)$.
  Since $f'$ is Riemann integrable,
  by taking the limit $|\Gamma| \rightarrow 0$ of both sides we obtain
  $\int_a^b \rmd f = \int_a^b f' \rmd x$.
\end{exercise}

\begin{exercise}{2.32}
  Since $af(a)$ and $bf(b)$ are finite,
  it follows from Theorem 2.21 (integration by parts) that
  the (Riemann) integrability of $f \rmd x$
  is equivalent to that of $x \rmd f$.
  But $x$ is continuous and $f$ is of bounded variation on $[a,b]$,
  the integrability follows from Theorem 2.24.
\end{exercise}

%
%  \chapter{Lebesgue Measure and Outer Measure}
%  
\begin{exercise}{3.11}
  Suppose $E$ is measurable and $\mea{E} < + \infty$.
  Select an open set $G$ such that $G \supset E$ and $\mea{G-E} < \epsilon$.
  By Theorem 1.11, there exists a countable collection $\{I_k\}$
  of nonoverlapping intervals such that $G = \bigcup I_k$
  and therefore $\mea{G} = \sum \mea{I_k}$.
  Choose $N$ so large that $\sum_{N+1}^{\infty} \mea{I_k} < \epsilon$.
  Now take $S = \bigcup_1^N I_k, N_1 = \bigcup_{N+1}^\infty I_k$
  and $N_2 = G - E$.
  Then $E = (S \cup N_1) - N_2$, $\mea{N_1} < \epsilon$ and $\mea{N_2} < \epsilon$
  as desired.

  Conversely, suppose $E = (S \cup N_1) - N_2$,
  $\outmea{N_1} < \epsilon$ and $\outmea{N_2} < \epsilon$.
  We may assume $N_2 \subset (S \cup N_1)$ by taking appropriate intersection.
  Select an open set $G$ with $G \supset S$ and $\mea{G - S} < \epsilon$.
  Then select an open set $G_1$ with $G_1 \supset N_1$
  and $\mea{G_1} < \outmea{N_1} + \epsilon < 2 \epsilon$.
  By subadditivity of outer measure, we have
  \[
    \begin{aligned}
      \outmea{(G \cup G_1) - E} &= \outmea{(G \cup G_1) - (S \cup N_1) \cup N_2} \\
      &\le \outmea{(G \cup G_1) - (S \cup N_1)} + \outmea{N_2} \\
      &< \mea{G-S} + \outmea{G_1 - N_1} + \epsilon \\
      &< \epsilon + \outmea{G_1} + \epsilon \\
      &< 4\epsilon.
    \end{aligned}
  \]
  We have shown that $E$ can be approximated by open sets from outside.
  Thus $E$ is measurable.
\end{exercise}

\begin{exercise}{3.12}
  First let us fix $E_1$ to be a finite open interval.

  \begin{enumerate}[noitemsep]
  \item
  Since the measure of an interval is equal to its volume,
  the assertion $\mea{E_1 \times E_2} = \mea{E_1} \mea{E_2}$ holds
  for $E_2$ being an open interval.

  \item
  Suppose $E_2$ is an open set.
  By Theorem 1.10,
  we can write $E_2 = \bigcup I_k$
  where $\{I_k\}$ is a collection of disjoint open intervals.
  Then by what has been proved,
  \[
    E_1 \times E_2 = E_1 \times \left( \bigcup I_k \right)
    = \bigcup \left(E_1 \times I_k\right)
  \]
  is a countable union of disjoint measurable sets in $\bbR^2$
  and hence measurable.
  Moreover,
  \[
    \mea{E_1 \times E_2} = \sum \mea{E_1 \times I_k}
    = \mea{E_1} \sum \mea{I_k} = \mea{E_1} \mea{E_2}.
  \]

  \item
  If $\mea{E_2} = 0$, then we can find an open set $G \supset E_2$
  with $\mea{G} < \epsilon$. Then
  $\outmea{E_1 \times E_2} \le \mea{E_1 \times G} < \epsilon \mea{E_1}$
  for arbitrary $\epsilon > 0$.
  Hence $\mea{E_1 \times E_2} = 0$.

  \item
  Suppose $E_2 = \bigcap G_k$ is a $G_\delta$ set with finite measure.
  We may assume $\mea{G_1} < \infty$ and $\{G_k\}$ is decreasing.
  Then by what has been proved,
  \[
    E_1 \times E_2 = E_1 \times \left( \bigcap G_k \right)
    = \bigcap \left( E_1 \times G_k \right)
  \]
  is a countable decreasing intersection of measurable sets
  and hence measurable.
  Moreover,
  \[
    \mea{E_1 \times E_2}
    = \lim \mea{E_1 \times G_k}
    = \mea{E_1} \lim \mea{G_k} = \mea{E_1} \mea{E_2}.
  \]

  \item
  Suppose $E_2$ has finite measure.
  Write $E_2 = G - Z$ where $G$ is a $G_{\delta}$ set with finite measure
  and $Z$ has measure zero. Then by what has been proved,
  \[
    E_1 \times E_2 = E_1 \times \left( G-Z \right)
    = E_1 \times G - E_1 \times Z
  \]
  is the difference of two measurable sets and hence measurable.
  It also follows easily that
  $\mea{E_1 \times E_2} = \mea{E_1}\mea{G} - \mea{E_1}\mea{Z} = \mea{E_1}\mea{E_2}$.

  \item
  Suppose $E_2$ has infinite measure.
  We can write $E_2 = \bigcup E_k$
  where the collection $\{E_k\}$ is increasing and each $\mea{E_k} < \infty$.
  Then $E_1 \times E_2$ is a countable increasing union of measurable sets
  and hence measurable.
  Moreover, $\mea{E_1 \times E_2} = \mea{E_1} \lim {E_k} = \infty$.
  \end{enumerate}

  We have proved that the assertion $\mea{E_1 \times E_2} = \mea{E_1} \mea{E_2}$
  holds for $E_1$ being a finite open interval and $E_2$ an arbitrary measurable set.
  By interchanging the roles of $E_1$ and $E_2$
  and going through the above steps again,
  we can show that the assertion holds for arbitrary measurable $E_1$ and $E_2$.
\end{exercise}

\begin{exercise}{3.13}
  First recall that
  \[
    \outmea{E} = \inf \left\{ \mea{G} : G \text{ is open and } G \subset E\right\}.
  \]
  Since $\mea{F} \le \mea{G}$ for all closed $F \subset E$
  and all open $G \supset E$,
  taking infimum and supremum we obtain
  $\inmea{E} \le \outmea{E}$.
  This proves part (i).

  Suppose $E$ has finite measure.
  Lemma 3.22 states that for an arbitrary $\epsilon > 0$
  there exists a closed set $F \subset E$ with
  $\mea{E} - \mea{F} = \mea{E-F} < \epsilon$.
  It follows easily that $\inmea{E} = \mea{E} = \outmea{E}$.
  Conversely, suppose $\inmea{E} = \outmea{E} < \infty$.
  For each $k \in \bbZ^+$,
  select an open set $G_k$ and a closed set $F_k$ such that
  \[
    F_k \subset E \subset G_k, \quad \mea{G_k} - \mea{F_k} < 1/k.
  \]
  Let $G = \bigcap G_k$.
  Then $G$ is a $G_\delta$ (and hence measurable) set.
  Moreover, it follows from $G - E \subset G_k - F_k$ that
  \[
    \outmea{G-E} \le \mea{G_k - F_k} = \mea{G_k} - \mea{F_k} < 1/k
  \]
  for every $k \in \bbZ^+$.
  Hence $G-E$ has measure zero.
  Finally, being the difference of two measurable sets,
  $E = G - (G-E)$ is measurable.
  This completes the proof of part (ii).
\end{exercise}

\begin{exercise}{3.14}
  Let $E_1$ be a nonmeasurable subset of $[-2, -1]$.
  Let $E_2 = [0, +\infty)$ and $E = E_1 \cup E_2$.
  Clearly $E$ is nonmeasurable.
  Moreover, we have
  \[
    \begin{aligned}
      \outmea{E} \ge \outmea{E_2} = \infty \quad &\Rightarrow \quad \outmea{E} = \infty, \\
      \inmea{E} \ge \mea{[0, n]} = n \text{ for every } n \in \bbZ^+
      \quad &\Rightarrow \quad \inmea{E} = \infty.
    \end{aligned}
  \]
  Hence part (ii) of Exercise 13 is false if $\outmea{E} = \infty$.
\end{exercise}

\begin{exercise}{3.16}
  Recall that a parallelepiped is
  \[
    P = \left\{ \sum_1^n t_k \bfe_k : 0 \le t_k \le 1 \right\}.
  \]
  It is easy to show that $P$ is a compact, and hence measurable set.

  First we show $v(P) \le \mea{P}$.
  Suppose $\{I_k\}$ is a collection of intervals that covers $P$.
  We can enlarge each $I_k$ to some open interval $J_k$ such that
  $I_k$ is contained in $J_k^\circ$ and $v(J_k) < (1+\epsilon) v(I_k)$.
  By the Heine-Borel property,
  there exists a finite cover $\{J_k\}_1^N$ of $P$.
  It follows that
  \[
    v(P) \le \sum_1^N v(J_k) \le (1+\epsilon) \sum_1^N v(I_k)
    \le (1 + \epsilon) \sum_1^\infty v(I_k).
  \]
  Since this holds for all covers $\{I_k\}$ of $P$,
  taking infimum of the right-hand-side we obtain
  $v(P) \le (1 + \epsilon) \mea{P}$.
  But $\epsilon$ is arbitrary,
  so $v(P) \le \mea{P}$.

  To show $\mea{P} \le v(P)$,
  we first claim that for each $\epsilon > 0$
  there exists an open set $G \supset P$
  such that $\mea{G} \le v(P) + \epsilon$.
  If $\{\bfe_k\}$ is linearly independent,
  then $v(P) > 0$ and we can take
  \[
    G = \left\{ \sum_1^n t_k \bfe_k : -\delta < t_k < 1 + \delta \right\}.
  \]
  Then by p.\ 8 in Section 1.3 we know
  %the volume formula for a parallelepiped states that
  $v(G) = (1 + 2\delta)^n v(P) < v(P) + \epsilon$
  for sufficiently small $\delta$.
  Since $G$ is open,
  by Theorem 1.11 $G$ can be written as a countable union
  of the nonoverlapping intervals $\{I_j\}$.
  Then we have $\sum_1^N v(I_j) \le v(G)$ for every $N$;
  taking limit we obtain
  \[
    \mea{G} = \sum v(I_j) \le v(G) \le v(P) + \epsilon.
  \]

  If $\{\bfe_k\}$ is not linearly independent,
  let $U$ be the proper subspace of $\bbR^n$ spanned by $\{\bfe_k\}$
  and let $V$ be the orthogonal complement of $U$.
  %with orthonormal basis. $\{\bff_k\}_1^m$.
  Since $P$ is a compact subset of $U$,
  we can find in the subspace $U$
  a bounded open rotated interval $B_1$ that contains $P$.
  If $B_2$ is an open rotated interval centered at the origin
  in the subspace of $V$,
  then $P$ is contained in the open rotated interval $G = B_1 \times B_2$
  and $\mea{G} < v(P) + \epsilon = \epsilon$ for sufficiently small $B_2$.
  This proves the claim.

  Finally, it follows from the claim that for each $\epsilon > 0$
  we have $\mea{P} \le \mea{G} \le v(P) + \epsilon$ for suitably chosen $G \supset P$.
  Thus $\mea{P} \le v(P)$ and the proof of $\mea{P} = v(P)$ is completed.
\end{exercise}

\begin{exercise}{3.17}
  Let $C$ be the Cantor set and $G = [0, 1] - C$.
  Recall that the Cantor-Lebesgue function $f$ is continuous
  and maps $[0, 1]$ onto $[0, 1]$.
%  By definition we see that
  Observe that
  \[
    f(G) = \left\{ \frac{k}{2^n} : n \in \bbZ^+, 1 \le k \le 2^n-1 \right\}
  \]
  is a countable subset of $[0, 1]$ and therefore has measure zero.
  Let $E$ be a nonmeasurable subset of $[0, 1] - f(G)$.
  Since $E \cap f(G) = \emptyset$,
  we must have $f^{-1}(E) \subset C$.
  Since $\mea{C} = 0$,
  we see that $f^{-1}(E)$ has measure zero.
  But by construction $E = f (f^{-1}(E))$ is nonmeasurable.
\end{exercise}

\begin{exercise}{3.18}
  Suppose $E \subset \bbR^n$ and $\bfh \in \bbR^n$.
%  We show that $\outmea{E_{\bfh}} \le \outmea{E}$.
  Suppose $\{I_k\}$ is a cover of $E$ by intervals and
  $\sum v(I_k) < \outmea{E} + \epsilon$.
  Then $\{I_{k, \bfh}\}$ is a cover of $E_{\bfh}$ and
  \[
    \outmea{E_{\bfh}} \le \sum v(I_{k, \bfh}) = \sum v(I_k) < \outmea{E} + \epsilon,
  \]
  since the volume of an interval is translation-invariant.
  But $\epsilon$ is arbitrary, so $\outmea{E_{\bfh}} \le \outmea{E}$.
  By symmetry we have $\outmea{E} \le \outmea{E_{\bfh}}$.

  If $E$ is measurable, for each $\epsilon > 0$
  there exists an open set $G$ such that $G \supset E$ and $\mea{G-E} < \epsilon$.
  But $G_{\bfh}$ is open,
  $G_{\bfh} \supset E_{\bfh}$
  and $\outmea{G_{\bfh} - E_{\bfh}} = \mea{G - E} < \epsilon$
  by what has been proved.
  Therefore $E_{\bfh}$ is measurable.
\end{exercise}

To complete Exercise 19,
we extend Lemma 3.37 to multidimensions.

\begin{lemma}
  Let $E \subset \bbR^n$ be a measurable set with $\mea{E} > 0$.
  Then the set of differences
  $\{d : d = x-y, x \in E, y \in E\}$
  contains an interval centered at the origin.
\end{lemma}
\begin{proof}
  Given $\epsilon > 0$, there exists an open set $G$
  such that $G \supset E$ and $\mea{G} \le (1+\epsilon) \mea{E}$.
  By Theorem 1.11, we can write $G = \bigcup I_k$
  where $\{I_k\}$ are nonoverlapping intervals
  and each $I_k$ has equal sides.
  Set $E_k = E \cap I_k$.
  Then $\mea{G} = \sum \mea{I_k}$ and $\mea{E} = \sum \mea{E_k}$,
  and there must exist at least one $\calI \in \{I_k\}$ and one $\calE \in \{E_k\}$
  such that $\mea{\calI} \le (1+\epsilon) \mea{\calE}$.
  Suppose $\calI$ has side length $l > 0$.
  We claim that $\calE \bigcap \calE_{\bfh} \neq \emptyset$
  for all sufficiently small $|\bfh|$.

  If $\calE \bigcap \calE_{\bfh} = \emptyset$ for some $\bfh$
  with $d = |\bfh| > 0$ and $d \le \sqrt{n} l$,
  then we must have $\mea{\calI \cup \calI_{\bfh}} \ge 2 \mea{\calE}$
  since the left-hand-side contains both $\calE$ and $\calE_{\bfh}$.
  Observe that %the volume of the union of the intervals
  \[
    \mea{\calI \cup \calI_{\bfh}} \le l^n + l^n - \left( l - \frac{d}{\sqrt{n}} \right)^n
    = \left[ 2 - \left( 1 - \frac{d}{\sqrt{n} l} \right)^n \right] l^n,
  \]
  where the maximum is attained when
  $\bfh = \left(d / \sqrt{n}, \cdots, d / \sqrt{n}\right)$.
  Combining the estimate for $\mea{\calE}$, we obtain
  \[
    \frac{2}{1 + \epsilon} \le  2 - \left( 1 - \gamma \right)^n,
  \]
  where $\gamma = d / \sqrt{n} l \in (0, 1]$.
  However, this inequality must be violated
  by sufficiently small $\epsilon$ and $\gamma$.
  Therefore, there exists a sufficiently small $d > 0$
  such that $\calE \cap \calE_{\bfh} \neq \emptyset$
  for all $|\bfh| < d$.
  This completes the proof of the Lemma.
\end{proof}

\begin{exercise}{3.19}
  In $\bbR^n$ we define an equivalence relation by saying
  $x \sim y$ if $x - y \in \bbQ^n$;
  that is, $x \sim y$ if every component of $x-y$ is rational.
  Using Axiom of Choice,
  we construct a set $E$ that consists of exactly one element
  from each equivalence class.
  If the set of differences of $E$ contains an interval centered at the origin,
  it must contain a nonzero number in $\bbQ^n$,
  which is impossible by the construction of $E$.
  By the preceding Lemma,
  either $E$ is nonmeasurable or $E$ has measure zero.

  Denote by $E_q$ the translate of $E$ by a rational number $q$.
  Then $E_q \cap E_r \neq \emptyset$ if $q$ and $r$ are distinct rational numbers.
  Since $\bbR^n$ is the countable disjoint union of all $E_q$
  where $q$ ranges over $\bbQ^n$,
  $\bbR^n$ would have measure zero if $E$ did.
  Thus $E$ is nonmeasurable.
\end{exercise}



  % Homework part
  %=====================================================================
%  
\begin{exercise}{1.15}
  By Exercise 1.1 (r),
  we show that $\inf_{\Gamma} U_\Gamma = \sup_{\Gamma} L_\Gamma = A$
  if and only if for each $\epsilon > 0$,
  there exists a partition $\Gamma$ such that $0 \le U_\Gamma - L_\Gamma < \epsilon$.

  ($\Longleftarrow$) We have
  \[
    0 \le \inf U_\Gamma - \sup L_\Gamma \le U_\Gamma - L_\Gamma < \epsilon
  \]
  for each $\epsilon > 0$.
  It follows immediately that $\inf U_\Gamma = \sup L_\Gamma$.

  ($\Longrightarrow$) For each $\epsilon > 0$,
  select a partition $\Gamma_1$ such that $U_{\Gamma_1} < A + \epsilon/2$
  and also a partition $\Gamma_2$ such that $L_{\Gamma_2} > A - \epsilon/2$.
  Let $\Gamma = \Gamma_1 \cup \Gamma_2$ be the common refinement of $\Gamma_1$ and $\Gamma_2$.
  It follows that
  \[
    0 \le U_{\Gamma} - L_{\Gamma} \le U_{\Gamma_1} - L_{\Gamma_2}
    < \left( A + \frac{\epsilon}{2} \right) - \left( A - \frac{\epsilon}{2} \right) = \epsilon.
    \qedhere
  \]
\end{exercise}

\begin{exercise}{1.20}
  For the first part, if $E_q = \left\{ q \right\}^C$, then
  \[
    \bbR - \bbQ = \bigcap_{q \in \bbQ} E_q
  \]
  is a $G_\delta$ set since each $E_q$ is open (and also dense).
  For the second part,
  suppose $\bbQ = \bigcap_n G_n$ is a $G_\delta$ set.
  Each $G_n$ is dense in $\bbR$ since $\bbQ \in G_n$ and $\bbQ$ is.
  It follows that
  \[
    \emptyset = \left( \bbR - \bbQ \right) \cap \bbQ
    = \left( \bigcap_{q \in \bbQ} E_q \right) \bigcap \left( \bigcap_n G_n \right)
  \]
  is a countable intersection of open dense subsets of $\bbR$,
  which is dense again by Exercise 1.19.
  But an empty set cannot be dense in $\bbR$.
\end{exercise}

\begin{exercise}{2.5}
  First observe that $f(x) \le |f(b)| + M$ for every $x \in (a, b]$.
  It follows that $L = \max \left\{ |f(b)| + M, f(a) \right\}$ is a bound of $f$.
  Suppose $\Gamma = \left\{ x_j \right\}_0^k$ is a partition of $[a, b]$.
  Then we have
  \[
    S_{\Gamma} = \left| f(x_1) - f(a) \right| +
    \sum_{j = 1}^{k-1} \left| f(x_{j+1}) - f(x_j) \right|
    \le 2L + V[f; x_1,b]
    \le 2L + M.
  \]
  Taking supremum we see that $V[f;a,b] \le 2L + M < +\infty$
  and $f$ is of bounded variation.

%  However, it does not necessarily follow that $V[f; a,b] \le M$.
  Assume further that $f$ is right-hand continuous at $a$.
  For each $\epsilon > 0$, there exists a $\xi$ such that
  $a < \xi < x_1$ and $|f(\xi) - f(a)| < \epsilon$.
  If $\Gamma'$ is the partition formed by inserting $\xi$ into $\Gamma$,
  we have
  \[
    S_\Gamma \le S_{\Gamma'} \le \left| f(\xi) - f(a) \right| + V[f; \xi, b]
    < \epsilon + M.
  \]
  Since $\epsilon$ is arbitrary, we have $S_{\Gamma} \le M$ for every $\Gamma$.
  Taking supremum, we obtain $V[f;a,b] \le M$.
\end{exercise}

\begin{exercise}{2.15}
  Let $\Gamma = \{x_j\}_0^k$ be a partition of $[a,b]$.
  Then
  \[
    S_\Gamma = \sum \left\vert \psi(x_j) - \psi(x_{j-1}) \right\vert
    = \sum \left\vert \int_{x_{j-1}}^{x_j} f \rmd \phi \right\vert
    \le \left(\sup |f|\right) \sum V[\phi; x_{j-1}, x_j]
    \le \left(\sup |f|\right) V[\phi; a,b],
  \]
  where the first inequality follows from Theorem 2.24.
  This holds for all $\Gamma$ and therefore $\psi$ is of bounded variation.
  If $g$ is continuous on $[a,b]$,
  both $\int_a^b g \rmd \psi$ and $\int_a^b fg \rmd \phi$ exist
  by Theorem 2.24.
  Consider the Riemann-Stieltjes sum for $g \rmd \psi$, we have
  \begin{align*}
    & \sum g(\xi_j) \left( \psi(x_j) - \psi(x_{j-1}) \right) \\
    = & \sum g(\xi_j) f(\eta_j) \left( \phi(x_j) - \phi(x_{j-1}) \right) \\
    = & \sum g(\xi_j) f(\xi_j) \left( \phi(x_j) - \phi(x_{j-1}) \right)
    + \sum g(\xi_j) \left( f(\eta_j) - f(\xi_j) \right)
    \left( \phi(x_j) - \phi(x_{j-1}) \right) \\
    = & I_1 + I_2,
  \end{align*}
  where the first equality follows from Theorem 2.27 (Mean-Value Theorem)
  with each $\eta_j \in [x_{j-1}, x_j]$.
  The term $I_1$ is the Riemann-Stieltjes sum for $gf \rmd \phi$
  and tends to $\int_a^b fg \rmd \phi$ as $|\Gamma| \rightarrow 0$.
  Using the uniform continuity of $f$ on $[a,b]$,
  we can select $|\Gamma|$ so small that
  \[
    \left\vert f(\eta_j) - f(\xi_j) \right\vert <
    \frac{\epsilon} {\left( \sup g \right) V[\phi; a,b]}
  \]
  for every $0 \le j < k$.
  It follows that $|I_2| < \epsilon$
  and consequently $\lim_{|\Gamma| \rightarrow 0} I_2 = 0$.
  The proof of $\int_a^b g \rmd \psi = \int_a^b fg \rmd \phi$ is thus completed.
\end{exercise}

\begin{exercise}{2.20}
  Clearly $S \coloneqq \lim_{|\Gamma| \rightarrow 0} S_\Gamma[f;a,b] \le V[f;a,b]$
  if the limit exists.
  To show the reverse inequality,
  first select a partition $\Gamma'$ such that
  $S_{\Gamma'} > V[f;a,b] - \epsilon/2$.
  Then select a $\delta > 0$ such that
  $|S_\Gamma - S| < \epsilon/2$ whenever $|\Gamma| < \delta$.
  Let $\Gamma^\ast$ be the refinement of $\Gamma'$
  subject to $|\Gamma^\ast| < \delta$.
  Then we have $S_{\Gamma'} \le S_{\Gamma^\ast}$, and
  \[
    V[f;a,b] - S < V - S_{\Gamma^\ast} + \epsilon/2
    \le V - S_{\Gamma'} + \epsilon/2
    < \epsilon/2 + \epsilon/2 = \epsilon.
  \]
  Since $\epsilon > 0$ is arbitrary,
  we have $V[f;a,b] \le S$.
\end{exercise}

\begin{exercise}{3.1}
  Suppose $b > 1$ is an integer and $0 \le x < 1$.
  We determine the integral sequence $c_k$ as follows.
  Let $c_1 = \lfloor bx \rfloor$, i.e.\
  the greatest integer less than or equal to $bx$.
  Then $0 \le c_1 < b$ and $0 \le bx - c_1 < 1$.
  Having determined $c_1, \cdots, c_{k-1}$,
  we determine $c_k$ by choosing
  \[
    c_k = \lfloor b^k x - \sum_{j=1}^{k-1} c_j b^{k-j} \rfloor.
  \]
  Since by the induction hypothesis $0 \le b^{k-1} x - \sum_1^{k-1} c_j b^{k-1-j} < 1$,
  we have $0 \le c_k < b$ and also
  \begin{equation}
    0 \le b^k x - \sum_1^{k} c_j b^{k-j} < 1.
    \label{eq:base_b_sandwich}
  \end{equation}
  This completes the induction step and
  hence the determination of the sequence $c_k$.
  Dividing by $b^{-k}$ in~\eqref{eq:base_b_sandwich}
  and letting $k \rightarrow \infty$,
  we obtain $x = \sum_1^{\infty} c_j b^{-j}$.
  In the exceptional case of $x = 1$,
  we shall choose $c_j = b-1$ for all $j \ge 1$.

  Suppose $\{c_j\}$ and $\{d_j\}$ are integral sequence
  such that $0 \le c_j, d_j < b$ and
  $\sum_{1}^{\infty} (c_j - d_j) b^{-j} = 0$.
  Let $k$ be the smallest index such that $c_k \neq d_k$.
  If no such $k$ exists, then $\{c_j\}$ are $\{d_j\}$ are identical sequences.
  Otherwise, we have
  \[
    b^{-k} \le \left| \left( c_k - d_k \right) b^{-k} \right|
    = \left| \sum_{j=k+1}^{\infty} \left( d_j - c_j \right) b^{-j} \right|
    \le \sum_{j=k+1}^{\infty} (b-1) b^{-j}
    = b^{-k}
  \]
  since $c_k \neq d_k$ and $|d_j - c_j| \le b-1$ for all $j \ge k+1$.
  It follows that the inequalities are in fact equalities,
  and this is the case if and only if
  $c_k - d_k = 1$ and $d_j - c_j = b-1 \Leftrightarrow d_j = b-1, c_j = 0$
  for all $j \ge k+1$ (or vice versa).
  It also follows that
  \[
    x = \sum_{j=1}^k c_j b^{-j} = c b^{-k},
  \]
  where $1 \le c = \sum_1^k c_j b^{k-j} \le b^k-1$.
\end{exercise}

\begin{exercise}{3.9}
  For each $\epsilon > 0$,
  select $n \in \bbZ^+$ so large that $\sum_n^{\infty} \outmea{E_k} < \epsilon$.
  Then $\limsup E_k \subset \cup_{k=n}^{\infty} E_k$,
  and by countable subadditivity we have
  \[
    \outmea{\limsup E_k} \le \outmea{ \cup_{k=n}^{\infty} E_k}
    \le \sum_{k=n}^{\infty} \outmea{E_k} < \epsilon.
  \]
  Since $\epsilon$ is arbitrary,
  we see that $\limsup E_k$ has measure zero,
  and so does $\liminf E_k$ since it is a subset of the former.
\end{exercise}

%  \setcounter{section}{3}
%  \section{Lebesgue Measurable Functions}

\begin{exercise}{4.3}
  If $F$ is measurable as defined, then the sets
  \begin{align*}
    &F^{-1}((a,+\infty)\times\mathbb{R})=\{f>a\}
  \end{align*}
  and
  \begin{align*}
    &F^{-1}(\mathbb{R}\times(b,+\infty))=\{g>b\}
  \end{align*}
  are measurable for any $a,b\in\mathbb{R}.$
  Therefore $f$ and $g$ are measurable.
  Conversely, let $G$ be any open set in $\mathbb{R}^2$,
  then there exist partly open intervals $I_i$ and $J_i$ such that
  \[G=\bigcup_{i=1}^\infty I_i\times J_i.\]
  Then \[F^{-1}(G)=\bigcup_{i=1}^\infty F^{-1}(I_i\times J_i)=\bigcup_{i=1}^\infty f^{-1}(I_i)\cap g^{-1}(J_i).\]
  Since $f$ and $g$ are measurable,
  we have $f^{-1}(I_i)$ and $g^{-1}(J_i)$ are measurable sets.
  Thus $F^{-1}(G)$ is a measurable set and $F$ is measurable.
\end{exercise}

\begin{exercise}{4.5}
  We consider the equivalence relation on $[0,1]$
  defined by $x\sim y\Leftrightarrow x-y\in\mathbb{Q}$.
  Let $E$ be a set consisting exactly one element from each equivalence class
  and in particular let $0\in E$.
  Thus no nonzero rational numbers belong to $E$ and we also know that $E$ is nonmeasurable.
  Let $F$ be the Cantor-Lebesgue function and $Z=F^{-1}(E)$.
  By definition of $F$, elements outside the Cantor set are mapped to nonzero rational values by $F$,
  so we must have $Z\subset C$, the Cantor set.
  Thus $|Z|\leq |C|=0.$
  Define $\phi=\chi_Z$, the characteristic function of $Z$.
  Since $Z$ is a measurable set,
  $\phi$ is a measurable function.
  Now we define \[f(y)=\inf\{x\in[0:1]:x\in F^{-1}(y)\}. \]
  Then since $F$ is continuous and monotone,
  we have $F(f(y))=y$ and $\{f>a\}$ is an interval.
  We deduce that $f$ is also measurable.
  Now for any $y\in E$, we have \[y\in E\Rightarrow f(y)\in F^{-1}(y)\subset F^{-1}(E)=Z.\]
  On the other hand if $f(y)\in Z$, we have $y=F(f(y))\in F(Z)=F(F^{-1}(E))=E$.
  The last equality comes from the fact that $F$ is surjective from $[0,1]$ to $[0,1]$.
  Therefore we have \[f(y)\in Z\Leftrightarrow y\in E.\]
  Now \[(\phi\circ f)^{-1}(\{1\})=\{y:f(y)\in Z\}=E.\]
  Therefore $\phi\circ f$ is not measurable.

  Now if $g(x)=x+F(x)$, we notice that $[0,1]\setminus C$ is a disjoint union of open intervals $\{I_i\}$
  and $F$ is constant on each of the intervals.
  Since $g$ is strictly monotone, it follows that $g(I_i)$ are still disjoint intervals with the same measures as $I_i$.
  Thus \[|g([0,1]\setminus C)|=\sum_i|g(I_i)|=\sum_i|I_i|=|[0,1]\setminus C|=1.\]
  Therefore since $g$ is strictly monotone,
  we have \[|g(C)|=|[0,2]-g([0,1]\setminus C)|=2-1=1.\]
  By Corollary 3.39 there exists a nonmeasurable set $E$ in $g(C)$.
  Let $Z=g^{-1}(E)$.
  Then $Z$ is a subset of $C$ and is a measurable set of measure zero.
  Define $\phi=\chi_Z$ and $f=g^{-1}$, then $\phi$ is measurable and $f$ is continuous.
  However, we have
  \begin{align*}
  (\phi\circ f)^{-1}(\{1\})=\{y\in[0,2]:f(y)\in Z\}=\{y:y\in g(Z)\}=g(Z)=E.
  \end{align*}
  Therefore $\phi\circ f$ is not measurable.
\end{exercise}

\noindent\textbf{Exercise 4.12. }

\begin{proof} (Constructive proof) Suppose $f$ is continuous at almost every point of an interval $[a,b]$ and so $f$ is finite a.e.\ on $[a,b]$.
Let $\{\Gamma_k\}$ be a sequence of partitions of $[a,b]$ with norms $|\Gamma_k|$ tending to zero.
For each $k$, define a simple measurable function $l_k$ as follows:
if $x_1^{(k)}<x_2^{(k)}<\cdots$ are the partitioning points of $\Gamma_k$,
let $l_k(x)$ be defined in each semiopen interval $[x_i^{(k)},x_{i+1}^{(k)})$ as
$\max\{\inf_{[x_i^{(k)},x_{i+1}^{(k)}]}f,-k\}$ which is finite.
Suppose $f$ is continuous at $x\in[a,b)$, then for a given $\epsilon>0$,
there is a neighborhood $[a_x,b_x]$ of $x$ such that $|f(x)-f(y)|\leq \epsilon$
whenever $y\in [a_x,b_x]$.
Then we can choose a $K_1$ sufficiently large such that $f(y)> -K_1$ for each $y\in [a_x,b_x]$.
Since $|\Gamma_k|\to 0$, there is a $K_2>K_1$ such that for any $k\geq K_2$,
$x\in [x_{i_k}^{(k)},x_{{i_k}+1}^{(k)})\subset [a_x,b_x]$ for some $i_k$.
Then $|f(x)-l_k(x)|=|f(x)-\inf_{[x_{i_k}^{(k)},x_{i_{k}+1}^{(k)}]}f|\leq\epsilon$.
This proves that $l_k$ converges a.e.\ to $f$ and
thus that $f$ is measurable on $[a,b]$ by Theorem 4.12.
\end{proof}

Now we generalize this to functions defined in $\mathbf{R^n}$ by using a non-constructive way.

\begin{proof}(Non-constructive proof)
  Suppose $f$ is continuous a.e.\ on a measurable set $E\subset \mathbf{R^n}$,
  then we can write $E$ as a disjoint union $E=E_1\cup E_2$,
  where $f$ is continuous on $E_1$ and $|E_2|=0$.
  Note that the fact $f$ is continuous a.e.\ on $E$ implies
  $\{\bfx\in E:f(x)=+\infty\}$ is a set of measure zero and hence is measurable.
  To prove $f$ is measurable on $E$, by Corollary 4.2,
  we only need to show $\{\bfx\in E: a\leq f(\bfx)<+\infty\}$ is measurable for every finite $a$.
  Note that \[\{\bfx\in E: a\leq f(\bfx)<+\infty\}=\{\bfx\in E_1: a\leq f(\bfx)<+\infty\}\cup\{\bfx\in E_2: a\leq f(\bfx)<+\infty\}\eqqcolon A\cup B,\]
  where $B$ is a set of measure zero and hence is measurable.
  Let $\bfx_0$ be a limit point of $A$ that lies in $E_1$.
  Then there exist $\bfx_k\in E_1$ such that $\bfx_k\to \bfx_0$ and $a\leq f(\bfx_k)<+\infty$.
  Since $f$ is continuous at $\bfx_0\in E_1$,
  we have $a\leq \lim_{k\to\infty}f(\bfx_k)=f(\bfx_0)<+\infty$
  and then $\bfx_0\in A$.
  By Theorem 1.8, it follows that $A$ is relatively closed with respect to $E_1$ and then $A=E_1\cap F$ for some closed $F$.
  Hence $A$ is measurable and so is $\{\bfx\in E: a\leq f(\bfx)<+\infty\}$.
\end{proof}

\begin{exercise}{4.15}
  Define
  \[
    E_n = \left\{ x \in E : \lvert f_k(x) \rvert \le n
    \text{ for all $k$} \right\}
    = \bigcap_{k=1}^{\infty}
    \left\{ x \in E : \lvert f_k(x) \rvert \le n \right\}.
  \]
  Note that $E_n$ is measurable
  since $f_k$ is measurable
  and so is $\left\{ x \in E : \lvert f_k(x) \rvert \le n \right\}$.
  For each $x \in E$,
  we have $x \in E_n$ for all $n \ge M_x$,
  so $E = \bigcup E_n$.
  Since $E_n$ is increasing,
  we have $\lim \mea{E_n} = \mea{E} < +\infty$.
  Therefore, we can select $M \in \bbZ^+$ so large that
  $\mea{E - E_M} = \mea{E} - \mea{E_M} < \epsilon/2$,
  and in turn select a closed set $F \subset E_M$ such that
  $\mea{E_M - F} < \epsilon/2$.
  It follows that $\mea{E - F} = \mea{E - E_M} + \mea{E_M - F} < \epsilon$,
  and for each $x \in F$ we have
  $\lvert f_k(x) \rvert \le M$ for all $k$.
\end{exercise}

\begin{exercise}{4.16}
  Recall that $f_k \xrightarrow{m} f$ if for every $\epsilon > 0$,
  \begin{equation}
    \lim_{k \rightarrow \infty} \mea{\left\{ \lvert f - f_k \rvert > \epsilon \right\}} = 0.
    \label{eq:defOfCvgInMeasure}
  \end{equation}
  Assume $f_k \xrightarrow{m} f$.
  Given $\epsilon > 0$,
  there exists, according to~\eqref{eq:defOfCvgInMeasure}, a $K$ such that
  $\mea{\left\{ \lvert f - f_k \rvert > \epsilon \right\}} < \epsilon$
  for all $k > K$.

  Conversely, assume for each $\epsilon > 0$ there exists a $K$ such that
  $\mea{\left\{ \lvert f - f_k \rvert > \epsilon \right\}} < \epsilon$
  for all $k > K$.
  Then for each $\epsilon > 0$ and each $\epsilon_1 > 0$,
  we can take $\delta = \min \left\{ \epsilon, \epsilon_1 \right\} > 0$,
  and by assumption there exists a $K$ such that
  $\mea{\left\{ \lvert f - f_k \rvert > \delta \right\}} < \delta$
  for all $k > K$.
  Then
  \[
    \mea{\left\{ \lvert f - f_k \rvert > \epsilon \right\}}
    \le \mea{\left\{ \lvert f - f_k \rvert > \delta \right\}}
    < \delta \le \epsilon_1
  \]
  for all $k > K$, which shows exactly $f_k \xrightarrow{m} f$.

  An analogous Cauchy criterion is that
  $\{f_k\}$ is Cauchy in measure if and only if
  for each $\epsilon > 0$,
  there exists a $K$ such that
  \[
    \mea{\left\{ \lvert f_j - f_k \rvert > \epsilon \right\}} < \epsilon
  \]
  for all $j, k > K$.
\end{exercise}

\begin{lemma}
  If $g$ is finite a.e.\ on $E$ and $|E|<+\infty$,
  then for each $\epsilon > 0$
  there exist a subset $F$ of $E$ and a sufficiently large $M$ such that
  $\mea{E - F} < \epsilon$ and $\lvert g \rvert \le M$ on $F$.
\end{lemma}
\begin{proof}
  This is a special case of Exercise 4.15.
\end{proof}

\begin{exercise}{4.17}
  Suppose $f_k \xrightarrow{m} f$ and $g_k \xrightarrow{m} g$ on $E$.
  First we show $f_k + g_k \xrightarrow{m} f+g$.
  Given $\epsilon > 0$,
  we can select $K$ so large that
  \[
    \mea{\left\{ \lvert f - f_k \rvert > \frac{\epsilon}{2} \right\}} < \frac{\epsilon}{2},
    \quad
    \mea{\left\{ \lvert g - g_k \rvert > \frac{\epsilon}{2} \right\}} < \frac{\epsilon}{2},
  \]
  for all $k \ge K$.
  Since $\lvert (f_k + g_k) - (f+g) \rvert \le
  \lvert f_k - f \rvert + \lvert g_k - g \rvert$,
  we have
  \[
    \mea{\left\{ \lvert (f_k + g_k) - (f+g) \rvert > \epsilon \right\}} \le
    \mea{\left\{ \lvert f_k - f \rvert > \frac{\epsilon}{2} \right\}}
    + \mea{\left\{ \lvert g_k - g \rvert > \frac{\epsilon}{2} \right\}}
    < \frac{\epsilon}{2} + \frac{\epsilon}{2} = \epsilon
  \]
  for all $k \ge K$, which shows that $f_k + g_k \xrightarrow{m} f + g$.

  Suppose further that $\mea{E} < +\infty$.
  For each $\epsilon > 0$,
  we first invoke the preceding Lemma to select $M$ such that
  \[
    \mea{\left\{ \lvert f \rvert > M \right\}} < \epsilon/6, \quad
    \mea{\left\{ \lvert g \rvert > M \right\}} < \epsilon/6.
  \]
  Let $\delta = \min\left\{ \epsilon/3, \epsilon/3M, 1 \right\}$.
  By convergence in measure we can select $K$ so large that
  \[
    \mea{\left\{ \lvert f_k - f \rvert > \delta \right\}} < \epsilon/6, \quad
    \mea{\left\{ \lvert g_k - g \rvert > \delta \right\}} < \epsilon/6,
  \]
  for all $k \ge K$.
  Then we have
  \[
    \mea{\left\{ \lvert f_k - f)(g_k - g) \rvert > \epsilon/3 \right\} }
    \le \mea{\left\{ \lvert f_k - f \rvert > \epsilon/3 \right\}}
    + \mea{\left\{ \lvert g_k - g \rvert > 1 \right\}}
    < \epsilon/6 + \epsilon/6 = \epsilon/3.
  \]
  Moreover, we have
  \[
    \mea{\left\{ \lvert f (g_k - g) \rvert > \epsilon/3 \right\}}
    \le \mea{\left\{ \lvert f \rvert > M \right\}}
    + \mea{\left\{ \lvert g_k - g \rvert > \frac{\epsilon}{3M} \right\}}
    < \epsilon/6 + \epsilon/6 = \epsilon/3,
  \]
  and the same argument suggests
  \[
    \mea{\left\{ \lvert g (f_k - f) \rvert > \epsilon/3 \right\}} < \epsilon/3.
  \]
  Writing $f_k g_k - fg = (f_k - f)(g_k - g) + f(g_k - g) + g(f_k - f)$,
  we obtain
  \[
    \begin{aligned}
      \mea{\left\{ \lvert f_k g_k - fg \rvert \right\} > \epsilon}
      \le &\mea{\left\{ \lvert f_k - f)(g_k - g) \rvert > \epsilon/3 \right\} } \\
      + &\mea{\left\{ \lvert f (g_k - g) \rvert > \epsilon/3 \right\}} \\
      + &\mea{\left\{ \lvert g (f_k - f) \rvert > \epsilon/3 \right\}} \\
      < &\epsilon
    \end{aligned}
  \]
  for all $k \ge K$, which shows that $f_k g_k \xrightarrow{m} fg$.

  Finally suppose $g_k \rightarrow g$ on $E$ and $g \neq 0$ a.e.
  Since $\mea{E} < +\infty$, $1/g_k \rightarrow 1/g$ and $1/g$ is finite a.e.,
  Theorem 4.21 suggests that $1/g_k \xrightarrow{m} 1/g$.
  It follows immediately from the last case that $f_k/g_k \xrightarrow{m} f/g$.
\end{exercise}

\begin{exercise}{4.18}
  For the first part, fix $a \in (-\infty, +\infty)$.
  For each $x \in E$,
  it follows from $f_k \nearrow f$ that
  $f(x) > a$ if and only if $f_k(x) > a$ for some $k$.
  That is, $\left\{ f > a \right\} = \bigcup_{k=1}^{\infty} \left\{ f_k > a \right\}$.
  Since $\left\{ f_k > a \right\}$ is increasing,
  we see that $\omega_{f_k}(a) = \mea{\left\{ f_k > a \right\}} \nearrow
  \mea{\left\{ f > a \right\}} = \omega_f(a)$ as $k \rightarrow \infty$.

  Now suppose $f_k \xrightarrow{m} f$.
  For each $\epsilon > 0$, we have
  \[
    \omega_f(a + \epsilon) = \mea{\left\{ f > a + \epsilon \right\}} \le
    \mea{\left\{ f_k > a \right\}} + \mea{\left\{ \lvert f_k - f \rvert > \epsilon \right\}}
    = \omega_{f_k}(a) + \mea{\left\{ \lvert f_k - f \rvert > \epsilon \right\}}.
  \]
  But $f_k \xrightarrow{m} f$ implies that
  $\mea{\left\{ \lvert f_k - f \rvert > \epsilon \right\}} \rightarrow 0$
  as $k \rightarrow \infty$;
  taking limit inferior of the last inequality we obtain
  \[
    \omega_f(a + \epsilon) \le \liminf_{k \rightarrow \infty} \omega_{f_k}(a).
  \]
  Since $\epsilon > 0$ is arbitrary,
  by letting $\epsilon \rightarrow 0^+$ we see that
  $\omega_f(a) \le \liminf \omega_{f_k}(a)$
  provided $\omega_f$ is continuous at $a$.
  A similar argument shows that $\limsup \omega_{f_k}(a) \le \omega_f(a)$.
  We conclude that $\lim \omega_{f_k}(a) = \omega_f(a)$
  provided $\omega_f$ is continuous at $a$.
\end{exercise}

%  \section{The Lebesgue Integral}

\begin{exercise}{5.4}
  For each $k=1,2,\cdots$, the continuous function $x^k$ is measurable,
  so is $x^kf(x)$ (Theorem 4.10).
  Since $[x^kf(x)]^\pm=x^kf^\pm(x)\leq f^\pm(x)$ on $(0,1)$,
  the existence of the integral $\int_0^1x^kf(x)\rmd x$ follows from Theorem 5.5 (i).
  Note that
  \[\left|\int_0^1 x^kf(x)\rmd x\right|\leq\int_0^1\left|x^kf(x)\right|\rmd x\leq\int_0^1|f(x)|\rmd x<\infty\]
  as $f\in L(0,1)$.
  Hence $x^kf(x)\in L(0,1)$ for $k=1,2,\cdots$.

  Since $f\in L(0,1)$, it follows that $f$ is finite a.e.\ in $(0,1)$ (Theorem 5.22)
  and then $x^kf(x)\to 0$ a.e.\ in $(0,1)$.
  Moreover, $|x^kf(x)|\leq |f(x)|$ in $(0,1)$ for all $k$.
  By Lebesgue's dominated convergence theorem (Theorem 5.36),
  we have $\int_0^1x^kf(x)\rmd x\to0$.
\end{exercise}

\begin{exercise}{5.5}
  Let $\{f_k\}$ be a sequence of measurable functions on $E$ such that
  $f_k\to f$ a.e.\ in $E$ (so $f$ is measurable by Theorem 4.12).
  Suppose $|E|<+\infty$ and there is a finite constant $M$
  such that $|f_k|\leq M$ a.e.\ in $E$ (so $f_k\in L(E)$).
  We want to prove $\int_Ef_k\to \int_E f$ by using Egorov's theorem.

  We note that $|f|\leq M$ a.e.\ on $E$ and hence it is also finite a.e.
  Since $|E|<+\infty$, it follows that $f\in L(E)$.
  Given $\epsilon>0$, by Egorov's theorem,
  there is a closed subset $F$ of $E$ such that
  $|E-F|<\epsilon/(4M)$ and $\{f_k\}$ converges uniformly to $f$ on $F$.
  If $|E|=0$, then the result follows from Theorem 5.25.
  If $|E|\neq 0$, then by the uniform convergence of $\{f_k\}$ in $F$,
  we have $|f_k(x)-f(x)|\leq \epsilon/(2|E|)$ for all $x\in F$ if $k$ is sufficiently large.
  From Theorem 5.28 and (5.20), we obtain
  \begin{equation*}
    \begin{aligned}
      \left|\int_E f_k-\int_Ef\right|&=	\left|\int_E (f_k-f)\right|\leq \int_E|f_k-f|\\&=\int_{E-F}|f_k-f|+\int_F|f_k-f|\\&\leq\int_{E-F}(|f_k|+|f|)+\int_F\frac{\epsilon}{2|E|}\\
      &\leq 2M|E-F|+\frac{\epsilon}{2|E|}|F|\\&\leq\frac{\epsilon}{2}+\frac{\epsilon}{2}=\epsilon
    \end{aligned}
  \end{equation*}
  provided $k$ is sufficiently large.
  Since $\epsilon>0$ is arbitrary, we have $\int_Ef_k\to \int_E f$.
\end{exercise}

\begin{exercise}{5.6}
  Given $x\in [0,1]$, let $\{x_n\}$ be any sequence in $[0,1]$ converging to $x$ and $x_n\neq x$.
  We define
  \[h_n(y)=\frac{f(x_n,y)-f(x,y)}{x_n-x},\quad n=1,2,\cdots.\]
  Since for each $x$, $f(x,y)$ is a measurable function of $y$,
  so are $h_n(y)$ by Theorem 4.9 and Theorem 4.10.
  Note that
  \[\frac{\partial}{\partial x}f(x,y)=\lim_{n\to\infty}h_n(y)\] exists
  and then the measurability of $(\partial f(x,y)/\partial x)$ w.r.t $y$ follows from Theorem 4.12.

  Since $(\partial f(x,y)/\partial x)$ is a bounded function of $(x,y)$,
  the mean value theorem implies that
  \[|h_n(y)|\leq \sup_{x\in[0,1]}\left|\frac{\partial}{\partial x}f(x,y)\right|\leq M, \quad\forall y\in[0,1],\]
  for some constant $M$.
  So the bounded convergence theorem (Corollary 5.37) can be invoked to give
  \[\frac{\rmd}{\rmd x}\int_0^1f(x,y)\rmd y=\lim_{n\to\infty}\int_0^1h_n(y)\rmd y=\int_0^1\lim_{n\to\infty}h_n(y)\rmd y=\int_0^1\frac{\partial}{\partial x}f(x,y)\rmd y.
  \qedhere\]
\end{exercise}

\begin{exercise}{5.9}
  Given $\epsilon>0$, we have
  \[|\left\{\bfx\in E:|f(\bfx)-f_k(\bfx)|>\epsilon\right\}|\leq\frac{1}{\epsilon^p}\int_{\{\bfx\in E:|f(\bfx)-f_k(\bfx)|>\epsilon\}}|f-f_k|^p\leq\frac{1}{\epsilon^p}\int_E|f-f_k|^p \to 0 \]
  as $k\to\infty$.
  This proves that $f_k\overset{m}{\longrightarrow} f$ on $E$ and thus
  that there is a subsequence $f_{k_j}\to f$ a.e.\ in $E$ by Theorem 4.22.
\end{exercise}

\begin{exercise}{5.10}
  Suppose $p > 0$ and $\int_E \lvert f_k - f \rvert^p \rightarrow 0$.
  By Exercise 5.9, we know that $f_k \xrightarrow{m} f$ on $E$.
%  First we claim that $f_k \xrightarrow{m} f$ on $E$ (Exercise 5.9).
%  Given $\epsilon > 0$, Chebyshev's inequality states that
%  \[
%    \mea{\left\{ \lvert f - f_k \rvert > \epsilon \right\}}
%    \le \frac{1}{\epsilon^p} \int_E \lvert f_k - f \rvert^p.
%  \]
%  Letting $k \rightarrow \infty$
%  we see that $\mea{\left\{ \lvert f - f_k \rvert > \epsilon \right\}} \rightarrow 0$
%  as $k \rightarrow \infty$,
%  which is exactly $f_k \xrightarrow{m} f$.
  By Theorem 4.22, there exists a subsequence $\{f_{k_j}\} \rightarrow f$ a.e.\ in $E$.
  If $\int_E \lvert f_k \rvert^p \le M$ for all $k$,
  we conclude by Fatou's Lemma that
  \[
    \int_E \lvert f \rvert^p = \int_E \lim \lvert f_{k_j} \rvert^p
    \le \liminf \int_E \lvert f_{k_j} \rvert^p \le M.
    \qedhere
  \]
\end{exercise}

\begin{exercise}{5.18}  The result of Exercise 5.16 suggests that if $0 < p < \infty$ and $f \ge 0$, then
  \[
    \int_E f^p = p \int_0^{\infty} \alpha^{p-1} \omega(\alpha) \rmd \alpha,
  \]
  regardless of the finiteness of either $\mea{E}$ or $\lVert f \rVert_p$.
  By definition of improper Riemann integral, we have
  \[
    \int_0^\infty \alpha^{p-1} \omega(\alpha) \rmd \alpha =
    \sum_{k = -\infty}^{+\infty}
    \int_{2^k}^{2^{k+1}} \alpha^{p-1} \omega(\alpha) \rmd \alpha.
%    \eqqcolon \sum_{k = -\infty}^{+\infty} I_k.
  \]
  Since $\omega$ is decreasing and $\alpha^{p-1} > 0$,
  we have
  \[
    I_k \coloneqq \int_{2^k}^{2^{k+1}} \alpha^{p-1} \omega(\alpha) \rmd \alpha
    \le \omega(2^k) \int_{2^k}^{2^{k+1}} \alpha^{p-1} \rmd \alpha
    = \frac{2^p-1}{p} 2^{kp} \omega(2^k).
  \]
  On the other hand,
  \[
    I_k \ge \omega(2^{k+1}) \int_{2^k}^{2^{k+1}} \alpha^{p-1} \rmd \alpha
    = \frac{2^p-1}{p 2^p} 2^{(k+1)p} \omega(2^{k+1}).
  \]
  Combining the above inequalities, we find
  \[
    \frac{2^p-1}{p 2^p}
    \sum 2^{(k+1)p} \omega(2^{k+1})
    \le \sum I_k
    \le \frac{2^p-1}{p} \sum 2^{kp} \omega(2^k).
  \]
  Note that the summation terms on the left-hand-side
  can be changed to $2^{kp} \omega(2^k)$.
  Summarizing, there exist constants $c_1, c_2 > 0$ such that
  $c_1 \sum 2^{kp} \omega(2^k) \le \int f^p \le c_2 \sum 2^{kp} \omega(2^k)$.
%  and consequently all the terms in the last inequalities
%  are simultaneously finite or infinite.
  We conclude that $f \in L^p$ if and only if $\sum 2^{kp} \omega(2^k) < + \infty$.
\end{exercise}


\begin{exercise}{5.21}
  If $f$ is not zero almost everywhere.
  Then at least one of the sets $\{f>0\}$ and $\{f<0\}$ has strictly positive measure.
  Suppose we have $|\{f>0\}|>0$.
  Since \[\{f>\frac{1}{n}\}\nearrow\{f>0\},\]
  we have $\lim_{n\rightarrow\infty}|\{f>1/n\}|=|\{f>0\}|$.
  Therefore for some $n_0$ sufficiently large,
  $A=\{f>1/n_0\}$ has strictly positive measure.
  In this case we can deduce that\[\int_A f\geq\frac{|A|}{n_0}>0, \] contradiction.
\end{exercise}

\begin{exercise}{5.22}
  We have a sequence of measurable functions $\{|f_k-f|\}$ on $E$
  such that $|f_k-f|\rightarrow 0$ a.e.\ in $E$.
  Now suppose on $E\setminus Z_k$, we have $|f_k|\leq \phi$ where $|Z_k|=0$.
  Let \[Z=\bigcup_k Z_k\cup\{x\in E:f_k(x)\not\rightarrow f(x)\},\]
  then $|Z|=0$ and on $E\setminus Z$,
  we have $|f|\leq \phi$.
  Therefore we deduce that $|f_k-f|\leq 2\phi$ a.e.\ in $E$.
  Since $2\phi\in L(E)$, from Lebesgue's Dominated Convergence Theorem,
  we have $\int_E|f_k-f|\rightarrow 0.$
\end{exercise}

\begin{exercise}{5.23}
  Let $g_k=|f_k-f|$, then $g_k\rightarrow 0$ a.e.\ in $E$
  and $|g_k|\leq \phi_k+\phi$ a.e.\ in $E$
  and $\int_E(\phi_k+\phi)\rightarrow 2\int_E\phi.$
  Therefore $\phi_k+\phi-g_k$ is nonnegative a.e.\ in $E$.
  By Fatou's Lemma, we have
  \[2\int_E\phi=\int_E\varliminf(\phi_k+\phi-g_k)\leq\varliminf\int_E\phi_k+\phi-g_k=2\int_E\phi-\varlimsup\int_Eg_k.\]
  Therefore \[0\leq\varliminf\int_E g_k\leq\varlimsup\int_Eg_k\leq 0.\]
  Thus $\lim\int_Eg_k=0.$
\end{exercise}

\begin{exercise}{5.24}
(a) Since $f\in L^p(E)$,
we have \[\alpha^p\omega_{|f|}(\alpha)\leq\int_{\{|f|>\alpha\}}|f|^p\leq\int_E|f|^p<\infty.\]
Thus \[\omega_{|f|}(\alpha)\leq\frac{\int_E|f|^p}{\alpha^p},\]
which implies $f$ belongs to weak $L^p(E)$.
Now let $E=(0,+\infty)$ and $f(x)=1/x$ defined on $E$,
then $\int_E f=\infty$.
But for any $\alpha>0$, we have $\omega_{|f|}(\alpha)=1/\alpha$.
Thus $f$ belongs to weak $L^1(E)$ but not $L^1(E)$.

(b) By definition we have nonnegative constants $A$ and $A'$ such that
$\omega_{|f|}(\alpha)\leq A/\alpha$ and $\omega_{|f|}(\alpha)\leq A'/\alpha^r$.
Thus $\omega_{|f|}$ is finite on $(0,\infty)$ and by results in Exercise 5.16 we have
\begin{align*}
  \int_E|f|^p&=p\int_0^\infty\alpha^{p-1}\omega_{|f|}(\alpha)d\alpha\\
  &=p\int_0^1\alpha^{p-1}\omega_{|f|}(\alpha)d\alpha+p\int_1^\infty\alpha^{p-1}\omega_{|f|}(\alpha)d\alpha\\
  &\leq p\int_0^1\frac{A}{\alpha^{2-p}}d\alpha+p\int_1^\infty \frac{A'}{\alpha^{1+r-p}}d\alpha.
\end{align*}
Now since $2-p<1$ and $1+r-p>1$, the right hand side integrals are finite,
which implies $f$ belongs to $L^p(E)$.

(c) Suppose $|f| \le M$ for some $M>0$.
Thus $\omega_{|f|}(\alpha)=0$ when $\alpha\geq M$.
We still have $\omega_{|f|}$ is finite on $(0,\infty)$ and therefore
\[\int_E|f|^p=p\int_0^M\alpha^{p-1}\omega_{|f|}(\alpha)d\alpha\leq p\int_0^M\frac{A}{\alpha^{2-p}}d\alpha<\infty\]
since $2-p<1.$
\end{exercise}

  \setcounter{section}{5}
  \section{Repeated Integration}
\begin{exercise}{6.3}
Let $F(x,y)=f(x)-f(y)$ be integrable over the square $[0,1]\times [0,1]$. By Fubini's theorem, it follows that for almost every $y\in [0,1]$, $F(x,y)$ is integrable on $[0,1]$ as a function of $x$. Since $f$ is finite a.e. on $[0,1]$, there exists $y_0\in[0,1]$ such that both $F(x,y_0)$ and $f(y_0)$ (which is a finite constant) are integrable with respect to $x$ on $[0,1]$. Thus the sum $f(x)=F(x,y_0)+f(y_0)$ is integrable on $[0,1]$ as a function of $x$.
\end{exercise}
\begin{exercise}{6.4}
Let $F(x,t)=|f(x+t)-f(-x+t)|$, then $\int_0^1F(x,t)\rmd t$ is measurable as a function of $x$ by Tonelli's Theorem; and $\int_0^1\int^1_0F(x,t)\rmd t\rmd x\leq c$. Making the change of variables $\xi=x+t, \eta=-x+t$, we have
\begin{equation*}
	\begin{aligned}
	\int_0^1\int^1_0F(x,t)\rmd t\rmd x&=\frac{1}{2}\int_0^1\int_{-\xi}^\xi|f(\xi)-f(\eta)|\rmd \eta\rmd \xi+\frac{1}{2}\int_1^2\int_{\xi-2}^{-\xi+2}|f(\xi)-f(\eta)|\rmd \eta \rmd \xi\\
	&\geq\frac{1}{2}\int_0^1\int_{0}^\xi|f(\xi)-f(\eta)|\rmd \eta \rmd \xi + \frac{1}{2}\int_1^2\int_{\xi-2}^{0}|f(\xi)-f(\eta)|\rmd \eta \rmd \xi\\
&	=\frac{1}{2}\int_0^1\int_{0}^\xi|f(\xi)-f(\eta)|\rmd \eta \rmd \xi + \frac{1}{2}\int_0^1\int_{\xi}^{1}|f(\xi+1)-f(\eta-1)|\rmd \eta \rmd \xi\\
	&=\frac{1}{2}\int_0^1\int_0^1|f(\xi)-f(\eta)|\rmd\eta\rmd\xi,
	\end{aligned}
\end{equation*}
where the last equality is provided by the periodicity of $f$. Thus $f(\xi)-f(\eta)$ is integrable over the square $[0,1]\times [0,1]$ and then $f\in L[0,1]$ by Exercise 6.3.
\end{exercise}
\begin{exercise}{6.5}
  (a) By Tonelli's theorem and the definition of Lebesgue integral, we have
  \[
    \int_Ef=\mea{R(f,E)}=\iint_{R(f,E)}\rmd\bfx \rmd y=\int_0^\infty\bigg[\int_{\mathbb{R}^n}\chi_{R(f,E)}\rmd\bfx\bigg]\rmd y.
  \]
  Now
  \[
    \int_{\mathbb{R}^n}\chi_{R(f,E)}\rmd\bfx=\mea{\{\bfx\in E:f(\bfx)\geq y\}}
  \] 
  and when $\omega$ is continuous at $y$, we have $\omega(y)=\mea{\{\bfx\in E:f(\bfx)\geq y\}}$. Since $\omega$ is monotone, it has countable number of points of discontinuity. We deduce that $\omega(y)=\mea{\{\bfx\in E:f(\bfx)\geq y\}}$ almost everywhere on $(0,\infty)$. Thus
  \[
    \int_Ef=\int_0^\infty\omega(y)\rmd y.
  \]
  (b) By (a) we have \[\int_Ef^p=\int_0^\infty\omega_{f^p}(y)\rmd y.\]Now $\omega_{f^p}(y)=\mea{\{\bfx\in E:f^p(\bfx)>y\}}=\mea{\{\bfx\in E:f(\bfx)>\sqrt[p]{y}\}}=\omega_f(\sqrt[p]{y})$ for $y\geq 0$. Since $\omega$ is improper Riemann integrable, we can use change of variables in improper Riemann integral to get
  \[
    \int_Ef^p=\int_0^\infty\omega_{f^p}(y)\rmd y=\int_0^\infty\omega_f(\sqrt[p]{y})\rmd y=p\int_0^\infty\omega(t)t^{p-1}\rmd t
    \qedhere
  \]
\end{exercise}
\begin{exercise}{6.6}
	Suppose $f$ and $g$ belong to $L(\bfr)$, then $f\ast g\in L(\bfr)$. Thus the Fourier transform $\widehat{f\ast g}$ is well defined and the application of Fubini’s theorem is justified; and then
	\begin{equation*}
		\begin{aligned}
		\widehat{(f\ast g)}(x)&=\frac{1}{2\pi}\int_\bfr (\widehat{f\ast g})(t)e^{-ixt}\rmd t\\
		&=\frac{1}{2\pi}\int_\bfr \left[\int_{\bfr}f(t-y)g(y)\rmd y\right]e^{-ixt}\rmd t\\
		&=\frac{1}{2\pi}\int_\bfr g(y)e^{-ixy}\left[\int_{\bfr}f(t-y)e^{-ix(t-y)}\rmd t\right]\rmd y\\
		&=2\pi \widehat{f}(x)\widehat{g}(x).
		\end{aligned}
	\end{equation*}
\end{exercise}

\begin{exercise}{6.10}
  Denote by $B^n_r$ the ball centered at the origin
  with radius $r$ in $\bbR^{n}$.
  First we claim that $\vol(B^n_r) = r^n v_n$.
  Indeed, the linear transformation $T : \bfx \mapsto r\bfx$
  maps the unit ball to $B^n_r$ bijectively.
  Since $|\det{T}| = r^n$, we have
  \[
    \vol(B^n_r) = \int_{B^n_r} \rmd \bfx
    = \int_{B^n_1} |\det{T}| \ \rmd \bfx
    = r^n v_n.
  \]

  Now we prove the formula.
  First we have $B^1_1 = [-1, 1]$ and $v_1 = 2$.
  For $n \ge 2$, we have
  \[
    \begin{aligned}
      v_n &= \int_{B^n_1} \rmd \bfx \\
      &= \int_{-1}^1 \left(\idotsint_{x_2^2 + \cdots + x_n^2 \le 1 - x_1^2}
      \rmd x_2 \cdots \rmd x_n\right) \rmd x_1 \quad
      &&(\text{Tonelli's Theorem}) \\
      &= \int_{-1}^1
      \left( \idotsint_{B^{n-1}_{(1-x_1^2)^{1/2}}} \rmd x_2 \cdots \rmd x_n \right)
      \rmd x_1 \\
      &= \int_{-1}^1 \left( 1 - x_1^2 \right)^{(n-1)/2} v_{n-1} \rmd x_1 \\
      &= 2 v_{n-1} \int_0^1 (1-t^2)^{(n-1)/2} \rmd t.
      \quad &&(t \mapsto \left( 1-t^2 \right)^{(n-1)/2} \text{ is even})
    \end{aligned}
  \]
  By setting $w = t^2$, we find $\rmd t = \frac{1}{2} w^{-1/2} \rmd w$ and
  \[
    \int_0^1 \left( 1-t^2 \right)^{(n-1)/2} \rmd t
    = \int_0^1 \frac{1}{2} \left( 1-w \right)^{(n-1)/2} w^{-1/2} \rmd w
    = \frac{1}{2} B(\frac{n+1}{2}, \frac{1}{2})
    = \frac{1}{2} \frac {\Gamma(\frac{n+1}{2}) \Gamma(\frac{1}{2})} {\Gamma(\frac{n}{2}+1)}.
    \qedhere
  \]
\end{exercise}


\begin{exercise}{6.11}
  When $n=1$, we have
  \[
    \bigg(\int_{\mathbb{R}}e^{-x^2}\rmd x\bigg)^2=\iint_{\mathbb{R}^2}e^{-(x^2+y^2)}\rmd x\rmd y=\int_0^{2\pi}\int_0^\infty e^{-r^2}r\rmd r\rmd\theta=\pi.
  \]
  Thus $\int_{\mathbb{R}}e^{-x^2}\rmd x=\pi^{1/2}$. Now for $n>1$, since $e^{-|\bfx|^2}$ is nonnegative and measurable on $\bbR^n$, we can use Tonelli's theorem for $n-1$ times and get
  \[
    \int_{\bbR^n}e^{-|\bfx|^2}\rmd\bfx=\int_{\bbR^n}e^{-x_1^2}e^{-x_2^2}\cdots e^{-x_n^2}\rmd x_1\rmd x_2\cdots \rmd x_n=\bigg(\int_{\mathbb{R}}e^{-x^2}\rmd x\bigg)^n=\pi^{n/2}.
    \qedhere
  \]
\end{exercise}
\begin{exercise}{6.13}
Let $F(x,y)=\frac{1}{2h}f(y)\chi_{[x-h,x+h]}(y)$ for fixed $h>0$, and note that $\chi_{[x-h,x+h]}(y)=\chi_{[y-h,y+h]}(x)$ for all $x,y\in\bfr$, then $\int_\bfr|F(x,y)|\rmd x=|f(y)|\in L(\rmd y)$. Thus $F(x,y)\in L(\rmd x\rmd y)$ and Fubini's Theorem gives 
\begin{equation*}
	\begin{aligned}
	\int_{-\infty}^{\infty}\left(\frac{1}{2h}\int_{x-h}^{x+h}f(y)\rmd y\right)\rmd x&=	\int_{-\infty}^{\infty}\left(	\int_{-\infty}^{\infty} F(x,y)\rmd y\right)\rmd x\\&=\int_{-\infty}^{\infty}\left(	\int_{-\infty}^{\infty} F(x,y)\rmd x\right)\rmd y\\
	&=	\int_{-\infty}^{\infty}\frac{1}{2h}f(y)\left(	\int_{-\infty}^{\infty} \chi_{[y-h,y+h]}(x)\rmd x\right)\rmd y\\
	&=\int_{-\infty}^{\infty}f(x)\rmd x.
	\end{aligned}
\end{equation*}
\end{exercise}


\end{document}