
\begin{exercise}{3.11}
  Suppose $E$ is measurable and $\mea{E} < + \infty$.
  Select an open set $G$ such that $G \supset E$ and $\mea{G-E} < \epsilon$.
  By Theorem 1.11, there exists a countable collection $\{I_k\}$
  of nonoverlapping intervals such that $G = \bigcup I_k$
  and therefore $\mea{G} = \sum \mea{I_k}$.
  Choose $N$ so large that $\sum_{N+1}^{\infty} \mea{I_k} < \epsilon$.
  Now take $S = \bigcup_1^N I_k, N_1 = \bigcup_{N+1}^\infty I_k$
  and $N_2 = G - E$.
  Then $E = (S \cup N_1) - N_2$, $\mea{N_1} < \epsilon$ and $\mea{N_2} < \epsilon$
  as desired.

  Conversely, suppose $E = (S \cup N_1) - N_2$,
  $\outmea{N_1} < \epsilon$ and $\outmea{N_2} < \epsilon$.
  We may assume $N_2 \subset (S \cup N_1)$ by taking appropriate intersection.
  Select an open set $G$ with $G \supset S$ and $\mea{G - S} < \epsilon$.
  Then select an open set $G_1$ with $G_1 \supset N_1$
  and $\mea{G_1} < \outmea{N_1} + \epsilon < 2 \epsilon$.
  By subadditivity of outer measure, we have
  \[
    \begin{aligned}
      \outmea{(G \cup G_1) - E} &= \outmea{(G \cup G_1) - (S \cup N_1) \cup N_2} \\
      &\le \outmea{(G \cup G_1) - (S \cup N_1)} + \outmea{N_2} \\
      &< \mea{G-S} + \outmea{G_1 - N_1} + \epsilon \\
      &< \epsilon + \outmea{G_1} + \epsilon \\
      &< 4\epsilon.
    \end{aligned}
  \]
  We have shown that $E$ can be approximated by open sets from outside.
  Thus $E$ is measurable.
\end{exercise}

\begin{exercise}{3.12}
  First let us fix $E_1$ to be a finite open interval.

  \begin{enumerate}[noitemsep]
  \item
  Since the measure of an interval is equal to its volume,
  the assertion $\mea{E_1 \times E_2} = \mea{E_1} \mea{E_2}$ holds
  for $E_2$ being an open interval.

  \item
  Suppose $E_2$ is an open set.
  By Theorem 1.10,
  we can write $E_2 = \bigcup I_k$
  where $\{I_k\}$ is a collection of disjoint open intervals.
  Then by what has been proved,
  \[
    E_1 \times E_2 = E_1 \times \left( \bigcup I_k \right)
    = \bigcup \left(E_1 \times I_k\right)
  \]
  is a countable union of disjoint measurable sets in $\bbR^2$
  and hence measurable.
  Moreover,
  \[
    \mea{E_1 \times E_2} = \sum \mea{E_1 \times I_k}
    = \mea{E_1} \sum \mea{I_k} = \mea{E_1} \mea{E_2}.
  \]

  \item
  If $\mea{E_2} = 0$, then we can find an open set $G \supset E_2$
  with $\mea{G} < \epsilon$. Then
  $\outmea{E_1 \times E_2} \le \mea{E_1 \times G} < \epsilon \mea{E_1}$
  for arbitrary $\epsilon > 0$.
  Hence $\mea{E_1 \times E_2} = 0$.

  \item
  Suppose $E_2 = \bigcap G_k$ is a $G_\delta$ set with finite measure.
  We may assume $\mea{G_1} < \infty$ and $\{G_k\}$ is decreasing.
  Then by what has been proved,
  \[
    E_1 \times E_2 = E_1 \times \left( \bigcap G_k \right)
    = \bigcap \left( E_1 \times G_k \right)
  \]
  is a countable decreasing intersection of measurable sets
  and hence measurable.
  Moreover,
  \[
    \mea{E_1 \times E_2}
    = \lim \mea{E_1 \times G_k}
    = \mea{E_1} \lim \mea{G_k} = \mea{E_1} \mea{E_2}.
  \]

  \item
  Suppose $E_2$ has finite measure.
  Write $E_2 = G - Z$ where $G$ is a $G_{\delta}$ set with finite measure
  and $Z$ has measure zero. Then by what has been proved,
  \[
    E_1 \times E_2 = E_1 \times \left( G-Z \right)
    = E_1 \times G - E_1 \times Z
  \]
  is the difference of two measurable sets and hence measurable.
  It also follows easily that
  $\mea{E_1 \times E_2} = \mea{E_1}\mea{G} - \mea{E_1}\mea{Z} = \mea{E_1}\mea{E_2}$.

  \item
  Suppose $E_2$ has infinite measure.
  We can write $E_2 = \bigcup E_k$
  where the collection $\{E_k\}$ is increasing and each $\mea{E_k} < \infty$.
  Then $E_1 \times E_2$ is a countable increasing union of measurable sets
  and hence measurable.
  Moreover, $\mea{E_1 \times E_2} = \mea{E_1} \lim {E_k} = \infty$.
  \end{enumerate}

  We have proved that the assertion $\mea{E_1 \times E_2} = \mea{E_1} \mea{E_2}$
  holds for $E_1$ being a finite open interval and $E_2$ an arbitrary measurable set.
  By interchanging the roles of $E_1$ and $E_2$
  and going through the above steps again,
  we can show that the assertion holds for arbitrary measurable $E_1$ and $E_2$.
\end{exercise}

\begin{exercise}{3.13}
  First recall that
  \[
    \outmea{E} = \inf \left\{ \mea{G} : G \text{ is open and } G \subset E\right\}.
  \]
  Since $\mea{F} \le \mea{G}$ for all closed $F \subset E$
  and all open $G \supset E$,
  taking infimum and supremum we obtain
  $\inmea{E} \le \outmea{E}$.
  This proves part (i).

  Suppose $E$ has finite measure.
  Lemma 3.22 states that for an arbitrary $\epsilon > 0$
  there exists a closed set $F \subset E$ with
  $\mea{E} - \mea{F} = \mea{E-F} < \epsilon$.
  It follows easily that $\inmea{E} = \mea{E} = \outmea{E}$.
  Conversely, suppose $\inmea{E} = \outmea{E} < \infty$.
  For each $k \in \bbZ^+$,
  select an open set $G_k$ and a closed set $F_k$ such that
  \[
    F_k \subset E \subset G_k, \quad \mea{G_k} - \mea{F_k} < 1/k.
  \]
  Let $G = \bigcap G_k$.
  Then $G$ is a $G_\delta$ (and hence measurable) set.
  Moreover, it follows from $G - E \subset G_k - F_k$ that
  \[
    \outmea{G-E} \le \mea{G_k - F_k} = \mea{G_k} - \mea{F_k} < 1/k
  \]
  for every $k \in \bbZ^+$.
  Hence $G-E$ has measure zero.
  Finally, being the difference of two measurable sets,
  $E = G - (G-E)$ is measurable.
  This completes the proof of part (ii).
\end{exercise}

\begin{exercise}{3.14}
  Let $E_1$ be a nonmeasurable subset of $[-2, -1]$.
  Let $E_2 = [0, +\infty)$ and $E = E_1 \cup E_2$.
  Clearly $E$ is nonmeasurable.
  Moreover, we have
  \[
    \begin{aligned}
      \outmea{E} \ge \outmea{E_2} = \infty \quad &\Rightarrow \quad \outmea{E} = \infty, \\
      \inmea{E} \ge \mea{[0, n]} = n \text{ for every } n \in \bbZ^+
      \quad &\Rightarrow \quad \inmea{E} = \infty.
    \end{aligned}
  \]
  Hence part (ii) of Exercise 13 is false if $\outmea{E} = \infty$.
\end{exercise}

\begin{exercise}{3.16}
  Recall that a parallelepiped is
  \[
    P = \left\{ \sum_1^n t_k \bfe_k : 0 \le t_k \le 1 \right\}.
  \]
  It is easy to show that $P$ is a compact, and hence measurable set.

  First we show $v(P) \le \mea{P}$.
  Suppose $\{I_k\}$ is a collection of intervals that covers $P$.
  We can enlarge each $I_k$ to some open interval $J_k$ such that
  $I_k$ is contained in $J_k^\circ$ and $v(J_k) < (1+\epsilon) v(I_k)$.
  By the Heine-Borel property,
  there exists a finite cover $\{J_k\}_1^N$ of $P$.
  It follows that
  \[
    v(P) \le \sum_1^N v(J_k) \le (1+\epsilon) \sum_1^N v(I_k)
    \le (1 + \epsilon) \sum_1^\infty v(I_k).
  \]
  Since this holds for all covers $\{I_k\}$ of $P$,
  taking infimum of the right-hand-side we obtain
  $v(P) \le (1 + \epsilon) \mea{P}$.
  But $\epsilon$ is arbitrary,
  so $v(P) \le \mea{P}$.

  To show $\mea{P} \le v(P)$,
  we first claim that for each $\epsilon > 0$
  there exists an open set $G \supset P$
  such that $\mea{G} \le v(P) + \epsilon$.
  If $\{\bfe_k\}$ is linearly independent,
  then $v(P) > 0$ and we can take
  \[
    G = \left\{ \sum_1^n t_k \bfe_k : -\delta < t_k < 1 + \delta \right\}.
  \]
  Then by p.\ 8 in Section 1.3 we know
  %the volume formula for a parallelepiped states that
  $v(G) = (1 + 2\delta)^n v(P) < v(P) + \epsilon$
  for sufficiently small $\delta$.
  Since $G$ is open,
  by Theorem 1.11 $G$ can be written as a countable union
  of the nonoverlapping intervals $\{I_j\}$.
  Then we have $\sum_1^N v(I_j) \le v(G)$ for every $N$;
  taking limit we obtain
  \[
    \mea{G} = \sum v(I_j) \le v(G) \le v(P) + \epsilon.
  \]

  If $\{\bfe_k\}$ is not linearly independent,
  let $U$ be the proper subspace of $\bbR^n$ spanned by $\{\bfe_k\}$
  and let $V$ be the orthogonal complement of $U$.
  %with orthonormal basis. $\{\bff_k\}_1^m$.
  Since $P$ is a compact subset of $U$,
  we can find in the subspace $U$
  a bounded open rotated interval $B_1$ that contains $P$.
  If $B_2$ is an open rotated interval centered at the origin
  in the subspace of $V$,
  then $P$ is contained in the open rotated interval $G = B_1 \times B_2$
  and $\mea{G} < v(P) + \epsilon = \epsilon$ for sufficiently small $B_2$.
  This proves the claim.

  Finally, it follows from the claim that for each $\epsilon > 0$
  we have $\mea{P} \le \mea{G} \le v(P) + \epsilon$ for suitably chosen $G \supset P$.
  Thus $\mea{P} \le v(P)$ and the proof of $\mea{P} = v(P)$ is completed.
\end{exercise}

\begin{exercise}{3.17}
  Let $C$ be the Cantor set and $G = [0, 1] - C$.
  Recall that the Cantor-Lebesgue function $f$ is continuous
  and maps $[0, 1]$ onto $[0, 1]$.
%  By definition we see that
  Observe that
  \[
    f(G) = \left\{ \frac{k}{2^n} : n \in \bbZ^+, 1 \le k \le 2^n-1 \right\}
  \]
  is a countable subset of $[0, 1]$ and therefore has measure zero.
  Let $E$ be a nonmeasurable subset of $[0, 1] - f(G)$.
  Since $E \cap f(G) = \emptyset$,
  we must have $f^{-1}(E) \subset C$.
  Since $\mea{C} = 0$,
  we see that $f^{-1}(E)$ has measure zero.
  But by construction $E = f (f^{-1}(E))$ is nonmeasurable.
\end{exercise}

\begin{exercise}{3.18}
  Suppose $E \subset \bbR^n$ and $\bfh \in \bbR^n$.
%  We show that $\outmea{E_{\bfh}} \le \outmea{E}$.
  Suppose $\{I_k\}$ is a cover of $E$ by intervals and
  $\sum v(I_k) < \outmea{E} + \epsilon$.
  Then $\{I_{k, \bfh}\}$ is a cover of $E_{\bfh}$ and
  \[
    \outmea{E_{\bfh}} \le \sum v(I_{k, \bfh}) = \sum v(I_k) < \outmea{E} + \epsilon,
  \]
  since the volume of an interval is translation-invariant.
  But $\epsilon$ is arbitrary, so $\outmea{E_{\bfh}} \le \outmea{E}$.
  By symmetry we have $\outmea{E} \le \outmea{E_{\bfh}}$.

  If $E$ is measurable, for each $\epsilon > 0$
  there exists an open set $G$ such that $G \supset E$ and $\mea{G-E} < \epsilon$.
  But $G_{\bfh}$ is open,
  $G_{\bfh} \supset E_{\bfh}$
  and $\outmea{G_{\bfh} - E_{\bfh}} = \mea{G - E} < \epsilon$
  by what has been proved.
  Therefore $E_{\bfh}$ is measurable.
\end{exercise}

To complete Exercise 19,
we extend Lemma 3.37 to multidimensions.

\begin{lemma}
  Let $E \subset \bbR^n$ be a measurable set with $\mea{E} > 0$.
  Then the set of differences
  $\{d : d = x-y, x \in E, y \in E\}$
  contains an open ball centered at the origin.
\end{lemma}
\begin{proof}
  Assume first that $E$ is compact with $\mea{E} > 0$.
  Then there exists a open set $G$ such that
  $G \supset E$ and $\mea{G} < 2 \mea{E}$.
  Since $G^C$ is closed, $E$ is compact and $G^C \cap E = \emptyset$,
  we must have $\dist(G^C, E) = \delta > 0$.
  It follows that if $d \in \bbR^n$ and $|x| < \delta/2$,
  then the translate $E_{d}$ of $E$ by $\delta$ is disjoint from $G^C$;
  equivalently, $E_{d} \subset G$ for all $|d| < \delta/2$.
  For such $d$, we claim that $E_{d} \cap E \neq \emptyset$.
  Otherwise, $G$ would contain $E_{d} \cup E$ and
  \[
    2 \mea{E} = \mea{E_d} + \mea{E} = \mea{E_d \cup E} \le \mea{G},
  \]
  a contradiction.
  We have shown that the set of differences
  contains an open ball of radius $\delta/2$
  in the case of compact $E$.

  Suppose $E$ has positive (possibly infinite) measure.
  Since
  \[
    E = \bigcup_{n \in \bbZ^+} \left( E \cap B(0, n) \right), \quad
    \mea{E \cap B(0,n)} \rightarrow \mea{E},
  \]
  there exists a bounded subset $E'$ of $E$ with $\mea{E'} > 0$.
  There exists, by Lemma 3.22,
  a closed (and hence compact) subset $F$ of $E'$ with $\mea{F} > 0$.
  By the preceding case, the set of differences of $F$
  contains an open ball centered at the origin.
  The proof is completed by noting that
  the set of differences of $F$ is contained in that of $E$.
\end{proof}

\begin{exercise}{3.19}
  In $\bbR^n$ we define an equivalence relation by saying
  $x \sim y$ if $x - y \in \bbQ^n$;
  that is, $x \sim y$ if every component of $x-y$ is rational.
  Using Axiom of Choice,
  we construct a set $E$ that consists of exactly one element
  from each equivalence class.
  If the set of differences of $E$ contains an open ball centered at the origin,
  it must contain a nonzero number of $\bbQ^n$,
  which is impossible by the construction of $E$.
  By the preceding Lemma,
  either $E$ is nonmeasurable or $E$ has measure zero.

  Denote by $E_q$ the translate of $E$ by a rational number $q$.
  Then $E_q \cap E_r \neq \emptyset$ if $q$ and $r$ are distinct rational numbers.
  Since $\bbR^n$ is the countable disjoint union of all $E_q$
  where $q$ ranges over $\bbQ^n$,
  $\bbR^n$ would have measure zero if $E$ did.
  Thus $E$ is nonmeasurable.
\end{exercise}

