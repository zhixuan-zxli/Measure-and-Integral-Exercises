
\begin{exercise}{1.15}
  By Exercise 1.1 (r),
  we show that $\inf_{\Gamma} U_\Gamma = \sup_{\Gamma} L_\Gamma = A$
  if and only if for each $\epsilon > 0$,
  there exists a partition $\Gamma$ such that $0 \le U_\Gamma - L_\Gamma < \epsilon$.

  ($\Longleftarrow$) We have
  \[
    0 \le \inf U_\Gamma - \sup L_\Gamma \le U_\Gamma - L_\Gamma < \epsilon
  \]
  for each $\epsilon > 0$.
  It follows immediately that $\inf U_\Gamma = \sup L_\Gamma$.

  ($\Longrightarrow$) For each $\epsilon > 0$,
  select a partition $\Gamma_1$ such that $U_{\Gamma_1} < A + \frac{\epsilon}{2}$
  and also a partition $\Gamma_2$ such that $L_{\Gamma_2} > A - \frac{\epsilon}{2}$.
  Let $\Gamma = \Gamma_1 \cup \Gamma_2$ be the common refinement of $\Gamma_1$ and $\Gamma_2$.
  It follows that
  \[
    0 \le U_{\Gamma} - L_{\Gamma} \le U_{\Gamma_1} - L_{\Gamma_2}
    < \left( A + \frac{\epsilon}{2} \right) - \left( A - \frac{\epsilon}{2} \right) = \epsilon.
    \qedhere
  \]
\end{exercise}

\begin{exercise}{1.20}
  For the first part, if $E_q = \left\{ q \right\}^C$, then
  \[
    \bbR - \bbQ = \bigcap_{q \in \bbQ} E_q
  \]
  is a $G_\delta$ set since each $E_q$ is open (and also dense).
  For the second part,
  suppose $\bbQ = \bigcap_n G_n$ is a $G_\delta$ set.
  Each $G_n$ is dense in $\bbR$ since $\bbQ \in G_n$ and $\bbQ$ is.
  It follows that
  \[
    \emptyset = \left( \bbR - \bbQ \right) \cap \bbQ
    = \left( \bigcap_{q \in \bbQ} E_q \right) \bigcap \left( \bigcap_n G_n \right)
  \]
  is a countable intersection of open dense subsets of $\bbR$,
  which is dense again by Exercise 19.
  But an empty set cannot be dense in $\bbR$.
\end{exercise}

\begin{exercise}{2.5}
  First observe that if $L_1 = \max \left\{ |f(b) + M|, |f(b) - M| \right\}$,
  then $f(x) \le L_1$ for every $x \in (a, b]$.
  It follows that $L = \max \left\{ L, f(a) \right\} < + \infty$
  is an upper bound of $f$.

  Suppose $\Gamma = \left\{ x_j \right\}_0^k$ is a partition of $[a, b]$.
  Then we have
  \[
    S_{\Gamma} = \left| f(x_1) - f(a) \right| +
    \sum_{j = 1}^{k-1} \left| f(x_{j+1}) - f(x_j) \right|
    \le 2L + M.
  \]
  Taking supremum we see that $V[f;a,b] \le 2L + M < +\infty$
  and $f$ is of bounded variation.

%  However, it does not necessarily follow that $V[f; a,b] \le M$.
  Assume further that $f$ is right-hand continuous at $a$.
  For each $\epsilon > 0$, there exists a $\xi$ such that
  $a < \xi < x_1$ and $|f(\xi) - f(a)| < \epsilon$.
  If $\Gamma'$ is the partition formed by inserting $\xi$ into $\Gamma$,
  we have
  \[
    S_\Gamma \le S_{\Gamma'} \le \left| f(\xi) - f(a) \right| + V[f; \xi, b]
    < \epsilon + M.
  \]
  Since $\epsilon$ is arbitrary, we have $S_{\Gamma} \le M$ for every $\Gamma$.
  Taking supremum, we obtain $V[f;a,b] \le M$.
\end{exercise}

\begin{exercise}{2.15}
  Let $\Gamma = \{x_j\}_0^k$ be a partition of $[a,b]$. Then
  \[
    S_\Gamma = \sum \left\vert \psi(x_j) - \psi(x_{j-1}) \right\vert
    = \sum \left\vert \int_{x_{j-1}}^{x_j} f \rmd \phi \right\vert
    \le \left(\sup |f|\right) \sum V[\phi; x_{j-1}, x_j]
    \le \left(\sup |f|\right) V[\phi; a,b],
  \]
  where the first inequality follows from Theorem 2.24.
  This holds for all $\Gamma$ and thus $\psi$ is of bounded variation.
  If $g$ is continuous on $[a,b]$,
  both $\int_a^b g \rmd \psi$ and $\int_a^b fg \rmd \phi$ exist
  by Theorem 2.24.
  Consider the Riemann-Stieltjes sum for $g \rmd \psi$, we have
  \begin{align*}
    & \sum g(\xi_j) \left( \psi(x_j) - \psi(x_{j-1}) \right) \\
    = & \sum g(\xi_j) f(\eta_j) \left( \phi(x_j) - \phi(x_{j-1}) \right) \\
    = & \sum g(\xi_j) f(\xi_j) \left( \phi(x_j) - \phi(x_{j-1}) \right)
    + \sum g(\xi_j) \left( f(\eta_j) - f(\xi_j) \right)
    \left( \phi(x_j) - \phi(x_{j-1}) \right) \\
    = & I_1 + I_2,
  \end{align*}
  where the first equality follows from Theorem 2.27 (Mean-Value Theorem)
  with each $\eta_j \in [x_{j-1}, x_j]$.
  The term $I_1$ is the Riemann-Stieltjes sum for $gf \rmd \phi$
  and tends to $\int_a^b fg \rmd \phi$ as $|\Gamma| \rightarrow 0$.
  Using the uniform continuity of $f$ on $[a,b]$,
  we can select $|\Gamma|$ so small that
  \[
    \left\vert f(\eta_j) - f(\xi_j) \right\vert <
    \frac{\epsilon} {\left( \sup g \right) V[\phi; a,b]}
  \]
  for every $0 \le j < k$.
  It follows that $|I_2| < \epsilon$
  and consequently $\lim_{|\Gamma| \rightarrow 0} I_2 = 0$.
  The proof of $\int_a^b g \rmd \psi = \int_a^b fg \rmd \phi$ is thus completed.
\end{exercise}

\begin{exercise}{2.20}
  Clearly $S \coloneqq \lim_{|\Gamma| \rightarrow 0} S_\Gamma[f;a,b] \le V[f;a,b]$
  if the limit exists.
  To show the reverse inequality,
  first select a partition $\Gamma'$ such that
  $S_{\Gamma'} > V[f;a,b] - \epsilon/2$.
  Then select a $\delta > 0$ such that
  $|S_\Gamma - S| < \epsilon/2$ whenever $|\Gamma| < \delta$.
  Let $\Gamma^\ast$ be the refinement of $\Gamma'$
  subject to $|\Gamma^\ast| < \delta$.
  Then we have $S_{\Gamma'} \le S_{\Gamma^\ast}$, and
  \[
    V[f;a,b] - S < V - S_{\Gamma^\ast} + \epsilon/2
    \le V - S_{\Gamma'} + \epsilon/2
    < \epsilon/2 + \epsilon/2 = \epsilon.
  \]
  Since $\epsilon > 0$ is arbitrary,
  we have $V[f;a,b] \le S$.
\end{exercise}
