\setcounter{section}{3}
\section{Lebesgue Measurable Functions}

\begin{exercise}{4.3}
  If $F$ is measurable as defined, then the sets
  \begin{align*}
    &F^{-1}((a,+\infty)\times\mathbb{R})=\{f>a\}
  \end{align*}
  and
  \begin{align*}
    &F^{-1}(\mathbb{R}\times(b,+\infty))=\{g>b\}
  \end{align*}
  are measurable for any $a,b\in\mathbb{R}.$
  Therefore $f$ and $g$ are measurable.
  Conversely, let $G$ be any open set in $\mathbb{R}^2$,
  then there exist partly open intervals $I_i$ and $J_i$ such that
  \[G=\bigcup_{i=1}^\infty I_i\times J_i.\]
  Then \[F^{-1}(G)=\bigcup_{i=1}^\infty F^{-1}(I_i\times J_i)=\bigcup_{i=1}^\infty f^{-1}(I_i)\cap g^{-1}(J_i).\]
  Since $f$ and $g$ are measurable,
  we have $f^{-1}(I_i)$ and $g^{-1}(J_i)$ are measurable sets.
  Thus $F^{-1}(G)$ is a measurable set and $F$ is measurable.
\end{exercise}

\begin{exercise}{4.5}
  We consider the equivalence relation on $[0,1]$ defined by $x\sim y\Leftrightarrow x-y\in\mathbb{Q}$.
  Let $E$ be a set consisting exactly one element from each equivalence class and in particular we let $0\in E$.
  Thus no nonzero rational numbers belong to $E$ and we also know that $E$ is nonmeasurable.
  Let $Z=F^{-1}(E)$.
  By definition of $F$, elements outside the Cantor set are mapped to nonzero rational values by $F$,
  so we must have $Z\subset C$, the Cantor set.
  Thus $|Z|\leq |C|=0.$ Define $\phi=\chi_Z$, the characteristic function of $Z$.
  Since $Z$ is a measurable set, $\phi$ is a measurable function.
  Now we define \[f(y)=\inf\{x\in[0:1]:x\in F^{-1}(y)\}.\]
  Then since $F$ is continuous and monotone, we have $F(f(y))=y$ and $\{f>a\}$ is an interval.
  We deduce that $f$ is also measurable.
  Now for any $y\in E$, we have \[y\in E\Rightarrow f(y)\in F^{-1}(y)\subset F^{-1}(E)=Z.\]
  On the other hand if $f(y)\in Z$, we have $y=F(f(y))\in F(Z)=F(F^{-1}(E))=E$.
  The last equality comes from the fact that $F$ is surjective from $[0,1]$ to $[0,1]$.
  Therefore we have \[f(y)\in Z\Leftrightarrow y\in E.\]
  Now \[(\phi\circ f)^{-1}(\{1\})=\{y:f(y)\in Z\}=E.\]
  Therefore $\phi\circ f$ is not measurable.

  Now if $g(x)=x+F(x)$, we notice that $[0,1]\setminus C$ is a disjoint union of open intervals $\{I_i\}$
  and $F$ is constant on each of the intervals.
  It follows that $g(I_i)$ are still disjoint intervals with the same measures.
  Thus \[|g([0,1]\setminus C)|=\sum_i|g(I_i)|=\sum_i|I_i|=|[0,1]\setminus C|=1.\]
  Therefore since $g$ is strictly monotone, we have \[|g(C)|=|[0,2]-g([0,1]\setminus C)|=2-1=1.\]
  By Corollary 3.39 there exists a nonmeasruable set $E$ in $g(C)$.
  Let $Z=g^{-1}(E)$.
  Then $Z$ is a subset of $C$ and is a measurable set of measure zero.
  Define $\phi=\chi_Z$ and $f=g^{-1}$, then $\phi$ is measurable and $f$ is continuous.
  However, we have
  \begin{align*}
  (\phi\circ f)^{-1}(\{1\})=\{y\in[0,2]:f(y)\in Z\}=\{y:y\in g(Z)\}=g(Z)=E.
  \end{align*}
  Therefore $\phi\circ f$ is not measurable.
\end{exercise}

\noindent\textbf{Exercise 4.12. }

\begin{proof} (Constructive proof) Suppose $f$ is continuous at almost every point of an interval $[a,b]$ and so $f$ is finite a.e.\ on $[a,b]$.
Let $\{\Gamma_k\}$ be a sequence of partitions of $[a,b]$ with norms $|\Gamma_k|$ tending to zero.
For each $k$, define a simple measurable function $l_k$ as follows:
if $x_1^{(k)}<x_2^{(k)}<\cdots$ are the partitioning points of $\Gamma_k$,
let $l_k(x)$ be defined in each semiopen interval $[x_i^{(k)},x_{i+1}^{(k)})$ as
$\max\{\inf_{[x_i^{(k)},x_{i+1}^{(k)}]}f,-k\}$ which is finite.
Suppose $f$ is continuous at $x\in[a,b)$, then for a given $\epsilon>0$,
there is a neighborhood $[a_x,b_x]$ of $x$ such that $|f(x)-f(y)|\leq \epsilon$
whenever $y\in [a_x,b_x]$.
Then we can choose a $K_1$ sufficiently large such that $f(y)> -K_1$ for each $y\in [a_x,b_x]$.
Since $|\Gamma_k|\to 0$, there is a $K_2>K_1$ such that for any $k\geq K_2$,
$x\in [x_{i_k}^{(k)},x_{{i_k}+1}^{(k)})\subset [a_x,b_x]$ for some $i_k$.
Then $|f(x)-l_k(x)|=|f(x)-\inf_{[x_{i_k}^{(k)},x_{i_{k}+1}^{(k)}]}f|\leq\epsilon$.
This proves that $l_k$ converges a.e.\ to $f$ and
thus that $f$ is measurable on $[a,b]$ by Theorem 4.12.
\end{proof}

Now we generalize this to functions defined in $\mathbf{R^n}$ by using a non-constructive way.

\begin{proof}(Non-constructive proof)
  Suppose $f$ is continuous a.e.\ on a measurable set $E\subset \mathbf{R^n}$,
  then we can write $E$ as a disjoint union $E=E_1\cup E_2$,
  where $f$ is continuous on $E_1$ and $|E_2|=0$.
  Note that the fact $f$ is continuous a.e.\ on $E$ implies
  $\{\bfx\in E:f(x)=+\infty\}$ is a set of measure zero and hence is measurable.
  To prove $f$ is measurable on $E$, by Corollary 4.2,
  we only need to show $\{\bfx\in E: a\leq f(\bfx)<+\infty\}$ is measurable for every finite $a$.
  Note that \[\{\bfx\in E: a\leq f(\bfx)<+\infty\}=\{\bfx\in E_1: a\leq f(\bfx)<+\infty\}\cup\{\bfx\in E_2: a\leq f(\bfx)<+\infty\}\eqqcolon A\cup B,\]
  where $B$ is a set of measure zero and hence is measurable.
  Let $\bfx_0$ be a limit point of $A$ that lies in $E_1$.
  Then there exist $\bfx_k\in E_1$ such that $\bfx_k\to \bfx_0$ and $a\leq f(\bfx_k)<+\infty$.
  Since $f$ is continuous at $\bfx_0\in E_1$,
  we have $a\leq \lim_{k\to\infty}f(\bfx_k)=f(\bfx_0)<+\infty$
  and then $\bfx_0\in A$.
  By Theorem 1.8, it follows that $A$ is relatively closed with respect to $E_1$ and then $A=E_1\cap F$ for some closed $F$.
  Hence $A$ is measurable and so is $\{\bfx\in E: a\leq f(\bfx)<+\infty\}$.
\end{proof}

\begin{exercise}{4.15}
  Define
  \[
    E_n = \left\{ x \in E : \lvert f_k(x) \rvert \le n
    \text{ for all $k$ and all $x \in E$} \right\}.
  \]
  Then for each $x \in E$,
  we have $x \in E_n$ for all $n \ge M_x$,
  so $E = \bigcup E_n$.
  Since $E_n$ is increasing,
  we have $\lim \mea{E_n} = \mea{E} < +\infty$.
  Therefore, we can select $M \in \bbZ^+$ so large that
  $\mea{E - E_M} = \mea{E} - \mea{E_M} < \epsilon/2$,
  and in turn select a closed set $F \subset E_M$ such that
  $\mea{E_M - F} < \epsilon/2$.
  It follows that $\mea{E - F} \le \mea{E - E_M} + \mea{E_M - F} < \epsilon$,
  and for each $x \in F$ we have
  $\lvert f_k(x) \rvert \le M$ for all $k$.
\end{exercise}

\begin{exercise}{4.16}
  Recall that $f_k \xrightarrow{m} f$ if for every $\epsilon > 0$,
  \begin{equation}
    \lim_{k \rightarrow \infty} \mea{\left\{ \lvert f - f_k \rvert > \epsilon \right\}} = 0.
    \label{eq:defOfCvgInMeasure}
  \end{equation}
  Assume $f_k \xrightarrow{m} f$.
  Given $\epsilon > 0$,
  there exists, according to~\eqref{eq:defOfCvgInMeasure}, a $K$ such that
  $\mea{\left\{ \lvert f - f_k \rvert > \epsilon \right\}} < \epsilon$
  for all $k > K$.

  Conversely, assume for each $\epsilon > 0$ there exists a $K$ such that
  $\mea{\left\{ \lvert f - f_k \rvert > \epsilon \right\}} < \epsilon$
  for all $k > K$.
  Then for each $\epsilon > 0$ and each $\epsilon_1 > 0$,
  we can take $\delta = \min \left\{ \epsilon, \epsilon_1 \right\} > 0$,
  and by assumption there exists a $K$ such that
  $\mea{\left\{ \lvert f - f_k \rvert > \delta \right\}} < \delta$
  for all $k > K$.
  Then
  \[
    \mea{\left\{ \lvert f - f_k \rvert > \epsilon \right\}}
    \le \mea{\left\{ \lvert f - f_k \rvert > \delta \right\}}
    < \delta \le \epsilon_1
  \]
  for all $k > K$, which shows exactly $f_k \xrightarrow{m} f$.

  An analogous Cauchy criterion is that
  $\{f_k\}$ is Cauchy in measure if and only if
  for each $\epsilon > 0$,
  there exists a $K$ such that
  \[
    \mea{\left\{ \lvert f_j - f_k \rvert > \epsilon \right\}} < \epsilon
  \]
  for all $j, k > K$.
\end{exercise}

\begin{lemma}
  If $g$ is finite a.e.,
  then for each $\epsilon > 0$
  there exist a subset $F$ of $E$ and a sufficiently large $M$ such that
  $\mea{E - F} < \epsilon$ and $\lvert g \rvert \le M$ on $F$.
\end{lemma}
\begin{proof}
  This is a special case of Exercise 4.15.
\end{proof}

\begin{exercise}{4.17}
  Suppose $f_k \xrightarrow{m} f$ and $g_k \xrightarrow{m} g$ on $E$.
  First we show $f_k + g_k \xrightarrow{m} f+g$.
  Given $\epsilon > 0$,
  we can select $K$ so large that
  \[
    \mea{\left\{ \lvert f - f_k \rvert > \frac{\epsilon}{2} \right\}} < \frac{\epsilon}{2},
    \quad
    \mea{\left\{ \lvert g - g_k \rvert > \frac{\epsilon}{2} \right\}} < \frac{\epsilon}{2},
  \]
  for all $k \ge K$.
  Since $\lvert (f_k + g_k) - (f+g) \rvert \le
  \lvert f_k - f \rvert + \lvert g_k - g \rvert$,
  we have
  \[
    \mea{\left\{ \lvert (f_k + g_k) - (f+g) \rvert > \epsilon \right\}} \le
    \mea{\left\{ \lvert f_k - f \rvert > \frac{\epsilon}{2} \right\}}
    + \mea{\left\{ \lvert g_k - g \rvert > \frac{\epsilon}{2} \right\}}
    < \frac{\epsilon}{2} + \frac{\epsilon}{2} = \epsilon
  \]
  for all $k \ge K$, which shows that $f_k + g_k \xrightarrow{m} f + g$.

  Suppose further that $\mea{E} < +\infty$.
  For each $\epsilon > 0$,
  we first invoke the preceding Lemma to select $M$ such that
  \[
    \mea{\left\{ \lvert f \rvert > M \right\}} < \epsilon/6, \quad
    \mea{\left\{ \lvert g \rvert > M \right\}} < \epsilon/6.
  \]
  Let $\delta = \min\left\{ \epsilon/3, \epsilon/3M, 1 \right\}$.
  By convergence in measure we can select $K$ so large that
  \[
    \mea{\left\{ \lvert f_k - f \rvert > \delta \right\}} < \epsilon/6, \quad
    \mea{\left\{ \lvert g_k - g \rvert > \delta \right\}} < \epsilon/6,
  \]
  for all $k \ge K$.
  Then we have
  \[
    \mea{\left\{ \lvert f_k - f)(g_k - g) \rvert > \epsilon/3 \right\} }
    \le \mea{\left\{ \lvert f_k - f \rvert > \epsilon/3 \right\}}
    + \mea{\left\{ \lvert g_k - g \rvert > 1 \right\}}
    < \epsilon/6 + \epsilon/6 = \epsilon/3.
  \]
  Moreover, we have
  \[
    \mea{\left\{ \lvert f (g_k - g) \rvert > \epsilon/3 \right\}}
    \le \mea{\left\{ \lvert f \rvert > M \right\}}
    + \mea{\left\{ \lvert g_k - g \rvert > \frac{\epsilon}{3M} \right\}}
    < \epsilon/6 + \epsilon/6 = \epsilon/3,
  \]
  and the same argument suggests
  \[
    \mea{\left\{ \lvert g (f_k - f) \rvert > \epsilon/3 \right\}} < \epsilon/3.
  \]
  Writing $f_k g_k - fg = (f_k - f)(g_k - g) + f(g_k - g) + g(f_k - f)$,
  we obtain
  \[
    \begin{aligned}
      \mea{\left\{ \lvert f_k g_k - fg \rvert \right\} > \epsilon}
      \le &\mea{\left\{ \lvert f_k - f)(g_k - g) \rvert > \epsilon/3 \right\} } \\
      + &\mea{\left\{ \lvert f (g_k - g) \rvert > \epsilon/3 \right\}} \\
      + &\mea{\left\{ \lvert g (f_k - f) \rvert > \epsilon/3 \right\}} \\
      < &\epsilon
    \end{aligned}
  \]
  for all $k \ge K$, which shows that $f_k g_k \xrightarrow{m} fg$.

  Finally suppose $g_k \rightarrow g$ on $E$ and $g \neq 0$ a.e.
  Since $\mea{E} < +\infty$, $1/g_k \rightarrow 1/g$ and $1/g$ is finite a.e.,
  Theorem 4.21 suggests that $1/g_k \xrightarrow{m} 1/g$.
  It follows immediately from the last case that $f_k/g_k \xrightarrow{m} f/g$.
\end{exercise}

\begin{exercise}{4.18}
  For the first part, fix $a \in (-\infty, +\infty)$.
  For each $x \in E$,
  it follows from $f_k \nearrow f$ that
  $f(x) > a$ if and only if $f_k(x) > a$ for all some $k$.
  That is, $\left\{ f > a \right\} = \bigcup_{k=1}^{\infty} \left\{ f_k > a \right\}$.
  Since $\left\{ f_k > a \right\}$ is increasing,
  we see that $\omega_{f_k}(a) = \mea{\left\{ f_k > a \right\}} \nearrow
  \mea{\left\{ f > a \right\}} = \omega_f(a)$ as $k \rightarrow \infty$.

  Now suppose $f_k \xrightarrow{m} f$.
  For each $\epsilon > 0$, we have
  \[
    \omega_f(a + \epsilon) = \mea{\left\{ f > a + \epsilon \right\}} \le
    \mea{\left\{ f_k > a \right\}} + \mea{\left\{ \lvert f_k - f \rvert > \epsilon \right\}}
    = \omega_{f_k}(a) + \mea{\left\{ \lvert f_k - f \rvert > \epsilon \right\}}.
  \]
  But $f_k \xrightarrow{m} f$ implies that
  $\mea{\left\{ \lvert f_k - f \rvert > \epsilon \right\}} \rightarrow 0$
  as $k \rightarrow \infty$;
  taking limit inferior of the last inequality we obtain
  \[
    \omega_f(a + \epsilon) \le \liminf_{k \rightarrow \infty} \omega_{f_k}(a).
  \]
  Since $\epsilon > 0$ is arbitrary,
  by letting $\epsilon \rightarrow 0^+$ we see that
  $\omega_f(a) \le \liminf \omega_{f_k}(a)$
  provided $\omega_f$ is continuous at $a$.
  A similar argument shows that $\limsup \omega_{f_k}(a) \le \omega_f(a)$.
  We conclude that $\lim \omega_{f_k}(a) = \omega_f(a)$
  provided $\omega_f$ is continuous at $a$.
\end{exercise}

\section{The Lebesgue Integral}

\begin{exercise}{5.4}
  For each $k=1,2,\cdots$, the continuous function $x^k$ is measurable,
  so is $x^kf(x)$ (Theorem 4.10).
  Since $[x^kf(x)]^\pm=x^kf^\pm(x)\leq f^\pm(x)$ on $(0,1)$,
  the existence of the integral $\int_0^1x^kf(x)\rmd x$ follows from Theorem 5.5 (i).
  Note that
  \[\left|\int_0^1 x^kf(x)\rmd x\right|\leq\int_0^1\left|x^kf(x)\right|\rmd x\leq\int_0^1|f(x)|\rmd x<\infty\]
  as $f\in L(0,1)$.
  Hence $x^kf(x)\in L(0,1)$ for $k=1,2,\cdots$.

  Since $f\in L(0,1)$, it follows that $f$ is finite a.e.\ in $(0,1)$ (Theorem 5.22)
  and then $x^kf(x)\to 0$ a.e.\ in $(0,1)$.
  Moreover, $|x^kf(x)|\leq |f(x)|$ in $(0,1)$ for all $k$.
  By Lebesgue's dominated convergence theorem (Theorem 5.36),
  we have $\int_0^1x^kf(x)\rmd x\to0$.
\end{exercise}

\begin{exercise}{5.5}
  Let $\{f_k\}$ be a sequence of measurable functions on $E$ such that
  $f_k\to f$ a.e.\ in $E$ (so $f$ is measurable by Theorem 4.12).
  Suppose $|E|<+\infty$ and there is a finite constant $M$
  such that $|f_k|\leq M$ a.e.\ in $E$ (so $f_k\in L(E)$).
  We want to prove $\int_Ef_k\to \int_E f$ by using Egorov's theorem.

  We first prove that $f$ is bounded a.e.\ on $E$ and hence it is also finite a.e.
  For almost every point $x$ of $E$ and every $\delta>0$,
  we have $|f(x)-f_k(x)|\leq\delta$ if $k$ is sufficiently large.
  Then triangle inequality implies that $|f(x)|\leq|f(x)-f_K(x)|+|f_K(x)|\leq \delta+M$.
  Since $\delta>0$ is arbitrary,
  we have $|f(x)|\leq M$ for almost every point $x\in E$.
  Since $|E|<+\infty$, it follows that $f\in L(E)$.

  Given $\epsilon>0$, by Egorov's theorem,
  there is a closed subset $F$ of $E$ such that
  $|E-F|<\epsilon/(4M)$ and $\{f_k\}$ converges uniformly to $f$ on $F$.
  If $|E|=0$, then the result follows from Theorem 5.25.
  If $|E|\neq 0$, then by the uniform convergence of $\{f_k\}$ in $F$,
  we have $|f_k(x)-f(x)|\leq \epsilon/(2|E|)$ for all $x\in F$ if $k$ is sufficiently large.
  From Theorem 5.28 and (5.20), we obtain
  \begin{equation*}
    \begin{aligned}
      \left|\int_E f_k-\int_Ef\right|&=	\left|\int_E (f_k-f)\right|\leq \int_E|f_k-f|\\&=\int_{E-F}|f_k-f|+\int_F|f_k-f|\\&\leq\int_{E-F}(|f_k|+|f|)+\int_F\frac{\epsilon}{2|E|}\\
      &\leq 2M|E-F|+\frac{\epsilon}{2|E|}|F|\\&\leq\frac{\epsilon}{2}+\frac{\epsilon}{2}=\epsilon
    \end{aligned}
  \end{equation*}
  provided $k$ is sufficiently large.
  Since $\epsilon>0$ is arbitrary, we have $\int_Ef_k\to \int_E f$.
\end{exercise}

\begin{exercise}{5.6}
  Given $x\in [0,1]$, let $\{x_n\}$ be any sequence in $[0,1]$ converging to $x$ and $x_n\neq x$.
  We define
  \[h_n(y)=\frac{f(x_n,y)-f(x,y)}{x_n-x},\quad n=1,2,\cdots.\]
  Since for each $x$, $f(x,y)$ is a measurable function of $y$,
  so are $h_n(y)$ by Theorem 4.9 and Theorem 4.10.
  Note that
  \[\frac{\partial}{\partial x}f(x,y)=\lim_{n\to\infty}h_n(y)\] exists
  and then the measurability of $(\partial f(x,y)/\partial x)$ w.r.t $y$ follows from Theorem 4.12.

  Since $(\partial f(x,y)/\partial x)$ is a bounded function of $(x,y)$,
  the mean value theorem implies that
  \[|h_n(y)|\leq \sup_{x\in[0,1]}\left|\frac{\partial}{\partial x}f(x,y)\right|\leq M, \quad\forall y\in[0,1],\]
  for some constant $M$.
  So the bounded convergence theorem (Corollary 5.37) can be invoked to give
  \[\frac{\rmd}{\rmd x}\int_0^1f(x,y)\rmd y=\lim_{n\to\infty}\int_0^1h_n(y)\rmd y=\int_0^1\lim_{n\to\infty}h_n(y)\rmd y=\int_0^1\frac{\partial}{\partial x}f(x,y)\rmd y.\]
\end{exercise}

\begin{exercise}{5.9}
  Given $\epsilon>0$, we have
  \[|\left\{\bfx\in E:|f(\bfx)-f_k(\bfx)|>\epsilon\right\}|\leq\frac{1}{\epsilon^p}\int_{\{\bfx\in E:|f(\bfx)-f_k(\bfx)|>\epsilon\}}|f-f_k|^p\leq\frac{1}{\epsilon^p}\int_E|f-f_k|^p \to 0 \]
  as $k\to\infty$.
  This proves that $f_k\overset{m}{\longrightarrow} f$ on $E$ and thus
  that there is a subsequence $f_{k_j}\to f$ a.e.\ in $E$ by Theorem 4.22.
\end{exercise}

\begin{exercise}{5.10}
  Suppose $p > 0$ and $\int_E \lvert f_k - f \rvert^p \rightarrow 0$.
  By Exercise 5.9, we know that $f_k \xrightarrow{m} f$ on $E$.
%  First we claim that $f_k \xrightarrow{m} f$ on $E$ (Exercise 5.9).
%  Given $\epsilon > 0$, Chebyshev's inequality states that
%  \[
%    \mea{\left\{ \lvert f - f_k \rvert > \epsilon \right\}}
%    \le \frac{1}{\epsilon^p} \int_E \lvert f_k - f \rvert^p.
%  \]
%  Letting $k \rightarrow \infty$
%  we see that $\mea{\left\{ \lvert f - f_k \rvert > \epsilon \right\}} \rightarrow 0$
%  as $k \rightarrow \infty$,
%  which is exactly $f_k \xrightarrow{m} f$.
  By Theorem 4.22, there exists a subsequence $\{f_{k_j}\} \rightarrow f$ a.e.\ in $E$.
  If $\int_E \lvert f_k \rvert^p \le M$ for all $k$,
  we conclude by Fatou's Lemma that
  \[
    \int_E \lvert f \rvert^p = \int_E \lim \lvert f_{k_j} \rvert^p
    \le \liminf \int_E \lvert f_{k_j} \rvert^p \le M.
    \qedhere
  \]
\end{exercise}

\begin{exercise}{5.18}
  The result of Exercise 5.16 suggests that if $0 < p < \infty$ and $f \ge 0$, then
  \[
    \int_E f^p = p \int_0^{\infty} \alpha^{p-1} \omega(\alpha) \rmd \alpha,
  \]
  regardless of the finiteness of either $\mea{E}$ or $\lVert f \rVert_p$.
  By definition of improper Riemann integral, we have
  \[
    \int_0^\infty \alpha^{p-1} \omega(\alpha) \rmd \alpha =
    \sum_{k = -\infty}^{+\infty}
    \int_{2^k}^{2^{k+1}} \alpha^{p-1} \omega(\alpha) \rmd \alpha
    \coloneqq \sum_{k = -\infty}^{+\infty} I_k.
  \]
  Since $\omega$ is decreasing, we have
  \[
    2^k \cdot 2^{k(p-1)} \omega(2^{k+1}) \le I_k \le 2^k \cdot 2^{(k+1)(p-1)} \omega(2^k).
  \]
  Simplifying, we obtain
  \[
    2^{-p} 2^{(k+1)p} \omega(2^{k+1}) \le I_k \le 2^{p-1} 2^{kp} \omega(2^k).
  \]
  Summing up, we find
  \[
    2^{-p} \sum 2^{(k+1)p} \omega(2^{k+1})
    = 2^{-p} \sum 2^{kp} \omega(2^k)
    \le \int_0^\infty \alpha^{p-1} \omega(\alpha) \rmd \alpha
    \le 2^{p-1} \sum 2^{kp} \omega(2^k).
  \]
  Consequently, all the terms in the above inequalities
  are simultaneously finite or infinite.
  We conclude that $f \in L^p$ if and only if $\sum 2^{kp} \omega(2^k) < + \infty$.
\end{exercise}


\begin{exercise}{5.21}
If $f$ is not zero almost everywhere.
Then at least one of the sets $\{f>0\}$ and $\{f<0\}$ has strictly positive measure.
Suppose we have $|\{f>0\}|>0$.
Since \[\{f>\frac{1}{n}\}\nearrow\{f>0\},\]
we have $\lim_{n\rightarrow\infty}|\{f>1/n\}|=|\{f>0\}|$.
Therefore for some $n_0$ sufficiently large,
$A=\{f>1/n_0\}$ has strictly positive measure.
In this case we can deduce tha\[\int_A f\geq\frac{|A|}{n_0}>0, \] contradiction.
\end{exercise}

\begin{exercise}{5.22}
We have a sequence of measurable functions $\{|f_k-f|\}$ on $E$
such that $|f_k-f|\rightarrow 0$ a.e.\ in $E$.
Now suppose on $E\setminus Z_k$, we have $|f_k|\leq \phi$ where $|Z_k|=0$.
Let \[Z=\bigcup_k Z_k\bigcup\{x\in E:f_k(x)\not\rightarrow f(x)\},\]
then $|Z|=0$ and on $E\setminus Z$,
we have $|f|\leq \phi$.
Therefore we deduce that $|f_k-f|\leq 2\phi$ a.e.\ in $E$.
Since $2\phi\in L(E)$, from Lebesgue's Dominated Convergence Theorem,
we have $\int_E|f_k-f|\rightarrow 0.$
\end{exercise}

\begin{exercise}{5.23}
Let $g_k=|f_k-f|$, then $g_k\rightarrow 0$ a.e.\ in $E$
and $|g_k|\leq \phi_k+\phi$ a.e.\ in $E$
and $\int_E(\phi_k+\phi)\rightarrow 2\int_E\phi.$
Therefore $\phi_k+\phi-g_k$ is nonnegative a.e.\ in $E$.
By Fatou's Lemma, we have
\[2\int_E\phi=\int_E\varliminf(\phi_k+\phi-g_k)\leq\varliminf\int_E\phi_k+\phi-g_k=2\int_E\phi-\varlimsup\int_Eg_k.\]
Therefore \[0\leq\varliminf\int_E g_k\leq\varlimsup\int_Eg_k\leq 0.\]
Thus $\lim\int_Eg_k=0.$
\end{exercise}

\begin{exercise}{5.24}
(a) Since $f\in L^p(E)$,
we have \[\alpha\omega_{|f|}(\alpha)\leq\int_{\{|f|>\alpha\}}|f|^p\leq\int_E|f|^p<\infty.\]
Thus \[\omega_{|f|}(\alpha)\leq\frac{\int_E|f|^p}{\alpha^p},\]
which implies $f$ belongs to weak $L^p(E)$.
Now let $E=(0,+\infty)$ and $f(x)=1/x$ defined on $E$,
then $\int_E f=\infty$.
But for any $\alpha>0$, we have $\omega_{|f|}(\alpha)=1/\alpha$.
Thus $f$ belongs to weak $L^1(E)$ but not $L^1(E)$.

(b) By definition we have nonnegative constants $A$ and $A'$
such that $\omega_{|f|}(\alpha)\leq A/\alpha$ and $\omega_{|f|}(\alpha)\leq A'/\alpha^r$.
Thus
\begin{align*}
  \int_E|f|^p&=p\int_0^\infty\alpha^{p-1}\omega_{|f|}(\alpha)d\alpha\\
  &=p\int_0^1\alpha^{p-1}\omega_{|f|}(\alpha)d\alpha+p\int_1^\infty\alpha^{p-1}\omega_{|f|}(\alpha)d\alpha\\
  &\leq p\int_0^1\frac{A}{\alpha^{2-p}}d\alpha+p\int_1^\infty \frac{A'}{\alpha^{1+r-p}}d\alpha
\end{align*}
Now since $2-p<1$ and $1+r-p>1$,
the right hand side integrals are finite,
which implies $f$ belongs to $L^p(E)$.

(c) Suppose $|f|<M$ for some $M\leq 0$.
Thus $\omega_{|f|}(\alpha)=0$ when $\alpha\geq M$.
We have
\[\int_E|f|^p=p\int_0^M\alpha^{p-1}\omega_{|f|}(\alpha)d\alpha\leq\int_0^M\frac{A}{\alpha^{2-p}}d\alpha<\infty\]
since $2-p<1.$
\end{exercise}
