
\begin{exercise}{7.4}
  Given $\epsilon > 0$, since $\mea{E_1} > 0$,
  there exists an open set $G_1$ such that
  $E_1 \subset G_1$ and $\mea{G_1} \le (1 + \epsilon) \mea{E_1}$.
  Write $G_1 = \bigcup I^{(1)}_k$ as a countable union of
  disjoint partly open dyadic intervals,
  and set $E^{(1)}_k = G_1 \cap I^{(1)}_k$.
  Clearly we have
  $\mea{G_1} = \sum \mea{I^{(1)}_k}$ and
  also $\mea{E_1} = \sum \mea{E^{(1)}_k}$,
  so there exists a $k_1$ such that
  \begin{equation}
    \mea{I^{(1)}_{k_1}} \le (1 + \epsilon) \mea{E^{(1)}_{k_1}}.
    \label{eq:measureEstimateForE1}
  \end{equation}
  Let $I^{(1)} \coloneqq I^{(1)}_{k_1}$ and $E^{(1)} \coloneqq E^{(1)}_{k_1}$.
  Repeating this construction for $E_2$, we have
  \[
    \mea{I^{(2)}} \le (1 + \epsilon) \mea{E^{(2)}}
  \]
  for some dyadic partly open interval $I^{(2)}$
  and some $E^{(2)} = G_2 \cap I^{(2)}$.

  If the length of $I^{(1)}$ is greater than that of $I^{(2)}$,
  we can subdivide $I^{(1)}$ to obtain
  subintervals which have the same length as $I^{(2)}$.
  Moreover,
  there must exist a subinterval, still denoted by $I^{(1)}$,
  such that \eqref{eq:measureEstimateForE1} holds.
  The opposite case is handled similarly.

  Suppose $I^{(1)} + q = I^{(2)}$ for some $q \in \bbR$.
  Note that for small $d > 0$,
  $E^{(1)} + (q+d)$ and $E^{(2)}$ are contained in
  an interval of length $|I^{(2)}| + |d|$.
  In case that they are disjoint, we have
  \[
    \frac{2}{1 + \epsilon} \mea{I^{(2)}} \le
    \mea{E^{(1)}} + \mea{E^{(2)}} \le \mea{I^{(2)}} + |d|.
  \]
  But this must be false for $\epsilon = \frac{1}{4}$
  and all $|d| < \frac{1}{2} |I^{(2)}|$.
  In other words, $E^{(1)} + (q+d)$ has nonempty intersection with $E^{(2)}$
  for all $|d| < \frac{1}{2}|I^{(2)}|$.
  We conclude that $E_2 - E_1$ must contain an open interval
  of length $|I^{(2)}|$ centered at $q$.
\end{exercise}

\begin{exercise}{7.5}
  Since $g$ is absoutely continuous,
  it is of bounded variation.
  It follows that all $f, g, h$ are of bounded variation,
  and that the Riemann-Stieltjes integrals
  $\int \phi \rmd f, \int \phi \rmd g$ and $\int \phi \rmd h$ exist.
  By linearity of Riemann-Stieltjes integral, we have
  \[
    \int_a^b \phi \rmd f = \int_a^b \phi \rmd g + \int_a^b \phi \rmd h.
  \]
  Next, we find
  \[
    \int_a^b \phi \rmd g = \int_a^b \phi g' \rmd x
    = \int_a^b \phi f' \rmd x,
  \]
  where the first equality follows from integration by parts
  and the second from the fact that $f' = g' + h' = g'$ a.e.
  We conclude that
  \[
    \int_a^b \phi \rmd f = \int_a^b \phi f' \rmd x + \int_a^b \phi \rmd h.
    \qedhere
  \]
\end{exercise}

\begin{exercise}{7.6}
  Let $\phi(x) = x^{\alpha}$ with $\alpha > 0$.
  Suppose $\epsilon > 0$ and $0 < M < +\infty$ are given.
  Since $\phi$ is increasing and right-hand continuous at $0$,
  there exists a $\delta_0 > 0$ such that
  $\phi(\delta_0) - \phi(0) < \epsilon/2$.
  Since $\phi$ is $C^1$ on $[\delta_0, M]$,
  we have
  \[
    L = \sup_{x \in [\delta_0, M]} \left| \phi'(x) \right| < \infty.
  \]
  Now choose $\delta = \min \left\{ \delta_0, {\epsilon}/{2 L} \right\}$,
  and let $\left\{ [a_k, b_k] \right\}_1^K$ be a collection of
  nonoverlapping intervals contained in $[0, M]$ subject to
  \[
    \sum_{k=1}^K \left(b_k - a_k\right) < \delta.
  \]
%  We may further assume that each $[a_k, b_k]$ does not contain
%  $\delta_0$ in its interior,
%  for otherwise we can split the interval into two
%  and the sum of lengths of intervals remains less than $\delta$.
  First asume that no $[a_k, b_k]$ contains $\delta_0$ in its interior.
  On one hand we have
  \[
    \sum_{b_k \le \delta_0} \left| \phi(b_k) - \phi(a_k) \right|
    \le \phi(\delta_0) - \phi(0) < \epsilon/2
  \]
  since $\phi$ is increasing.
  On the other hand we use Mean-Value Theorem to deduce
  \[
    \sum_{a_k \ge \delta_0} \vert \phi(b_k) - \phi(a_k) \vert
    \le L \sum_{a_k \ge \delta_0} \left( b_k - a_k \right)
    < L \delta \le \epsilon/2.
  \]
  Thus we have
  \[
    \sum_{k=1}^K \left| \phi(b_k) - \phi(a_k) \right|
    = \sum_{b_k \le \delta_0} \left| \phi(b_k) - \phi(a_k) \right|
    + \sum_{a_k \ge \delta_0} \left| \phi(b_k) - \phi(a_k) \right|
    < \epsilon.
  \]
  If $\delta_0$ is contained in some $(a_{k_0}, b_{k_0})$,
  we apply the above argument to the collection of intervals
  with $[a_{k_0}, b_{k_0}]$ replaced by
  $[a_{k_0}, \delta_0]$ and $[\delta_0, b_{k_0}]$.
  Then $\sum_{k=1}^K \left| \phi(b_k) - \phi(a_k) \right| < \epsilon$
  remains valid due to the triangular inequality
  $|\phi(b_{k_0}) - \phi(a_{k_0})| \le
  |\phi(b_{k_0}) - \phi(\delta_0)| + |\phi(\delta_0) - \phi(a_{k_0})|$.
  We have proved that $\phi$ is absoutely continuous on $[0, M]$.
\end{exercise}

\begin{exercise}{7.7}
  Necessity follows easily from the inequality
  \[
    \left| \sum \left[ f(b_i) - f(a_i) \right] \right|
    \le \sum \left| f(b_i) - f(a_i) \right|.
  \]
  We prove the sufficiency.
  Given $\epsilon > 0$,
  let $\delta$ be the value corresponding to $\epsilon/2$
  as dictated by the condition.
  Suppose $\left\{ [a_i, b_i] \right\}_{i=1}^m$ is a finite collection of
  nonoverlapping subintervals of $[a, b]$
  with $\sum \left( b_i - a_i \right) < \delta$.
  Define
  \[
    \begin{aligned}
      I^+ \coloneqq &\left\{ 1 \le i \le m : f(b_i) - f(a_i) \ge 0 \right\}, \\
      I^- \coloneqq &\left\{ 1 \le i \le m : f(b_i) - f(a_i) < 0 \right\}.
    \end{aligned}
  \]
  Since $\sum_{i \in I^{\pm}} \left( b_i - a_i \right) < \delta$,
  we have
  \[
    \left| \sum_{i \in I^\pm} \left( f(b_i) - f(a_i) \right) \right|
    < \epsilon/2.
  \]
  It follows that
  \[
    \begin{aligned}
      \sum_{i=1}^m \left| f(b_i) - f(a_i) \right|
      &=\sum_{i \in I^+} \left( f(b_i) - f(a_i) \right)
      - \sum_{i \in I^-} \left( f(b_i) - f(a_i) \right) \\
      &\le \left| \sum_{i \in I^+} \left( f(b_i) - f(a_i) \right) \right|
      + \left| \sum_{i \in I^-} \left( f(b_i) - f(a_i) \right) \right| \\
      &< \epsilon.
    \end{aligned}
  \]
  So $f$ is absolutely continuous.
\end{exercise}
