
Credit: 1-7 by Muhan; 8-13 by Heqing; 14-32 by Zhixuan.

\begin{exercise}{2.1}
  Since $|\sin{\frac{1}{x}}|\leq1$ for $0<x\leq 1$,
  we have $|f(x)|\leq x$ for $0<x\leq 1$,
  which is also true for $x=0$.
  Thus as $x$ tends to zero,
  $f(x)$ also tends to zero and equals $f(0)$.
  We have proved that $f$ is bounded and continuous.
  Now consider the partition
  $\Gamma=\{x_0=1,x_k=\frac{2}{k\pi}\text{ for }1\leq k\leq 2n, x_{2n+1}=1\}$.
  Since $f(x_{2n-1})=\pm\frac{2}{(2n-1)\pi}$ and $f(x_{2n})=0$,
  we have
  \[
    V[f;0,1]\geq S_{\Gamma}[f;0,1]>\sum_{k=1}^{n}|f(x_{2k-1})-f(x_{2k})|=\sum_{k=1}^{n}\frac{2}{(2k-1)\pi}.
  \]
  From knowledge in calculus,
  we know that the right hand sum of series tends to infinity
  when $n$ tends to infinity and thus $V[f;0,1]=+\infty$.
\end{exercise}

\begin{exercise}{2.2}
(i) For any $x\in [a,b]$, consider the partition $\Gamma=\{a,x,b\}$.
We have $V=V[f;a,b]\geq S_{\Gamma}[f;a,b]=|f(b)-f(x)|+|f(x)-f(a)|$.
We can deduce that
\[
  \frac{f(b)+f(a)-V}{2}\leq f(x)\leq \frac{V+f(a)+f(b)}{2},
\]
which implies that $f$ is bounded.\\
(ii) For any partition $\Gamma$, we have
\allowdisplaybreaks
\begin{align*}
  S_{\Gamma}[cf;0,1]&=|c|\cdot S_{\Gamma}[f;0,1];\\
  S_{\Gamma}[f+g;0,1]&=\sum_{i=0}^m|(f+g)(x_i)-(f+g)(x_{i-1})|\\
  &=\sum_{i=0}^m|f(x_i)-f(x_{i-1})+g(x_i)-g(x_{i-1})|\\
  &\leq\sum_{i=0}^m|f(x_i)-f(x_{i-1})|+\sum_{i=0}^m|g(x_i)-g(x_{i-1})|\\
  &=S_{\Gamma}[f;0,1]+S_{\Gamma}[g;0,1];\\
  S_{\Gamma}[fg;0,1]&=\sum_{i=0}^m|(fg)(x_i)-(fg)(x_{i-1})|\\
  &=\sum_{i=0}^m|g(x_i)[f(x_i)-f(x_{i-1})]+f(x_{i-1})[g(x_i)-g(x_{i-1})]|\\
  &\leq M\cdot(\sum_{i=0}^m|f(x_i)-f(x_{i-1})|+\sum_{i=0}^m|g(x_i)-g(x_{i-1})|)\\
  &=M\cdot (S_{\Gamma}[f;0,1]+S_{\Gamma}[g;0,1]).\\
  S_{\Gamma}[f/g;0,1]&=\sum_{i=0}^m|(f/g)(x_i)-(f/g)(x_{i-1})|\\
  &=\sum_{i=0}^m\bigg|\frac{f(x_i)-f(x_{i-1})}{g(x_i)}+f(x_{i-1})\frac{g(x_{i-1})-g(x_i)}{g(x_{i-1})g(x_i)}\bigg|\\
  &\leq \sum_{i=0}^m\frac{1}{\epsilon}|f(x_i)-f(x_{i-1})|+\frac{M}{\epsilon^2}\cdot|g(x_i)-g(x_{i-1})|\\
  &=\frac{1}{\epsilon}S_{\Gamma}[f;0,1]+\frac{M}{\epsilon^2}\cdot S_{\Gamma}[g;0,1]
\end{align*}
where by (i) we may assume that $M>0$ is an upper bound for both $f$ and $g$.
From these inequalities we can deduce that $cf$, $f+g$, $fg$ and $f/g$
are of bounded variation if $f$ and $g$ are so and $|g|\geq \epsilon$.
\end{exercise}

\begin{exercise}{2.3}
  For any partition $\Gamma$ of $[a',b']$,
  we can extend it to a partition $\Gamma'$ of $[a,b]$ by just adding two points $a$ and $b$.
  Then by definition we have
  \[
    P_{\Gamma}[f;a',b']\leq P_{\Gamma'}[f;a,b]\leq P[a,b]
  \]
  and
  \[
    N_{\Gamma}[f;a',b']\leq N_{\Gamma'}[f;a,b]\leq N[a,b]
  \]
  By taking supremum over all partitions of $[a',b']$ on the left hand side of the inequalities
  we get the desired result.
\end{exercise}

\begin{exercise}{2.4}
  For any partition $\Gamma=\{x_0,x_1,...,x_m\}$ of the interval $[a,b]$
  and any $\varepsilon>0$,
  there exists $N$ sufficiently large and when $k>N$,
  we have $|f(x_i)-f_k(x_i)|<\frac{\varepsilon}{2m}$
  for all $0\leq i\leq m.$
  Then for such $k$ we have
  \begin{align*}
    S_{\Gamma}[f;a,b]&=\sum_{i=1}^m|f(x_i)-f(x_{i-1})|\\
    &\leq \sum_{i=1}^m|f(x_i)-f_k(x_i)|+|f_k(x_i)-f_k(x_{i-1})|+|f_k(x_{i-1})-f(x_{i-1})|\\
    &\leq m\cdot \frac{\varepsilon}{2m}+M+m\cdot \frac{\varepsilon}{2m}=M+\varepsilon
  \end{align*}
  Since $\varepsilon$ is arbitrary,
  we have $S_{\Gamma}[f;a,b]\leq M$ and by taking supremum over all partitions of $[a,b]$,
  we get $V[f;a,b]\leq M.$

  Consider the functions $f_k$ defined over $[0,1]$
  where $f_k(x)=x\sin{\frac{1}{x}}$ on $(\frac{1}{k},1]$
  and equals 0 on $[0,\frac{1}{k}].$
  Then they are of bounded variation
  since they are of bounded variation on $[0,\frac{1}{k}]$ and $[\frac{1}{k},1]$
  and $V[f_k;0,1]=V[f_k;0,\frac{1}{k}]+V[f_k;\frac{1}{k},1].$
  However, their limit function is $f(x)=x\sin{\frac{1}{x}}$ for $0<x\leq 1$ and $f(0)=0$
  which is of unbounded variation by exercise 1.
\end{exercise}

\begin{exercise}{2.5}
  First we claim that $f$ is bounded on $[a,b]$.
  Otherwise, since $f$ is bounded on $[a+\varepsilon,b]$ for any $\varepsilon>0$,
  we can find a sequence $\{a_n\}$ converging to $a$ such that $|f(a_n)-f(a_{n-1})|>1$.
  Choose $N>M+1$ and on $[a_N,b]$ consider the partition $\Gamma=\{a_N,a_{N-1},...,a_1,b\}$,
  then $V[f;a_N,b]\geq S_{\Gamma}[f;a_N,b]\geq N-1>M$, contradiction.

  Now suppose $|f(x)|\leq L$ for some $L>0$.
  For any partition $\Gamma=\{x_0=a,x_1,...,x_m=b\}$,
  we have $S_{\Gamma}[f;a,b]\leq |f(a)-f(x_1)|+V[f;x_1,b]\leq 2L+M$.
  Take the supremum over all partitions of $[a,b]$ and we have $V[f;a,b]$ is finite.

  Suppose $f$ is continuous at $a$, that is, $\lim_{x\rightarrow{a+0}}f(x)=f(a)$,
  then for any $\varepsilon>0$,
  there exists $\delta>0$ such that $|f(x)-f(a)|<\varepsilon$ when $0\leq x-a<\delta$.
  Now for any partition $\Gamma=\{x_0=a,x_1,...,x_m=b\}$,
  we may assume $|x_1-a|<\delta$, otherwise we can refine it.
  Then $S_{\Gamma}[f;a,b]\leq |f(a)-f(x_1)|+V[f;x_1,b]\leq\varepsilon+M$.
  Since $\varepsilon$ is arbitrary,
  we have $S_{\Gamma}[f;a,b]\leq M$ and in consequence $V[f;a,b]\leq M.$
\end{exercise}

\begin{exercise}{2.6}
  For any $0<\varepsilon<1$,
  $f$ is finite and of bounded variation on $[\varepsilon,1]$
  and $f'(x)=2x\sin{\frac{1}{x}}-\cos{\frac{1}{x}}$ is continuous on $[\varepsilon,1]$.
  By Corollary 2.10,
  we have
  \[
    V[f;\varepsilon,1]=\int_{\epsilon}^1|f'|dx\leq\int_{\epsilon}^1 2x+1 dx=2-(\varepsilon^2+\varepsilon)<2.
  \]
  By the previous exercise we can deduce that $V[f;0,1]<+\infty.$
\end{exercise}

\begin{exercise}{2.7}
  For any $\varepsilon>0$,
  since $f$ is continuous at $\Bar{x}$,
  there exists $\delta>0$ such that when $|x-\Bar{x}|<\delta$
  we have $|f(x)-f(\Bar{x})|<\varepsilon$.
  Now for any such $x$, we may assume without loss of generality that $\Bar{x}<x\leq b$.
  For any $\varepsilon'>0$,
  there exists a partition $\Gamma$ of $[a,b]$
  such that $S_{\Gamma}\geq V[a,b]-\varepsilon'$.
  Let $\Gamma'=\Gamma\cup \{x,\Bar{x}\}$ and then
  $S_{\Gamma'}\geq S_{\Gamma}\geq V[a,b]-\varepsilon'$.
  On the other hand
  $S_{\Gamma'}\leq |f(x)-f(\Bar{x})|+V[a,\Bar{x}]+V[x,b]=|f(x)-f(\Bar{x})|+V[a,b]-V[\Bar{x},x]$.
  Thus $V[\Bar{x},x]\leq |f(x)-f(\Bar{x})|+\varepsilon'$ and
  consequently $V[\Bar{x},x]\leq |f(x)-f(\Bar{x})|<\varepsilon$
  by letting $\varepsilon'$ tend to zero.
  Since $V[\Bar{x},x]=V(x)-V(\Bar{x})$,
  we have proved that $V(x)$ is continuous at $\Bar{x}$.

  Now by the formulas $P(x)=\frac{V(x)+f(x)-f(a)}{2}$ and $N(x)=\frac{V(x)-f(x)+f(a)}{2}$,
  we have
  \begin{align*}
    |P(x)-P(\Bar{x})|&=\bigg|\frac{V(x)-V(\Bar{x})+f(x)-f(\Bar{x})}{2}\bigg|<\varepsilon, \\
    |N(x)-N(\Bar{x})|&=\bigg|\frac{V(x)-V(\Bar{x})-f(x)+f(\Bar{x})}{2}\bigg|<\varepsilon, \\
  \end{align*}
  which implies the continuity of $P$ and $N$ at $\Bar{x}$.
\end{exercise}


\begin{exercise}{2.8}
Let $f$ be of bounded variation on $(-\infty,+\infty)$, that is
\begin{equation}\label{jordan}
V(-\infty,+\infty)
=\lim _{a \to -\infty \atop  b \to+\infty}V[a,b]<\infty,
\end{equation}
where the limit exists in the sense that it is independent of how $a\to-\infty$ and $b\to+\infty$.

By Theorem 2.2(i) and  (\ref{jordan}), $f$ is of bounded variation on every interval
$[a,b]$ and
\begin{equation}\label{j2}
V[a,b]\leq V(-\infty,+\infty)
\end{equation}
for every $a,b$ with $-\infty<a<b<+\infty$.
By the analogue of Theorem 2.2(i) for $P$ and $N$ (see Exercise 3) and (\ref{j2}), it follows that $P[0,x]$and $N[0,x]$ are bounded and increasing on $[0,+\infty)$. Moreover, by Theorem 2.6 applied to $[0,x]$, $f(x)=[f(0)+P[0,x]]-N[0,x]$ when $0\leq x<+\infty$. Since $P[0,x]$ is bounded and increasing, so is $f(0)+P[0,x]$. Similarly, we have $f(x)=[f(0)-P[x,0]]+N[x,0]$ when $-\infty<x<0$.

Now we define
\[
  f_1(x)=\left\{\begin{array}{ll}
                  f(0)+P[0,x] & \text { if } x\geq0 \\
                  f(0)-P[x,0] & \text { if } x < 0,
  \end{array} \quad f_2(x)= \begin{cases} N[0,x]& \text { if } x\geq0 \\
  -N[x,0] & \text { if } x < 0 .\end{cases}\right.
\]
Then $f(x)=f_1(x)-f_2(x)$, where $f_1$ and $f_2$ are increasing bounded functions.
\end{exercise}

\begin{exercise}{2.9}
(a)
  Since $\phi$ and $\psi$ are of bounded variation, by Theorem 2.13, the curve $C$ is rectifiable, i.e. $L<+\infty$. Given $\epsilon>0$, we must find $\delta>0$ so that $L-l(\Gamma)<\epsilon$ if $|\Gamma|<\delta$. By the definition of $L$, $L=\sup_{\Gamma}l(\Gamma)$, we can choose a fixed partition $\bar{\Gamma}=\{\bar{t}_j\}_{j=0}^k$ of $[a,b]$ such that $l(\bar{\Gamma})>L-\epsilon/2$. Using the uniform continuity of $\phi$ and $\psi$ on $[a,b]$, pick $\eta>0$ such that
  \begin{enumerate}
    \item[(i)] $|\phi(t)-\phi(t^\prime)|<\epsilon/[4\sqrt{2}(k+1)]$ and $|\psi(t)-\psi(t^\prime)|<\epsilon/[4\sqrt{2}(k+1)]$ if $|t-t^\prime|<\eta$.

    Now let $\Gamma$ be any partition that satisfies
    \item[(ii)] $|\Gamma|<\eta$,
    \item[(iii)] $|\Gamma|<\min_j(\bar{t}_j-\bar{t}_{j-1}).$
  \end{enumerate}
  Write $\Gamma=\{t_i\}_{i=0}^m$ and
  $$l(\Gamma)=\sum_{i=1}^{m}\left(\left[\phi\left(t_{i}\right)-\phi\left(t_{i-1}\right)\right]^{2}+\left[\psi\left(t_{i}\right)-\psi\left(t_{i-1}\right)\right]^{2}\right)^{1 / 2}=\sideset{}{'}{\sum}+\sideset{}{''}{\sum} ,$$
  where $\sideset{}{''}{\sum}$ is extended over all $i$ such that $(t_{i-1},t_i)$ contains some $\bar{t}_j$. By (iii), any $(t_{i-1},t_i)$ can  contain at most one $\bar{t}_j$, and therefore the number of terms  $\sideset{}{''}{\sum}$ is at most $k+1$. Let $\Gamma\cup\bar{\Gamma}$ denote the partition formed by the union of the points of $\Gamma$ and $\bar{\Gamma}$. Then $\Gamma\cup\bar{\Gamma}$ is a refinement of both $\Gamma$ and $\bar{\Gamma}$. Moreover, $$l(\Gamma\cup\bar{\Gamma})=\sideset{}{'}{\sum}+\sideset{}{'''}{\sum},$$
  where $\sideset{}{'''}{\sum}$ is obtained from $\sideset{}{''}{\sum}$ by replacing each term by $$\left(\left[\phi\left(t_{i}\right)-\phi\left(\bar{t}_j\right)\right]^{2}+\left[\psi\left(t_{i}\right)-\psi\left(\bar{t}_j\right)\right]^{2}\right)^{1 / 2}+\left(\left[\phi\left(\bar{t}_j\right)-\phi\left({t}_{i-1}\right)\right]^{2}+\left[\psi\left(\bar{t}_j\right)-\psi\left({t}_{j-1}\right)\right]^{2}\right)^{1 / 2},$$
  $\bar{t}_j$ being the point of $\bar{\Gamma}$ in $(t_{j-1},t_j)$. By (i) and (ii), each of these two terms is less than $\epsilon/[4(k+1)]$, and therefore
  \begin{equation*}
    \sideset{}{'''}{\sum}<2(k+1)\frac{\epsilon}{4(k+1)}=\frac{\epsilon}{2}.
  \end{equation*}
  Hence,
  $$l(\Gamma)\geq\sideset{}{'}{\sum}=l(\Gamma\cup \bar{\Gamma})-\sideset{}{'''}{\sum}>l(\Gamma\cup \bar{\Gamma})-\frac{\epsilon}{2}\geq l(\bar{\Gamma})-\frac{\epsilon}{2}>L-\frac{\epsilon}{2}-\frac{\epsilon}{2}=L-\epsilon.$$
  We complete the proof by choosing $\delta=\min\{\eta,\min_j(\bar{t}_j-\bar{t}_{j-1})\}$.
  (b)
  Given a partition $\Gamma=\{t_i\}_{i=0}^m$
  of the interval $[a,b]$.
  Since $\phi$ and $\psi$ are differentiable on $[a,b]$, the mean-value theorem implies that there exist $\eta_i,\sigma_i\in (t_{i-1},t_i)$ such that
  $$\phi(t_i)-\phi(t_{i-1})=\phi^\prime(\eta_i)(t_i-t_{i-1}),\quad \psi(t_i)-\psi(t_{i-1})=\phi^\prime(\sigma_i)(t_i-t_{i-1}).$$
  Then
  \begin{equation*}
    \begin{aligned}
      l(\Gamma)=\sum_{i=1}^m\sqrt{[\phi^\prime(\eta_i)]^2+[\psi^\prime(\sigma_i)]^2}(t_i-t_{i-1}).
    \end{aligned}
  \end{equation*}
  Now we consider the Riemann sum
  $$R_\Gamma=\sum_{i=1}^m\sqrt{[\phi^\prime(\xi_i)]^2+[\psi^\prime(\xi_i)]^2}(t_i-t_{i-1}),\quad (\xi_i\in[t_{i-1},t_i]).$$
  We will use the inequalities
  \begin{equation*}
    \left|\sqrt{x_1^2+x_2^2}-\sqrt{y_1^2+y_2^2}\right|\leq \sqrt{(x_1-y_1)^2+(x_2-y_2)^2}\leq |x_1-y_1|+|x_2-y_2|,
  \end{equation*}
  where the first inequality is nothing but the reverse triangle inequality.
  Hence,
  \begin{equation*}
    \begin{aligned}
      |l(\Gamma)-R_\Gamma|&=\left|\sum_{i=1}^m\sqrt{[\phi^\prime(\eta_i)]^2+[\psi^\prime(\sigma_i)]^2}-\sum_{i=1}^m\sqrt{[\phi^\prime(\xi_i)]^2+[\psi^\prime(\xi_i)]^2}\right|(t_i-t_{i-1})\\&\leq\sum_{i=1}^m\left|\sqrt{[\phi^\prime(\eta_i)]^2+[\psi^\prime(\sigma_i)]^2}-\sqrt{[\phi^\prime(\xi_i)]^2+[\psi^\prime(\xi_i)]^2}\right|(t_i-t_{i-1})\\
      &\leq\sum_{i=1}^m|\phi^\prime(\eta_i)-\psi^\prime(\xi_i)|(t_i-t_{i-1})+\sum_{i=1}^m|\psi^\prime(\sigma_i)-\psi^\prime(\xi_i)|(t_i-t_{i-1})\\
      &\leq [U_\Gamma(\phi^\prime)-L_\Gamma(\phi^\prime)]+[U_\Gamma(\psi^\prime)-L_\Gamma(\psi^\prime)],
    \end{aligned}
  \end{equation*}
  where $U_\Gamma$ and $L_\Gamma$ are the upper and lower Riemann sums for $\Gamma$, respectively. Since $\phi^\prime$ and $\psi^\prime$ are continuous on $[a,b]$ and then are Riemann integrable on $[a,b]$, we have
  $$\lim_{|\Gamma|\to 0}[U_\Gamma(\phi^\prime)-L_\Gamma(\phi^\prime)]=0,\quad \lim_{|\Gamma|\to 0}[U_\Gamma(\psi^\prime)-L_\Gamma(\psi^\prime)]=0.$$
  Note that $\phi$ and $\psi$ are continuously differentiable on $[a,b]$ and then are of bounded variation on $[a,b]$ by Example 6. Using the result of part (a) we find
  \[L=\lim_{|\Gamma|\to0}l(\Gamma)=\lim_{|\Gamma|\to0}R_\Gamma=\int_a^b\left([\phi^\prime(t)]^2+[\psi^\prime(t)]^2\right)^{1/2}\mathrm{d}t. \qedhere \]
\end{exercise}

\begin{exercise}{2.10}
Let $a=\lambda_1<\lambda_2<\cdots<\lambda_m=b$ and
$\phi(x)$ be a step function on $[a,b]$ such that $\phi$ is constant on each interval $(\lambda_{i-1},\lambda_i)$.
Let
$$\phi(\lambda_i+)=\lim_{x\to\lambda_i+}\phi(x),i=1,2,\cdots,m-1,$$
and
$$\phi(\lambda_i-)=\lim_{x\to\lambda_i-}\phi(x),i=2,\cdots,m,$$
denote the limits from the right and left at $\lambda_i$, and let $a_i=\phi(\lambda_{i}+)-\phi(\lambda_{i}-)$, $i=2,\cdots,m-1$, $a_1=\phi(\lambda_1+)-\phi(\lambda_1)$, and $a_m=\phi(\lambda_m)-\phi(\lambda_m-)$ denote the jumps of $\phi$. By Remark 3 (page 28), for $f(x)=e^{-sx}$, we have
$$\sum_{k=1}^ma_ke^{-s\lambda_k}=\sum_{k=1}^mf(\lambda_k)a_k=\int_a^bf\mathrm{d}\phi.$$
\end{exercise}

\begin{exercise}{2.11}
Suppose that $I=\int_a^bf\mathrm{d}\phi$ exists and hence is finite, that is, if given $\epsilon>0$, there is a $\delta>0$ such that $|I-R_{\Gamma}|<\epsilon/2$ for any $\Gamma$ satisfying $|\Gamma|<\delta$ and for any choice of intermediate points. For such $\delta$, we have
$$|R_\Gamma-R_{\Gamma^\prime}|\leq|R_\Gamma-I|+|I-R_{\Gamma^\prime}|<\frac{\epsilon}{2}+\frac{\epsilon}{2}=\epsilon,\quad \text{if}\quad |\Gamma|,|\Gamma^\prime|<\delta.$$

Conversely, suppose that for $\epsilon_n=1/n$, there exist $\delta_n>0$ such that $|R_{\Gamma_n}-R_{\Gamma^\prime_n}|<\epsilon_n$ if $|\Gamma_n|,|\Gamma^\prime_n|<\delta_n$. We may assume that $\delta_1\geq \delta_2\geq\cdots\geq\delta_n\geq\cdots$.
Consider a sequence $\{R_{\Gamma_n}\}_{n=1}^{\infty}$ for some choice of partitions $\Gamma_n$ with $|\Gamma_n|<\delta_n$.
For any $\epsilon>0$, we choose a positive integer $N$ such
that $\epsilon_N< \epsilon$, then
\begin{equation}\label{11}
|R_{\Gamma_n}-R_{\Gamma_m}|<\epsilon,\quad \text{if} \quad n,m\geq N.
\end{equation}
Hence, $\{R_{\Gamma_n}\}_{n=1}^{\infty}$ is a Cauchy sequence in $\mathbf{R^1}$ and then there is a number $I$ such that $I=\lim_{n\to\infty} R_{\Gamma_n}$.
Letting $m\to+\infty$ in (\ref{11}) and then we have
\begin{equation*}		|R_{\Gamma_N}-I|\leq\epsilon.
\end{equation*}
For such $\epsilon$ and $N$ in (\ref{11}), we choose $\delta\in(0,\delta_N]$, then for any $\Gamma$ satisfying $|\Gamma|<\delta$,
\begin{equation*}
  \begin{aligned}
    |R_{\Gamma}-I|&\leq |R_{\Gamma}-R_{\Gamma_	N}|+|R_{\Gamma_N}-I|\\
    &\leq \epsilon_N+\epsilon\\&<2\epsilon.
  \end{aligned}
\end{equation*}
It follows that $\int_a^bf\mathrm{d}\phi=I$.
\end{exercise}

\begin{exercise}{2.12}
We will give the argument for the upper sums; the one for the lower sums is similar.
If $\phi$ is a constant, then $\lim_{|\Gamma|\to0}U_\Gamma=\inf_{\Gamma}U_\Gamma=0$.
This completes the proof.

Now we assume that $\phi$ is not a constant.
Let $\inf_{\Gamma}U_\Gamma=U$. Given $\epsilon>0$, we must find $\delta>0$ such that $U_{\Gamma}<U+\epsilon$ if $|\Gamma|<\delta$. Choose $\bar{\Gamma}=\{\bar{x}_j\}_{j=0}^k$ such that $U_{\bar{\Gamma}}<U+\epsilon/2$, and let $M=\sup_{[a,b]}|f|$. Since $f$ and $\phi$ have no common discontinuities, we have the fact that  either $f$ or $\phi$ is continuous at each point $\bar{x}_j$ of $\bar{\Gamma}$. There exits $\eta_1>0$ such that
$$|\phi(x)-\phi(\bar{x}_j)|<\frac{\epsilon}{8(k+1)M}$$
if $|x-\bar{x}_j|<\eta_1$ and  $\phi$ is continuous at $\bar{x}_j$.
We can find such $\eta$ because there are only a finite number of such points $\bar{x}_j$.
Similarly, we can find $\eta_2>0$ such that
$$|f(x)-f(\bar{x}_j)|<\frac{\epsilon}{8(k+1)[\phi(b)-\phi(a)]}$$
if $|x-\bar{x}_j|<\eta_2$ and $f$ is continuous at $\bar{x}_j$.
Note that here $\phi$ is monotone increasing and finite, so $0<\phi(b)-\phi(a)<+\infty$.

We set $\delta=\min\{\eta_1,\eta_2,\min_j(\bar{x}_j-\bar{x}_{j-1})\}$.
Now let $\Gamma=\{x_i\}_{i=0}^m$ be any partition for which $|\Gamma|<\delta$.
It is enough to show that $U_{\Gamma}<U+\epsilon.$
Write
$$U_\Gamma=\sum_{i=1}^mM_i[\phi(x_i)-\phi(x_{i-1})]=\sideset{}{'}{\sum}+\sideset{}{''}{\sum},$$
where $\sideset{}{'}{\sum}$ is as in the proof of Theorem 2.9. Then $U_{\Gamma\cup\bar{\Gamma}}=\sideset{}{'}{\sum}+\sideset{}{'''}{\sum}$, where $\sideset{}{'''}{\sum}$ is obtained from $\sideset{}{''}{\sum}$ by replacing each of the terms $M_i[\phi(x_i)-\phi(x_{i-1})]$ by
\begin{equation}\label{230}
\sup _{\left[x_{i-1}, \bar{x}_{j}\right]} f(x)\left[\phi\left(\bar{x}_{j}\right)-\phi\left(x_{i-1}\right)\right]+\sup _{\left[\bar{x}_{j}, x_{i}\right]} f(x)\left[\phi\left(x_{i}\right)-\phi\left(\bar{x}_{j}\right)\right],
\end{equation}
$\bar{x}_{j}$ being the point of $\bar{\Gamma}$ in $(x_{i-1},x_i)$. Hence, $U_{\Gamma}-U_{\Gamma\cup\bar{\Gamma}}=\sideset{}{''}{\sum}-\sideset{}{'''}{\sum}$.
At least one of $\sup _{\left[x_{i-1}, \bar{x}_{j}\right]}f$ and $\sup _{\left[\bar{x}_{j}, x_{i}\right]} f$ equals $M_i$. If it is the first, the difference between $M_i[\phi(x_i)-\phi(x_{i-1})]$ and (\ref{230}) is easily seen to be
$$\left(M_{i}-\sup _{\left[\bar{x}_{j}, x_{i}\right]} f\right)\left[\phi\left(x_{i}\right)-\phi\left(\bar{x}_{j}\right)\right].$$
If it is the second, the difference is
$$\left(M_{i}-\sup _{\left[x_{i-1}, \bar{x}_{j}\right]} f\right)\left[\phi\left(\bar{x}_{j}\right)-\phi\left(x_{i-1}\right)\right].$$
In either case, the difference is at most $2M\epsilon/[8(k+1)M]+2\epsilon/[8(k+1)]=\epsilon/[2(k+1)]$ in absolute value. Here the second term $2\epsilon/[8(k+1)]$ comes from, for example, $f$ is continuous at $\bar{x}_j$,
\begin{equation*}
  \begin{aligned}
    \left(M_{i}-\sup _{\left[\bar{x}_{j}, x_{i}\right]} f\right)\left[\phi\left(x_{i}\right)-\phi\left(\bar{x}_{j}\right)\right] \leq&\left(|M_i-f(\bar{x}_j)|+|f(\bar{x}_j)-\sup _{\left[\bar{x}_{j}, x_{i}\right]} f|\right)\left[\phi\left(x_{i}\right)-\phi\left(\bar{x}_{j}\right)\right]\\
    \leq&\frac{2\epsilon}{8(k+1)[\phi(b)-\phi(a)]}\left[\phi\left(x_{i}\right)-\phi\left(\bar{x}_{j}\right)\right]\\
    \leq&\frac{2\epsilon}{8(k+1)}.
  \end{aligned}
\end{equation*}
Hence,
\begin{equation*}
  U_{\Gamma}\leq U_{\Gamma\cup\bar{\Gamma}}+(k+1)\frac{\epsilon}{2(k+1)}\leq U_{\bar{\Gamma}}+\frac{\epsilon}{2}<U+\frac{\epsilon}{2}+\frac{\epsilon}{2}=U+\epsilon,
\end{equation*}
which completes the proof.
\end{exercise}

\begin{exercise}{2.13}
Let $\Gamma=\{a=x_0<x_1<\cdots<x_m=b\}$ and $x_{i-1}\leq\xi_i\leq x_i$.

(i) Suppose $\int_a^bf\mathrm{d}\phi$ exists. Then for any constant $c$,
\begin{equation*}
  \begin{aligned}
    cR_\Gamma(f;\phi)&=c\sum_{i=1}^mf(\xi_i)[\phi(x_i)-\phi(x_{i-1})]\\
    &=\sum_{i=1}^mcf(\xi_i)[\phi(x_i)-\phi(x_{i-1})]\\
    &=\sum_{i=1}^mf(\xi_i)[c\phi(x_i)-c\phi(x_{i-1})],
  \end{aligned}
\end{equation*}
that is, $cR_\Gamma(f;\phi)=	R_\Gamma(cf;\phi)=	R_\Gamma(f;c\phi)$. Since $\int_a^bf\mathrm{d}\phi$ exists, we get
$$\lim_{|\Gamma|\to 0}	R_\Gamma(cf;\phi)=\lim_{|\Gamma|\to 0}	R_\Gamma(f;c\phi)=c\int_a^bf\mathrm{d}\phi,$$
that is $\int_a^bcf\mathrm{d}\phi$ and $\int_a^bf\mathrm{d}(c\phi)$ exist and
$$\int_a^bcf\mathrm{d}\phi=\int_a^bf\mathrm{d}(c\phi)=c\int_a^bf\mathrm{d}\phi.$$

(ii) Suppose $\int_a^bf_1\mathrm{d}\phi$ and $\int_a^bf_2\mathrm{d}\phi$ exist. Then
\begin{equation*}
  \begin{aligned}
    R_\Gamma(f_1;\phi)+R_\Gamma(f_2;\phi)&=\sum_{i=1}^mf_1(\xi_i)[\phi(x_i)-\phi(x_{i-1})]+\sum_{i=1}^mf_2(\xi_i)[\phi(x_i)-\phi(x_{i-1})]\\
    &=\sum_{i=1}^m[f_1(\xi_i)+f_2(\xi_i)][\phi(x_i)-\phi(x_{i-1})]\\&=R_\Gamma(f_1+f_2;\phi).
  \end{aligned}
\end{equation*}
Since $\int_a^bf_1\mathrm{d}\phi$ and  $\int_a^bf_2\mathrm{d}\phi$ both exist, we get
$$\lim_{|\Gamma|\to 0}	R_\Gamma(f_1+f_2;\phi)=\int_a^bf_1\mathrm{d}\phi+\int_a^bf_2\mathrm{d}\phi,$$
that is, $\int_a^b(f_1+f_2)\mathrm{d}\phi$ exists and
$$\int_a^b(f_1+f_2)\mathrm{d}\phi=\int_a^bf_1\mathrm{d}\phi+\int_a^bf_2\mathrm{d}\phi.$$

(iii) Suppose $\int_a^bf\mathrm{d}\phi_1$ and $\int_a^bf\mathrm{d}\phi_2$ exist. Then
\begin{equation*}
  \begin{aligned}
    R_\Gamma(f;\phi_1)+R_\Gamma(f;\phi_2)&=\sum_{i=1}^mf(\xi_i)[\phi_1(x_i)-\phi_1(x_{i-1})]+\sum_{i=1}^mf(\xi_i)[\phi_2(x_i)-\phi_2(x_{i-1})]\\
    &=\sum_{i=1}^mf(\xi_i)\left\{[\phi_1(x_i)+\phi_2(x_i)]-[\phi_1(x_{i-1})+\phi_2(x_{i-1})]\right\}\\&=R_\Gamma(f;\phi_1+\phi_2).
  \end{aligned}
\end{equation*}
Since $\int_a^bf\mathrm{d}\phi_1$ and  $\int_a^bf\mathrm{d}\phi_2$ both exist, we get
$$\lim_{|\Gamma|\to 0}	R_\Gamma(f;\phi_1+\phi_2)=\int_a^bf\mathrm{d}\phi_1+\int_a^bf\mathrm{d}\phi_2,$$
that is, $\int_a^bf\mathrm{d}(\phi_1+\phi_2)$ exists and
\[
  \int_a^bf\mathrm{d}(\phi_1+\phi_2)=\int_a^bf\mathrm{d}\phi_1+\int_a^bf\mathrm{d}\phi_2.
  \qedhere
\]
\end{exercise}

\begin{exercise}{2.14}
  This is the example following (2.28).
  Take $[a,b] = [-1, 1], c=0$, and
  \[
    f(x) = \left\{
    \begin{aligned}
      0 & \quad \text{if} \quad -1 \le x < 0 \\
      1 & \quad \text{if} \quad 0 \le x \le 1,
    \end{aligned}
    \right.
    \quad
    \phi(x) = \left\{
    \begin{aligned}
      0 & \quad \text{if} \quad -1 \le x \le 0 \\
      1 & \quad \text{if} \quad 0 < x \le 1.
    \end{aligned}
    \right.
  \]
  $R_\Gamma(f\rmd \phi; [-1, 1])$ can be 0 or 1 for any $\Gamma$ straddling $0$,
  so $\int_{-1}^1 f\rmd \phi$ does not exist.
  Since $\phi \equiv 0$ on $[-1, 0]$,
  $R_\Gamma(f \rmd \phi; [-1, 0])$ equals to zero for any $\Gamma$
  and hence $\int_{-1}^0 f \rmd \phi = 0$.
  Finally, if $\Gamma = \{x_j\}$ is a partition of $[0, 1]$, then
  \[
    R_\Gamma(f \rmd \phi; [0, 1]) = f(\xi_1) \left( \phi(x_1) - \phi(x_0) \right)
    = 1 \cdot (1-0) = 1.
  \]
  Hence $\int_0^1 f \rmd \phi = 1$.
\end{exercise}

\begin{exercise}{2.15}
  Let $\Gamma = \{x_j\}_1^N$ be a partition of $[a,b]$. Then
  \[
    S_\Gamma = \sum \left\vert \psi(x_j) - \psi(x_{j-1}) \right\vert
    = \sum \left\vert \int_{x_{j-1}}^{x_j} f \rmd \phi \right\vert
    \le \left(\sup |f|\right) \sum V(\phi; [x_{j-1}, x_j])
    \le \left(\sup |f|\right) V(\phi; [a,b]),
  \]
  where the first inequality follows from Theorem 2.24.
  This holds for all $\Gamma$ and thus $\psi$ is of bounded variation.
  If $g$ is continuous on $[a,b]$,
  both $\int_a^b g \rmd \psi$ and $\int_a^b fg \rmd \phi$ exist
  by Theorem 2.24.
  Consider the Riemann-Stieltjes sum for $g \rmd \psi$, we have
  \begin{align*}
    & \sum g(\xi_j) \left( \psi(x_j) - \psi(x_{j-1}) \right) \\
    = & \sum g(\xi_j) f(\eta_j) \left( \phi(x_j) - \phi(x_{j-1}) \right) \\
    = & \sum g(\xi_j) f(\xi_j) \left( \phi(x_j) - \phi(x_{j-1}) \right)
    + \sum g(\xi_j) \left( f(\eta_j) - f(\xi_j) \right)
    \left( \phi(x_j) - \phi(x_{j-1}) \right) \\
    \coloneqq & I_1 + I_2,
  \end{align*}
  where the first equality follows from Theorem 2.27 (Mean-Value Theorem)
  with each $\eta_j \in [x_{j-1}, x_j]$.
  The term $I_1$ is the Riemann-Stieltjes sum for $gf \rmd \phi$
  and tends to $\int_a^b fg \rmd \phi$ as $|\Gamma| \rightarrow 0$.
  Using the uniform continuity of $f$ on $[a,b]$,
  we can select $|\Gamma|$ so small that
  \[
    \left\vert f(\eta_j) - f(\xi_j) \right\vert <
    \frac{\epsilon} {\left( \sup g \right) V(\phi)}
  \]
  for every $1 \le j \le N$.
  It follows that $|I_2| < \epsilon$
  and consequently $\lim_{|\Gamma| \rightarrow 0} I_2 = 0$.
  The proof of $\int_a^b g \rmd \psi = \int_a^b fg \rmd \phi$ is thus completed.

\end{exercise}

\begin{exercise}{2.16}
  In view of Corollary 2.7 (Jordan's Theorem),
  we assume that $\phi$ is monotone increasing.
  Suppose $\sup f = M$ and $\{y_j\}_1^L$ are the jump discontinuities.
  By the continuity of $\phi$ at $\{y_j\}$,
  for each $j$ there exists a closed interval $I_j$ containing $y_j$ such that
  \begin{equation}
    \left| \phi(y) - \phi(z) \right| < \frac{\epsilon}{4 M L}
    \label{eq:estOfPhi}
  \end{equation}
  for any $y, z \in I_j$.
  By the uniform continuity of $f$ on $[a,b] \setminus \bigcup I_j^\circ$,
  select $\delta'$ so small that
  \begin{equation}
    \left| f(y) - f(z) \right| < \frac{\epsilon}{2 V(\phi)}
    \label{eq:estOfF}
  \end{equation}
  for $y,z \in [a,b] \setminus \bigcup I_j^\circ$ and $|y-z| < \delta'$.
  Now suppose $\Gamma = \{x_k\}$ is a partition of $[a,b]$ satisfying
  $\left| \Gamma \right| < \min \left\{ \delta', |I_1|, \cdots, |I_L| \right\}$.
  Divide the closed intervals in $\Gamma$ into two subcollections:
  Let  $I \in \Gamma_1$
  if $I$ has nonempty intersection with $\{y_j\}$
  and let $I \in \Gamma_2$ otherwise.
  We estimate the difference between the upper and lower Riemann-Stieltjes sums as follows.
  First by \eqref{eq:estOfPhi} we have
  \[
    \left| \sum_{[x_{k-1}, x_k] \in I_1}
    \left( M_k - m_k \right) \left( \phi(x_k) - \phi(x_{k-1}) \right) \right|
    \le 2M \cdot L \cdot \frac{\epsilon}{4ML}
    \le \frac{\epsilon}{2}
  \]
  since there are at most $L$ intervals in the subcollection $\Gamma_1$.
  Second by \eqref{eq:estOfF} we have
  \[
    \left| \sum_{[x_{k-1}, x_k] \in I_2}
    \left( M_k - m_k \right) \left( \phi(x_k) - \phi(x_{k-1}) \right) \right|
    \le \frac{\epsilon}{2V(\phi)} \cdot V(\phi)
    \le \frac{\epsilon}{2}.
  \]
  We have shown $\lim_{|\Gamma| \rightarrow 0} (U_\Gamma - L_\Gamma) = 0$.
  It follows from $L_\Gamma \le R_\Gamma \le U_\Gamma$ that
  the integral $\int_a^b f \rmd \phi = \lim R_\Gamma$ exists.
\end{exercise}

\begin{exercise}{2.17}
  In view of Corollary 2.7 (Jordan's Theorem),
  we assume that $\phi$ is monotone increasing.
  Set $S_n = \int_{-n}^n f \rmd \phi$.
  We show that the sequence $\{S_n\}$ is Cauchy.
  Let $\phi(-\infty) = M^-$ and $\phi(+\infty) = M^+$.
  Since $\phi$ is monotone increasing,
  there exists $N_1 > 0$ so large that
  $\phi(x) > M^+ - \sqrt{\epsilon} / 2$ for all $x > N_1$
  and $\phi(x) < M^- + \sqrt{\epsilon} / 2$ for all $x < -N_1$.
  Since $f$ is continuous and $f(\infty) = 0$,
  there exists $N_2 > 0$ so large that
  $|f(x)| < \sqrt{\epsilon}$ for all $|x| > N_2$.
  Now if $n \ge m > \max\{N_1, N_2\}$, we have
  \[
    \begin{aligned}
      \left| S_n - S_m \right| =
      & \left| \int_{-n}^{-m} f \rmd \phi + \int_m^n f \rmd \phi \right| \\
      \le & \left( \sup_{|x| \ge m} |f| \right)
      \left( V(\phi;[-n,-m]) + V(\phi;[m,n]) \right) \\
      \le & \sqrt{\epsilon}
      \left( \phi(-m) - \phi(-\infty) + \phi(+\infty) - \phi(m) \right) \\
      < & \sqrt{\epsilon}
      \left( \frac{\sqrt{\epsilon}}{2} + \frac{\sqrt{\epsilon}}{2} \right) \\
      = &\epsilon,
    \end{aligned}
  \]
  where the first inequality follows from Theorem 2.24.
  Thus the improper integral $\int_{-\infty}^{+\infty} f \rmd \phi = \lim S_n$ exists.
\end{exercise}

\begin{exercise}{2.31}
  For the first part,
  note that the Riemann-Stieltjes sum
  \[
    R_\Gamma(\rmd f; [a,b]) = \sum f(x_j) - f(x_{j-1}) = f(b) - f(a)
  \]
  is independent of the partition $\Gamma$.
  For the second part, we have by the Mean-Value Theorem
  \[
    \sum f(x_j) - f(x_{j-1}) = \sum f'(\xi_j) \left( x_j - x_{j-1} \right)
  \]
  with $\xi_j \in (x_{j-1}, x_j)$.
  Since $f'$ is Riemann integrable,
  by taking the limit $|\Gamma| \rightarrow 0$ of both sides we obtain
  $\int_a^b \rmd f = \int_a^b f' \rmd x$.
\end{exercise}

\begin{exercise}{2.32}
  Since $af(a)$ and $bf(b)$ are finite,
  it follows from Theorem 2.21 (integration by parts) that
  the (Riemann) integrability of $f \rmd x$
  is equivalent to that of $x \rmd f$.
  But $x$ is continuous and $f$ is of bounded variation on $[a,b]$,
  the integrability follows from Theorem 2.24.
\end{exercise}
