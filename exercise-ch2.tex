
\begin{exercise}{2.14}
  This is the example following (2.28).
  Take $[a,b] = [-1, 1], c=0$, and
  \[
    f(x) = \left\{
    \begin{aligned}
      0 & \quad \text{if} \quad -1 \le x < 0 \\
      1 & \quad \text{if} \quad 0 \le x \le 1,
    \end{aligned}
    \right.
    \quad
    \phi(x) = \left\{
    \begin{aligned}
      0 & \quad \text{if} \quad -1 \le x \le 0 \\
      1 & \quad \text{if} \quad 0 < x \le 1.
    \end{aligned}
    \right.
  \]
  $R_\Gamma(f\rmd \phi; [-1, 1])$ can be 0 or 1 for any $\Gamma$ straddling $0$,
  so $\int_{-1}^1 f\rmd \phi$ does not exist.
  Since $\phi \equiv 0$ on $[-1, 0]$,
  $R_\Gamma(f \rmd \phi; [-1, 0])$ equals to zero for any $\Gamma$
  and hence $\int_{-1}^0 f \rmd \phi = 0$.
  Finally, if $\Gamma = \{x_j\}$ is a partion of $[0, 1]$, then
  \[
    R_\Gamma(f \rmd \phi; [0, 1]) = f(\xi_1) \left( \phi(x_1) - \phi(x_0) \right)
    = 1 \cdot (1-0) = 1.
  \]
  Hence $\int_0^1 f \rmd \phi = 1$.
\end{exercise}

\begin{exercise}{2.15}
  Let $\Gamma = \{x_j\}_1^N$ be a partition of $[a,b]$. Then
  \[
    S_\Gamma = \sum \left\vert \psi(x_j) - \psi(x_{j-1}) \right\vert
    = \sum \left\vert \int_{x_{j-1}}^{x_j} f \rmd \phi \right\vert
    \le \left(\sup |f|\right) \sum V(\phi; [x_{j-1}, x_j])
    \le \left(\sup |f|\right) V(\phi; [a,b]),
  \]
  where the first inequality follows from Theorem 2.24.
  This holds for all $\Gamma$ and thus $\psi$ is of bounded variation.
  If $g$ is continuous on $[a,b]$,
  both $\int_a^b g \rmd \psi$ and $\int_a^b fg \rmd \phi$ exist
  by Theorem 2.24.
  Consider the Riemann-Stieltjes sum for $g \rmd \psi$, we have
  \begin{align*}
    & \sum g(\xi_j) \left( \psi(x_j) - \psi(x_{j-1}) \right) \\
    = & \sum g(\xi_j) f(\eta_j) \left( \phi(x_j) - \phi(x_{j-1}) \right) \\
    = & \sum g(\xi_j) f(\xi_j) \left( \phi(x_j) - \phi(x_{j-1}) \right)
    + \sum g(\xi_j) \left( f(\eta_j) - f(\xi_j) \right)
    \left( \phi(x_j) - \phi(x_{j-1}) \right) \\
    \coloneqq & I_1 + I_2,
  \end{align*}
  where the first equality follows from Theorem 2.27 (Mean-Value Theorem)
  with each $\eta_j \in [x_{j-1}, x_j]$.
  The term $I_1$ is the Riemann-Stieltjes sum for $gf \rmd \phi$
  and tends to $\int_a^b fg \rmd \phi$ as $|\Gamma| \rightarrow 0$.
  Using the uniform continuity of $f$ on $[a,b]$,
  we can select $|\Gamma|$ so small that
  \[
    \left\vert f(\eta_j) - f(\xi_j) \right\vert <
    \frac{\epsilon} {\left( \sup g \right) V(\phi)}
  \]
  for every $1 \le j \le N$.
  It follows that $|I_2| < \epsilon$
  and consequently $\lim_{|\Gamma| \rightarrow 0} I_2 = 0$.
  The proof of $\int_a^b g \rmd \psi = \int_a^b fg \rmd \phi$ is thus completed.

\end{exercise}

\begin{exercise}{2.16}
  In view of Corollary 2.7 (Jordan's Theorem),
  we assume that $\phi$ is monotone increasing.
  Suppose $\sup f = M$ and $\{y_j\}_1^L$ are the jump discontinuities.
  By the continuity of $\phi$ at $\{y_j\}$,
  for each $j$ there exists a closed interval $I_j$ containing $y_j$ such that
  \begin{equation}
    \left| \phi(y) - \phi(z) \right| < \frac{\epsilon}{4 M L}
    \label{eq:estOfPhi}
  \end{equation}
  for any $y, z \in I_j$.
  By the uniform continuity of $f$ on $[a,b] \setminus \bigcup I_j^\circ$,
  select $\delta'$ so small that
  \begin{equation}
    \left| f(y) - f(z) \right| < \frac{\epsilon}{2 V(\phi)}
    \label{eq:estOfF}
  \end{equation}
  for $y,z \in [a,b] \setminus \bigcup I_j^\circ$ and $|y-z| < \delta'$.
  Now suppose $\Gamma = \{x_k\}$ is a partition of $[a,b]$ satisfying
  $\left| \Gamma \right| < \min \left\{ \delta', |I_1|, \cdots, |I_L| \right\}$.
  Divide the closed intervals in $\Gamma$ into two subcollections:
  Let  $I \in \Gamma_1$
  if $I$ has nonempty intersection with $\{y_j\}$
  and let $I \in \Gamma_2$ otherwise.
  We estimate the difference between the upper and lower Riemann-Stieltjes sums as follows.
  First by \eqref{eq:estOfPhi} we have
  \[
    \left| \sum_{[x_{k-1}, x_k] \in I_1}
    \left( M_k - m_k \right) \left( \phi(x_k) - \phi(x_{k-1}) \right) \right|
    \le 2M \cdot L \cdot \frac{\epsilon}{4ML}
    \le \frac{\epsilon}{2}
  \]
  since there are at most $L$ intervals in the subcollection $\Gamma_1$.
  Second by \eqref{eq:estOfF} we have
  \[
    \left| \sum_{[x_{k-1}, x_k] \in I_2}
    \left( M_k - m_k \right) \left( \phi(x_k) - \phi(x_{k-1}) \right) \right|
    \le \frac{\epsilon}{2V(\phi)} \cdot V(\phi)
    \le \frac{\epsilon}{2}.
  \]
  We have shown $\lim_{|\Gamma| \rightarrow 0} (U_\Gamma - L_\Gamma) = 0$.
  It follows from $L_\Gamma \le R_\Gamma \le U_\Gamma$ that
  the integral $\int_a^b f \rmd \phi = \lim R_\Gamma$ exists.
\end{exercise}

\begin{exercise}{2.17}
  In view of Corollary 2.7 (Jordan's Theorem),
  we assume that $\phi$ is monotone increasing.
  Set $S_n = \int_{-n}^n f \rmd \phi$.
  We show that the sequence $\{S_n\}$ is Cauchy.
  Let $\phi(-\infty) = M^-$ and $\phi(+\infty) = M^+$.
  Since $\phi$ is monotone increasing,
  there exists $N_1 > 0$ so large that
  $\phi(x) > M^+ - \sqrt{\epsilon} / 2$ for all $x > N_1$
  and $\phi(x) < M^- + \sqrt{\epsilon} / 2$ for all $x < -N_1$.
  Since $f$ is continuous and $f(\infty) = 0$,
  there exists $N_2 > 0$ so large that
  $|f(x)| < \sqrt{\epsilon}$ for all $|x| > N_2$.
  Now if $n \ge m > \max\{N_1, N_2\}$, we have
  \[
    \begin{aligned}
      \left| S_n - S_m \right| =
      & \left| \int_{-n}^{-m} f \rmd \phi + \int_m^n f \rmd \phi \right| \\
      \le & \left( \sup_{|x| \ge m} |f| \right)
      \left( V(\phi;[-n,-m]) + V(\phi;[m,n]) \right) \\
      \le & \sqrt{\epsilon}
      \left( \phi(-m) - \phi(-\infty) + \phi(+\infty) - \phi(m) \right) \\
      < & \sqrt{\epsilon}
      \left( \frac{\sqrt{\epsilon}}{2} + \frac{\sqrt{\epsilon}}{2} \right) \\
      = &\epsilon,
    \end{aligned}
  \]
  where the first inequality follows from Theorem 2.24.
  Thus the improper integral $\int_{-\infty}^{+\infty} f \rmd \phi = \lim S_n$ exists.
\end{exercise}

\begin{exercise}{2.31}
  For the first part,
  note that the Riemann-Stieltjes sum
  \[
    R_\Gamma(\rmd f; [a,b]) = \sum f(x_j) - f(x_{j-1}) = f(b) - f(a)
  \]
  is independent of the partition $\Gamma$.
  For the second part, we have by the Mean-Value Theorem
  \[
    \sum f(x_j) - f(x_{j-1}) = \sum f'(\xi_j) \left( x_j - x_{j-1} \right)
  \]
  with $\xi_j \in (x_{j-1}, x_j)$.
  Since $f'$ is Riemann integrable,
  by taking the limit $|\Gamma| \rightarrow 0$ of both sides we obtain
  $\int_a^b \rmd f = \int_a^b f' \rmd x$.
\end{exercise}

\begin{exercise}{2.32}
  Since $af(a)$ and $bf(b)$ are finite,
  it follows from Theorem 2.21 (integration by parts) that
  the (Riemann) integrability of $f \rmd x$
  is equivalent to that of $x \rmd f$.
  But $x$ is continuous and $f$ is of bounded variation on $[a,b]$,
  the integrability follows from Theorem 2.24.
\end{exercise}
