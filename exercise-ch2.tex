
\begin{exercise}{2.14}
  This is the example following (2.28).
  Take $[a,b] = [-1, 1], c=0$, and
  \[
    f(x) = \left\{
    \begin{aligned}
      0 & \quad \text{if} \quad -1 \le x < 0 \\
      1 & \quad \text{if} \quad 0 \le x \le 1,
    \end{aligned}
    \right.
    \quad
    \phi(x) = \left\{
    \begin{aligned}
      0 & \quad \text{if} \quad -1 \le x \le 0 \\
      1 & \quad \text{if} \quad 0 < x \le 1.
    \end{aligned}
    \right.
  \]
  $R_\Gamma(f\rmd \phi; [-1, 1])$ can be 0 or 1 for any $\Gamma$ straddling $0$,
  so $\int_{-1}^1 f\rmd \phi$ does not exist.
  Since $\phi \equiv 0$ on $[-1, 0]$,
  $R_\Gamma(f \rmd \phi; [-1, 0])$ equals to zero for any $\Gamma$
  and hence $\int_{-1}^0 f \rmd \phi = 0$.
  Finally, if $\Gamma = \{x_j\}$ is a partion of $[0, 1]$, then
  \[
    R_\Gamma(f \rmd \phi; [0, 1]) = f(\xi_1) \left( \phi(x_1) - \phi(x_0) \right)
    = 1 \cdot (1-0) = 1.
  \]
  Hence $\int_0^1 f \rmd \phi = 1$.
\end{exercise}

\begin{exercise}{2.15}
  Let $\Gamma = \{x_j\}_1^N$ be a partition of $[a,b]$. Then
  \[
    S_\Gamma = \sum \left\vert \psi(x_j) - \psi(x_{j-1}) \right\vert
    = \sum \left\vert \int_{x_{j-1}}^{x_j} f \rmd \phi \right\vert
    \le \left(\sup |f|\right) \sum V(\phi; [x_{j-1}, x_j])
    \le \left(\sup |f|\right) V(\phi; [a,b]),
  \]
  where the first inequality follows from Theorem 2.24.
  This holds for all $\Gamma$ and thus $\psi$ is of bounded variation.
  If $g$ is continuous on $[a,b]$,
  both $\int_a^b g \rmd \psi$ and $\int_a^b fg \rmd \phi$ exist
  by Theorem 2.24.
  Consider the Riemann-Stieltjes sum for $g \rmd \psi$, we have
  \begin{align*}
    & \sum g(\xi_j) \left( \psi(x_j) - \psi(x_{j-1}) \right) \\
    = & \sum g(\xi_j) f(\eta_j) \left( \phi(x_j) - \phi(x_{j-1}) \right) \\
    = & \sum g(\xi_j) f(\xi_j) \left( \phi(x_j) - \phi(x_{j-1}) \right)
    + \sum g(\xi_j) \left( f(\eta_j) - f(\xi_j) \right)
    \left( \phi(x_j) - \phi(x_{j-1}) \right) \\
    \coloneqq & I_1 + I_2,
  \end{align*}
  where the first equality follows from Theorem 2.27 (Mean-Value Theorem)
  with each $\eta_j \in [x_{j-1}, x_j]$.
  The term $I_1$ is the Riemann-Stieltjes sum for $gf \rmd \phi$
  and tends to $\int_a^b fg \rmd \phi$ as $|\Gamma| \rightarrow 0$.
  Using the uniform continuity of $f$ on $[a,b]$,
  we can select $|\Gamma|$ so small that
  \[
    \left\vert f(\eta_j) - f(\xi_j) \right\vert <
    \frac{\epsilon} {\left( \sup g \right) V(\phi)}
  \]
  for every $1 \le j \le N$.
  It follows that $|I_2| < \epsilon$
  and $\lim_{|\Gamma| \rightarrow 0} I_2 = 0$.
  The proof of $\int_a^b g \rmd \psi = \int_a^b fg \rmd \phi$ is thus completed.
\end{exercise}
