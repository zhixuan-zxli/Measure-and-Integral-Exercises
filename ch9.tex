\begin{lemma}[Generalized H\"{o}lder's Inquality]
  Let $1 \le p_i, r \le \infty$ and
  $\sum_{i=1}^k \frac{1}{p_i} = \frac{1}{r}$.
  Then
  \[
    \left\| f_1 \cdots f_k \right\|_r \le
    \left\| f_1 \right\|_{p_1} \cdots \left\| f_k \right\|_{p_k}.
  \]
\end{lemma}
\begin{proof}
  We first prove the case $k=2$.
  If $p_1 = \infty$, then $p_2 = r$,
  $|f_1 f_2| \le \|f_1\|_{\infty} |f_2|$,
  and by homogeneity $\|f_1 f_2\|_r \le \|f_1\|_{\infty} \|f_2\|_{p_2}$.
  The case $p_2 = \infty$ is similar.
  If $p_1, p_2 < \infty$, then $r < \infty$ and $1 \le p_i/r < \infty$.
  Applying H\"{o}lder's inequality with exponents $p_i/r$,
  we obtain
  \[
    \left\| f_1 f_2 \right\|_r^r
    \le \int \left| f_1 f_2 \right|^r
    \le \left( \int \left| f_1 \right|^{p_1} \right)^{r/p_1}
    \left( \int \left| f_2 \right|^{p_2} \right)^{r/p_2}
    = \left\| f_1 \right\|_{p_1}^r \left\| f_2 \right\|_{p_2}^r.
  \]
  Taking the $r$th root and we are done with $k=2$.
  Now assume the conclusion holds for $k-1$ terms.
  Let $\sum_{i=2}^{k} \frac{1}{p_i} = \frac{1}{r'}$.
  Then we have
  \[
    \begin{aligned}
      \left\| f_1 \cdots f_k \right\|_r
      &\le \left\| f_1 \right\|_{p_1} \left\| f_2 \cdots f_k \right\|_{r'}
      \quad &&(\text{induction hypothesis for }
      \frac{1}{p_1} + \frac{1}{r'} = \frac{1}{r}) \\
      &\le \left\| f_1 \right\|_{p_1}
      \left\| f_2 \right\|_{p_2} \cdots \left\| f_k \right\|_{p_k}
      \quad &&(\text{induction hypothesis for }
      \sum_{i=2}^{k} \frac{1}{p_i} = \frac{1}{r'}) \\
      &= \left\| f_1 \right\|_{p_1} \cdots \left\| f_k \right\|_{p_k},
    \end{aligned}
  \]
  and the proof of the Lemma is completed.
\end{proof}

\begin{exercise}{9.2}
  (a) The case $r = \infty$ reduces to H\"{o}lder's inequality.
  Indeed, we have $\frac{1}{p} + \frac{1}{q} = 1$ and
  \[
    \left| (f * g)(\bfx) \right| \le \int \left| f(\bft) g(\bfx-\bft) \right| \rmd t
    \le \left\| f \right\|_p \left\| g(x-\cdot) \right\|_q
    = \left\| f \right\|_p \left\| g \right\|_q
  \]
  for each $\bfx \in \bbR^n$, so $\| f*g \|_{\infty} \le \|f\|_p \|g\|_q$.

  If $r < \infty$,
  then $\frac{1}{p} + \frac{1}{q} = 1 + \frac{1}{r} > 1$
  and also $p, q < \infty$.
  We may assume $f, g \ge 0$ since
  $\left| (f*g)(\bfx) \right| \le \left( |f|*|g| \right) (\bfx)$
  and we can apply the special case to $|f|$ and $|g|$.
  We write
  \[
    (f*g)(\bfx) = \int f(\bft) g(\bfx-\bft) \rmd \bft
    = \int f(\bft)^{\frac{p}{r}} g(\bfx-\bft)^{\frac{q}{r}}
    \cdot f(\bft)^{p(\frac1p - \frac1r)}
    \cdot g(\bfx - \bft)^{q(\frac1q - \frac1r)} \rmd \bft.
  \]
%  Applying the H\"{o}lder's inequality for three functions
  Applying the Lemma
  with exponents
  $\frac{1}{r} + (\frac{1}{p} - \frac{1}{r})
  + (\frac{1}{q} - \frac{1}{r}) = 1$,
  then raising to the $r$th power,
  we find
  \[
    \begin{aligned}
    (f*g)(\bfx)^r
    &\le
    \left( \int f(\bft)^p g(\bfx-\bft)^q \rmd \bft \right)
    \left( \int f(\bft)^p \rmd \bft \right)^{\frac rp - 1}
    \left( \int g(\bfx-\bft)^q \rmd \bft \right)^{\frac rq - 1} \\
    &= \left\| f \right\|_p^{r-p} \left\| g \right\|_{q}^{r-q}
      \left( \int f(\bft)^p g(\bfx-\bft)^q \rmd \bft \right).
    \end{aligned}
  \]
  Integrating the term in parentheses with respect to $x$
  and applying Tonelli's Theorem, we find
  \[
    \int \int f(\bft)^p g(\bfx-\bft)^q \rmd \bft \rmd \bfx
    = \int f(\bft)^p \left( \int g(\bfx-\bft)^q \rmd \bfx \right) \rmd \bft
    = \left\| f \right\|_p^p \left\| g \right\|_q^q.
  \]
  It follows that
  \[
    \left\| f*g \right\|_r^r \le
    \left\| f \right\|_p^{r-p} \left\| g \right\|_{q}^{r-q}
    \left\| f \right\|_p^p \left\| g \right\|_q^q
    = \left\| f \right\|_p^r \left\| g \right\|_q^r.
  \]
  Taking the $r$th root we obtain the desired inequality
  $\left\| f*g \right\|_r \le \left\| f \right\|_p \left\| g \right\|_q$.

  (b) Define $f_{\lambda}(\bfx) = f(\lambda \bfx), \lambda > 0$.
  Then
  \[
    \left\| f_\lambda \right\|_p
    = \left( \int \left|f(\lambda \bfx)\right|^p \rmd \bfx \right)^{\frac 1p}
    = \left( \lambda^{-n} \int \left|f(\bfx)\right|^p \rmd \bfx \right)^{\frac 1p}
    = \lambda^{-\frac{n}{p}} \left\| f \right\|_p.
  \]
  Moreover, we have
  \[
    \begin{aligned}
      \left\| f_{\lambda}*g_{\lambda} \right\|_r
      &= \left( \int \left|
      \int f(\lambda\bft) g(\lambda\bfx-\lambda\bft) \rmd \bft
      \right|^r \rmd \bfx \right)^{\frac 1r} \\
      &= \left( \int \left|
      \lambda^{-n} \int f(\bft) g(\lambda\bfx-\bft) \rmd \bft
      \right|^r \rmd \bfx \right)^{\frac 1r} \\
      &= \lambda^{-n} \left( \lambda^{-n} \int \left|
      \int f(\bft) g(\bfx-\bft) \rmd \bft
      \right|^r \rmd \bfx \right)^{\frac 1r} \\
      &= \lambda^{-n - \frac nr} \left\| f*g \right\|_r.
    \end{aligned}
  \]
  Applying Young's Convolution Theorem to $f_{\lambda} * g_{\lambda}$,
  we obtain
  \[
    \lambda^{-n - \frac nr} \left\| f*g \right\|_r
    \le \lambda^{-\frac n p} \left\| f \right\|_p
    \lambda^{-\frac n q} \left\| g \right\|_q
    \Longleftrightarrow
    \left\| f*g \right\|_r \le
    \lambda^{-n\left( 1 + \frac1r - \frac 1p - \frac 1q \right)}
    \left\| f \right\|_p \left\| g \right\|_q.
  \]
  If $-n\left( 1 + \frac1r - \frac 1p - \frac 1q \right) > 0$
  (or $<0$, resp.),
  we can let $\lambda \rightarrow 0$
  (or $\rightarrow +\infty$, resp.)
  so that $\left\| f*g \right\|_r = 0$ for all $f \in L^p$ and $g \in L^q$.
  But this must be false, so we have
  \[
    \frac{1}{r} = \frac{1}{p} + \frac{1}{q} - 1.
    \qedhere
  \]
\end{exercise}

\begin{exercise}{9.3}
  (a) By H\"{o}lder's inequality, we have
  \[
    \left| (f*K)(\bfx) \right|
    \le \int \left| f(\bft) K(\bfx - \bft) \right| \rmd \bft
    \le \left\| f \right\|_p \left\| K \right\|_{p'},
  \]
  so $f*K$ is bounded in $\bbR^n$.
  If $1\le p' < \infty$ and $\bfx_1, \bfx_2 \in \bbR^n$,
  again by H\"{o}lder's inequality we have
  \[
    \begin{aligned}
      \left| (f*K)(\bfx_2) - (f*K)(\bfx_1) \right|
      &\le \int \left| f(\bft) \right| \cdot
      \left| K(\bfx_2 - \bft) - K(\bfx_1 - \bft)\right|
      \rmd \bft \\
      &\le \left\| f \right\|_p
      \left\| K(\bfx_2 - \cdot) - K(\bfx_1 - \cdot) \right\|_{p'}.
    \end{aligned}
  \]
  As $\bfx_2 \rightarrow \bfx_1$,
  the last norm term tends to zero
  due to continuity in $L^{p'}$ norm ($1 \le p' < \infty$).
  Thus  $f*K$ is continuous.
  If $p' = \infty$, then $p < \infty$,
  and the continuity follows from
  \[
    (f*K)(x) = \int f(\bft) K(\bfx - \bft) \rmd \bft
    = \int f(\bfx - \bft) K(\bft) \rmd \bft
    = (K*f)(x).
  \]

  (b)
%  Note that
%  \[
%    (\chi_I * \chi_J) (x) = \int \chi_I(t) \chi_J(x-t) \rmd t
%    = \mea{I \cap (x-J)}.
%  \]
  For $I = [a_1, b_1], J = [a_2, b_2]$,
  we sketch the graph of $\chi_I * \chi_J$ in Figure \ref{fig:graphOfChiIChiJ}.
  \begin{figure}[htbp]
    \centering
    \begin{subfigure}[b]{.4\linewidth}
      \includegraphics[height=0.4\linewidth]{pst/convGraph_1}
      \caption{Case $\mea{I} \ge \mea{J}$. }
    \end{subfigure}
    \begin{subfigure}[b]{.4\linewidth}
      \includegraphics[height=0.4\linewidth]{pst/convGraph_2}
      \caption{Case $\mea{I} < \mea{J}$. }
    \end{subfigure}%

    \begin{subfigure}[b]{.4\linewidth}
      \includegraphics[height=0.4\linewidth]{pst/convGraph_3}
      \caption{Case $I = J$. }
    \end{subfigure}%
    \caption{Graph of $\chi_I * \chi_J$. }
    \label{fig:graphOfChiIChiJ}
  \end{figure}
\end{exercise}
