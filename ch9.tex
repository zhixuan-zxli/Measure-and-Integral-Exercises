\section{Approximations of the Identity and Maximal Functions}
\begin{lemma}[Generalized H\"{o}lder's Inquality]
  Let $1 \le p_i, r \le \infty$ and
  $\sum_{i=1}^k \frac{1}{p_i} = \frac{1}{r}$.
  Then
  \[
    \left\| f_1 \cdots f_k \right\|_r \le
    \left\| f_1 \right\|_{p_1} \cdots \left\| f_k \right\|_{p_k}.
  \]
\end{lemma}
\begin{proof}
  We first prove the case $k=2$.
  If $p_1 = \infty$, then $p_2 = r$,
  $|f_1 f_2| \le \|f_1\|_{\infty} |f_2|$ a.e.,
  and by homogeneity $\|f_1 f_2\|_r \le \|f_1\|_{\infty} \|f_2\|_{p_2}$.
  The case $p_2 = \infty$ is similar.
  If $p_1, p_2 < \infty$, then $r < \infty$ and $1 \le p_i/r < \infty$.
  Applying H\"{o}lder's inequality with exponents $p_i/r$,
  we obtain
  \[
    \left\| f_1 f_2 \right\|_r^r
    \le \int \left| f_1 f_2 \right|^r
    \le \left( \int \left| f_1 \right|^{p_1} \right)^{r/p_1}
    \left( \int \left| f_2 \right|^{p_2} \right)^{r/p_2}
    = \left\| f_1 \right\|_{p_1}^r \left\| f_2 \right\|_{p_2}^r.
  \]
  Taking the $r$th root and we are done with $k=2$.
  Now assume the conclusion holds for $k-1$ terms.
  Let $\sum_{i=2}^{k} \frac{1}{p_i} = \frac{1}{r'}$.
  Then we have
  \[
    \begin{aligned}
      \left\| f_1 \cdots f_k \right\|_r
      &\le \left\| f_1 \right\|_{p_1} \left\| f_2 \cdots f_k \right\|_{r'}
      \quad &&(\text{induction hypothesis for }
      \frac{1}{p_1} + \frac{1}{r'} = \frac{1}{r}) \\
      &\le \left\| f_1 \right\|_{p_1}
      \left\| f_2 \right\|_{p_2} \cdots \left\| f_k \right\|_{p_k}
      \quad &&(\text{induction hypothesis for }
      \sum_{i=2}^{k} \frac{1}{p_i} = \frac{1}{r'}) \\
      &= \left\| f_1 \right\|_{p_1} \cdots \left\| f_k \right\|_{p_k},
    \end{aligned}
  \]
  and the proof of the Lemma is completed.
\end{proof}


\begin{exercise}{9.2}
  (a)
  First assume $f, g \ge 0$.
  Then $f * g$ is measurable by Tonelli's Theorem.
  Consider the case $r = \infty$ where we have $\frac{1}{p} + \frac{1}{q} = 1$.
  Then we find
  \[
    \left| (f * g)(\bfx) \right| \le \int \left| f(\bft) g(\bfx-\bft) \right| \rmd t
    \le \left\| f \right\|_p \left\| g(\bfx-\cdot) \right\|_q
    = \left\| f \right\|_p \left\| g \right\|_q
  \]
  for each $\bfx \in \bbR^n$, so $\| f*g \|_{\infty} \le \|f\|_p \|g\|_q$.
  Next consider the case $r < \infty$,
  where $\frac{1}{p} + \frac{1}{q} = 1 + \frac{1}{r} > 1$
  and $p, q < \infty$.
  We write
  \[
    (f*g)(\bfx) = \int f(\bft) g(\bfx-\bft) \rmd \bft
    = \int f(\bft)^{\frac{p}{r}} g(\bfx-\bft)^{\frac{q}{r}}
    \cdot f(\bft)^{p(\frac1p - \frac1r)}
    \cdot g(\bfx - \bft)^{q(\frac1q - \frac1r)} \rmd \bft.
  \]
%  Applying the H\"{o}lder's inequality for three functions
  Applying the Lemma
  with exponents
  $\frac{1}{r} + (\frac{1}{p} - \frac{1}{r})
  + (\frac{1}{q} - \frac{1}{r}) = 1$,
  then raising to the $r$th power,
  we find
  \[
    \begin{aligned}
    (f*g)(\bfx)^r
    &\le
    \left( \int f(\bft)^p g(\bfx-\bft)^q \rmd \bft \right)
    \left( \int f(\bft)^p \rmd \bft \right)^{\frac rp - 1}
    \left( \int g(\bfx-\bft)^q \rmd \bft \right)^{\frac rq - 1} \\
    &= \left\| f \right\|_p^{r-p} \left\| g \right\|_{q}^{r-q}
      \left( \int f(\bft)^p g(\bfx-\bft)^q \rmd \bft \right).
    \end{aligned}
  \]
  Integrating the term in parentheses with respect to $\bfx$
  and applying Tonelli's Theorem, we obtain
  \[
    \int \int f(\bft)^p g(\bfx-\bft)^q \rmd \bft \rmd \bfx
    = \int f(\bft)^p \left( \int g(\bfx-\bft)^q \rmd \bfx \right) \rmd \bft
    = \left\| f \right\|_p^p \left\| g \right\|_q^q.
  \]
  It follows that
  \[
    \left\| f*g \right\|_r^r \le
    \left\| f \right\|_p^{r-p} \left\| g \right\|_{q}^{r-q}
    \left\| f \right\|_p^p \left\| g \right\|_q^q
    = \left\| f \right\|_p^r \left\| g \right\|_q^r.
  \]
  Taking the $r$th root we obtain the desired inequality
  $\left\| f*g \right\|_r \le \left\| f \right\|_p \left\| g \right\|_q$.

  For general $f \in L^p$ and $g \in L^q$,
  we show that $f * g$ is measurable.
  By the previous case we have $|f| * |g| \in L^r$.
  Therefore $|f| * |g| < \infty$ a.e.,
  $f(\bft) g(\bfx - \bft) \in L^1(\rmd \bft)$ for a.e.\ $\bfx$,
  and $f*g$ exists and is finite a.e.
  Next, define
  \[
    f_N = f \chi_{\{|\bfx| < N\}}, \quad
    g_N = g \chi_{\{|\bfx| < N\}}, \quad
    N = 1, 2, \cdots.
  \]
  Then $f_N, g_N \in L^1$,
  and $f_N * g_N$ is measurable by Fubini's Theorem.
  Note that
  \[
    \lim_{N \rightarrow \infty} f_N(\bft) g_N(\bfx - \bft)
    = f(\bft) g(\bfx - \bft)
    \text{ and }
    |f_N(\bft) g_N(\bfx - \bft)| \le |f(\bft) g(\bfx-\bft)|.
  \]
  By Dominated Convergence Theorem
  we have $\lim_{N \rightarrow \infty} (f_N * g_N) = f * g$ a.e.,
  so we know that $f * g$ is measurable.
  Finally, the inequality for general $f$ and $g$
  follows by noting that
  $\left| (f*g)(\bfx) \right| \le \left( |f|*|g| \right) (\bfx)$
  and applying the previous case to $|f|$ and $|g|$.

  (b) Define $f_{\lambda}(\bfx) = f(\lambda \bfx), \lambda > 0$.
  Then
  \[
    \left\| f_\lambda \right\|_p
    = \left( \int \left|f(\lambda \bfx)\right|^p \rmd \bfx \right)^{\frac 1p}
    = \left( \lambda^{-n} \int \left|f(\bfx)\right|^p \rmd \bfx \right)^{\frac 1p}
    = \lambda^{-\frac{n}{p}} \left\| f \right\|_p.
  \]
  Moreover, we have
  \[
    \begin{aligned}
      \left\| f_{\lambda}*g_{\lambda} \right\|_r
      &= \left( \int \left|
      \int f(\lambda\bft) g(\lambda\bfx-\lambda\bft) \rmd \bft
      \right|^r \rmd \bfx \right)^{\frac 1r} \\
      &= \left( \int \left|
      \lambda^{-n} \int f(\bft) g(\lambda\bfx-\bft) \rmd \bft
      \right|^r \rmd \bfx \right)^{\frac 1r} \\
      &= \lambda^{-n} \left( \lambda^{-n} \int \left|
      \int f(\bft) g(\bfx-\bft) \rmd \bft
      \right|^r \rmd \bfx \right)^{\frac 1r} \\
      &= \lambda^{-n - \frac nr} \left\| f*g \right\|_r.
    \end{aligned}
  \]
  Applying Young's Convolution Theorem to $f_{\lambda} * g_{\lambda}$,
  we obtain
  \[
    \lambda^{-n - \frac nr} \left\| f*g \right\|_r
    \le \lambda^{-\frac n p} \left\| f \right\|_p
    \lambda^{-\frac n q} \left\| g \right\|_q
    \Longleftrightarrow
    \left\| f*g \right\|_r \le
    \lambda^{n\left( 1 + \frac1r - \frac 1p - \frac 1q \right)}
    \left\| f \right\|_p \left\| g \right\|_q.
  \]
  If $n\left( 1 + \frac1r - \frac 1p - \frac 1q \right) > 0$
  (or $<0$, resp.),
  we can let $\lambda \rightarrow 0$
  (or $\rightarrow +\infty$, resp.)
  so that $\left\| f*g \right\|_r = 0$ for all $f \in L^p$ and $g \in L^q$.
  But this must be false, so we have
  \[
    \frac{1}{r} = \frac{1}{p} + \frac{1}{q} - 1.
    \qedhere
  \]
\end{exercise}

\begin{exercise}{9.3}
  (a) By H\"{o}lder's inequality, we have
  \[
    \left| (f*K)(\bfx) \right|
    \le \int \left| f(\bft) K(\bfx - \bft) \right| \rmd \bft
    \le \left\| f \right\|_p \left\| K \right\|_{p'},
  \]
  so $f*K$ is bounded in $\bbR^n$.
  If $1\le p' < \infty$ and $\bfx_1, \bfx_2 \in \bbR^n$,
  again by H\"{o}lder's inequality we have
  \[
    \begin{aligned}
      \left| (f*K)(\bfx_2) - (f*K)(\bfx_1) \right|
      &\le \int \left| f(\bft) \right| \cdot
      \left| K(\bfx_2 - \bft) - K(\bfx_1 - \bft)\right|
      \rmd \bft \\
      &\le \left\| f \right\|_p
      \left\| K(\bfx_2 - \cdot) - K(\bfx_1 - \cdot) \right\|_{p'}.
    \end{aligned}
  \]
  As $\bfx_2 \rightarrow \bfx_1$,
  the last norm term tends to zero
  due to continuity in $L^{p'}$ norm ($1 \le p' < \infty$).
  Thus  $f*K$ is continuous.
  If $p' = \infty$, then $p < \infty$,
  and the continuity follows from
  \[
    (f*K)(x) = \int f(\bft) K(\bfx - \bft) \rmd \bft
    = \int f(\bfx - \bft) K(\bft) \rmd \bft
    = (K*f)(x).
  \]

  (b)
%  Note that
%  \[
%    (\chi_I * \chi_J) (x) = \int \chi_I(t) \chi_J(x-t) \rmd t
%    = \mea{I \cap (x-J)}.
%  \]
  For $I = [a_1, b_1], J = [a_2, b_2]$,
  we sketch the graph of $\chi_I * \chi_J$ in Figure \ref{fig:graphOfChiIChiJ}.
  \begin{figure}[htbp]
    \centering
    \begin{subfigure}[b]{.4\linewidth}
      \includegraphics[height=0.4\linewidth]{pst/convGraph_1}
      \caption{Case $\mea{I} \ge \mea{J}$. }
    \end{subfigure}
    \begin{subfigure}[b]{.4\linewidth}
      \includegraphics[height=0.4\linewidth]{pst/convGraph_2}
      \caption{Case $\mea{I} < \mea{J}$. }
    \end{subfigure}%

    \begin{subfigure}[b]{.4\linewidth}
      \includegraphics[height=0.4\linewidth]{pst/convGraph_3}
      \caption{Case $I = J$. }
    \end{subfigure}%
    \caption{Graph of $\chi_I * \chi_J$. }
    \label{fig:graphOfChiIChiJ}
  \end{figure}
\end{exercise}
\begin{exercise}{9.4}\label{94}
  (a) \textbf{Claim}: For any positive integer $m$, the derivative of order $m$ of $h$ has the form $$h^{(m)}(x)=\sum_{i=1}^{N(m)}c_i(m)\frac{1}{x^i}e^{-1/x^2}\quad \text{for}\ x>0,$$
  where $N(m)$ is an integer and $c_i(m)$ are constants depend on $m$.
  
  We will prove the claim by induction on $m$. For $m=1$, we have $h^\prime(x)=2\frac{1}{x^3}e^{-1/x^2}$ for $x>0$. Assuming that the formula holds for $m$, we have
  $$h^{(m+1)}=\frac{\rmd}{\rmd x}\sum_{i=1}^{N(m)}c_i(m)\frac{1}{x^i}e^{-1/x^2}=\sum_{i=1}^{N(m)}c_i(m)\left(\frac{2}{x^{i+3}}-\frac{i}{x^{i+1}}\right)e^{-1/x^2}\quad \text{for}\ x>0.$$
  This proves the claim.
  
 Note that $h^{(m)}(x)=0$ for $x<0$ for any positive integer $m$. It is obvious that $h$ is continuous on $\mathbf{R}$ and is infinitely differentiable on $\mathbf{R}\backslash\{0\}$. Therefore, to prove $h\in C^\infty$, we only need to show that $h$ is infinitely differentiable at $x=0$. We will prove $h\in C^m$ and $h^{(m)}(0)=0$ for any positive integer $m$ by induction. For $m=1$, we have
  \begin{equation*}
      \begin{aligned}
         h_+^{\prime}(0)&=\lim_{x\to 0+}\frac{h(x)-h(0)}{x-0}=\lim_{x\to 0+}\frac{e^{-1/x^2}}{x}=\lim_{t\to+\infty}\frac{t}{e^{t^2}}=0,\\
             h_-^{\prime}(0)&=\lim_{x\to 0-}\frac{h(x)-h(0)}{x-0}=\lim_{x\to 0-}\frac{0}{x}=0,
      \end{aligned}
  \end{equation*}
  which imply that $h\in C^1$ and $h^\prime(0)=0$. Suppose that this statement is true for $m$. Then for $m+1$, we have 
  \begin{equation*}
      \begin{aligned}
         h_+^{(m+1)}(0)&=\lim_{x\to 0+}\frac{h^{(m)}(x)-h^{(m)}(0)}{x-0}= \lim_{x\to 0+}\frac{\sum_{i=1}^{N(m)}c_i(m)\frac{1}{x^i}e^{-1/x^2}}{x}=\sum_{i=1}^{N(m)}c_i(m)\lim_{t\to+\infty}\frac{t^{i+1}}{e^{t^2}}=0,\\
          h_-^{(m+1)}(0)&=\lim_{x\to 0-}\frac{h^{(m)}(x)-h^{(m)}(0)}{x-0}=\lim_{x\to 0-}\frac{0}{x}=0,
      \end{aligned}
  \end{equation*}
  which imply that $h\in C^{(m+1)}$ and $h^{(m+1)}(0)=0$. This proves $h\in C^\infty$.
  
  (b) For $a<b$, we have $$g(x)=h(x-a)h(b-x)=e^{-\left[\frac{1}{(x-a)^2}+\frac{1}{(b-x)^2}\right]}\chi_{(a,b)}.$$ Thus $\operatorname{supp} g=\overline{\{x:g(x)\neq 0\}}=[a,b]$. Since $h(x)\in C^\infty$, it follows that the compositions $h(x-a),h(b-x)\in C^\infty$ and then the product $g(x)=h(x-a)h(b-x)\in C^\infty$.
  
  (c) For $\bfx=(x_1,\cdots,x_n)\in\bfrn$, we denote the Euclidean norm by $|\bfx|=(x_1^2+\cdots+x_n^2)^{1/2}$ and define 
  $$  \eta(\bfx)= \begin{cases}e^{\frac{1}{|\bfx|^2-1}} & \text { if } |\mathbf{x}|<1,  \\ 0 & \text { if } |\mathbf{x}|\geq 1.\end{cases}$$
  Then $\eta\in C_0^\infty(\bfrn)$ with support $\{\bfx:|\bfx|\leq1\}$.
  Given a closed ball $\bar{B}(\bfy,\epsilon)=\{\bfx:|\bfx-\bfy|\leq\epsilon\}$ of radius $\epsilon$ centred at $\bfy$. Let $\eta_{\epsilon,\bfy}(\bfx)=\eta\left(\frac{\bfx-\bfy}{\epsilon}\right)$, then $\eta_{\epsilon,\bfy}(x)\in C_0^\infty(\bfrn)$ with support $\bar{B}(\bfy,\epsilon)$. 
  
  Given an interval $I=\{\bfx=(x_1,\cdots,x_n): a_k\leq x_k\leq b_k, k=1,\cdots,n\}$, we define $$\rho(\bfx)=\prod_{k=1}^ng(x_k)=\prod_{k=1}^nh(x_k-a_k)h(b_k-x_k),$$
  where $h$ and $g$ are as defined in (a) and (b) respectively. Then $\rho\in  C_0^\infty(\bfrn)$ with support $I$.
\end{exercise}
\begin{exercise}{9.5} We will show that such function $h$ exists in three steps below.

  Step 1: We can choose an open set $G_2$ such that $\overline{G}_1\subset G_2$, $\overline{G}_2\subset G$.
  
  In fact, the boundary $\partial G=\overline{G}\backslash G=\overline{G}\cap G^c$ is compact and nonempty since $G$ is open and bounded. Similarly, $\partial G_1$ is also compact and nonempty.
   Note that $\overline{G}_1\subset G$, thus $\partial G_1\cap\partial G=\varnothing$ and then $d(\partial G,\partial G_1)=\inf\{|\bfx-\bfy|:\bfx\in\partial G,\bfy\in\partial G_1\}>0$. Let $\delta=d(\partial G,\partial G_1)/3$ and $G_2=\{\bfx:|\bfx-\bfy|<\delta\ \text{for some}\ \bfy\in G_1\}$, then $G_2$ is open and $\overline{G}_1\subset G_2\subset\overline{G}_2\subset G$.
   
   Step 2: we can find a $K\in C^\infty$ with $\operatorname{supp} K=\overline{B}_{\delta}=\{\bfx:|\bfx|\leq\delta\}$ and $\int K=1$.
   
   Recall from \textbf{Exercise} \ref{94} (c) that 
   $$K(\bfx)=\frac{C}{\delta^n}\eta\left(\frac{\bfx}{\delta}\right),$$ where $C=\left(\int \eta\right)^{-1}$,
   satisfies our requirements.
   
   Step 3: Let $h=\chi_{G_2}\ast K$, where $G_2$ and $K$ are as defined in Step 1 and Step 2 respectively, then $h\in C_0^\infty$ such that $h=1$ in $G_1$ and $h=0$ outside $G$.
   
   Since $G_2$ is bounded and $K\in C_0^\infty$, the function $h$ is $C_0^\infty$ with 
   $$\operatorname{supp} h\subset \operatorname{supp}\chi_{G_2}+\operatorname{supp}K=\overline{G}_2+\overline{B}_{\delta}\subset G.$$
   (Note that the validity of the last inclusion depends on the selection of $\delta$ in Step 1.) This shows that $h=0$ outside $G$. For $\bfx\in G_1$ and $\bft\notin G_2$, by the definition of $G_2$, we have $|\bfx-\bft|\geq \delta$. Since $\operatorname{supp}K=\overline{B}_\delta$, we have $K(\bfx-\bft)=0$. Therefore, for any $\bfx\in G_1$, it follows that $\chi_{G_2}(\bft)K(\bfx-\bft)=K(\bfx-\bft)$ and then 
   $$h(\bfx)=\int_{\bfrn}\chi_{G_2}(\bft)K(\bfx-\bft)\rmd\bft=\int_{\bfrn} K(\bfx-\bft)\rmd\bft=\int_{\bfrn} K(\bft)\rmd\bft=1.$$
\end{exercise}
\begin{exercise}{9.6}
  Suppose $|K|\leq M$ for some $M>0$. Then for any $\bfx\in\bbR^n$, we have
  \begin{align*}
      |f*K(\bfx)|&=\bigg|\int_{\bbR^n}f(\bft)K(\bfx-\bft)\rmd\bft\bigg|\\
      &\leq\int_{\bbR^n}|f(\bft)K(\bfx-\bft)|\rmd\bft\\
      &\leq M\int_{\bbR^n}|f(\bft)|\rmd\bft<\infty,
  \end{align*}
  which implies $f*K$ is bounded.
  
  Now for any given $\epsilon>0$, since $K$ is uniformly continuous, there exists $\delta>0$ such that as long as $|\bfx-\bfx'|<\delta$ we have $|K(\bfx)-K(\bfx')|<\epsilon.$ Then for such $\bfx$ and $\bfx'$ we have
  \begin{align*}
      |f*K(\bfx)-f*K(\bfx')|&=\bigg|\int_{\bbR^n}f(\bft)(K(\bfx-\bft)-K(\bfx'-\bft))\rmd\bft\bigg|\\
      &\leq \int_{\bbR^n}|f(\bft)(K(\bfx-\bft)-K(\bfx'-\bft))|\rmd\bft\\
      &\leq \epsilon\int_{\bbR^n}|f(\bft)|\rmd\bft.
  \end{align*}
  Thus $f*K$ is uniformly continuous.
\end{exercise}

\begin{exercise}{9.7}
  Let $g(x,y,t)=f(t)P_y(x-t)$, then 
  \[
  \frac{\partial g}{\partial x}(x,y,t)=f(t)\frac{-2y(x-t)}{[y^2+(x-t)^2]^2}.
  \]
  When $|x|\leq N$, we then have 
  \[
    \bigg|\frac{\partial g}{\partial x}(x,y,t)\bigg|\leq |f(t)|\frac{2y(t+N)}{y^4+(|t|-N)^4}.
  \]
  Denote the right hand side function by $h(y,t)$. By Holder's Inequality
  \[
    \int_\bbR h(y,t)\rmd t\leq \pnorm{f}\bigg(\int_\bbR\bigg[\frac{2y(t+N)}{y^4+(|t|-N)^4}\bigg]^{p'}\rmd t\bigg)^{1/{p'}}<\infty.
  \]
  Thus $h(y,t)\in L^1(\bbR)$. Now for fixed $x$ and $y$, choose an arbitrary sequence $\{h_n\}$ converging to zero and we may assume $|h_i|\leq 1$ for all $i$. We may also assume $|x|\leq N-1$ for some $N$ sufficiently large. Then by mean value theorem
  \[
    \frac{g(x+h_n,y,t)-g(x,y,t)}{h_n}=\frac{\partial g}{\partial x}(x+\xi_n,y,t)
  \]
  for some $\xi_n$ between 0 and $h_n$. Hence by earlier result we have
  \[
    \bigg|\frac{g(x+h_n,y,t)-g(x,y,t)}{h_n}\bigg|\leq h(y,t).
  \]
  Now by dominated convergence theorem, we have
  \begin{align*}
      \lim_{n\rightarrow\infty}\frac{f(x+h_n,y,t)-f(x,y,t)}{h_n}&=\lim_{n\rightarrow\infty}\int_\bbR\frac{g(x+h_n,y,t)-g(x,y,t)}{h_n}\rmd t\\
      &=\int_\bbR \frac{\partial g}{\partial x}(x,y,t)\rmd t=\int_\bbR f(t)\frac{\partial P_y}{\partial x}(x-t)\rmd t.
  \end{align*}
  Therefore
  \[
    \frac{\partial f}{\partial x}(x,y)=\int_\bbR f(t)\frac{\partial P_y}{\partial x}(x-t)\rmd t.
  \]
  Using similar methods, we can calculate that 
  \[
    (\frac{\partial^2}{\partial x^2}+\frac{\partial^2}{\partial y^2})f(x,y)=\int_\bbR f(t)(\frac{\partial^2}{\partial x^2}+\frac{\partial^2}{\partial y^2})P_y(x-t)\rmd t
  \]
  and since $P_y(x)$ is harmonic when $y>0$, we deduce that $f(x,y)$ is harmonic when $y>0.$
\end{exercise}

\begin{exercise}{9.8}
  For any $s> 0$, since $K(s,t)=s^{-1}K(1,t/s)$, we have
  \[
    (Tf)(s)=\int_0^\infty f(t)s^{-1}K(1,t/s)\rmd t=\int_0^\infty f(st)K(1,t)\rmd t.
  \]
  Using results in Exercise 8.8, it follows that
  \begin{align*}
     \pnorm{Tf}&=\bigg[\int_0^\infty\bigg[\int_0^\infty f(st)K(1,t)\rmd t\bigg]^p\rmd s\bigg]^{1/p}\\
     &\leq \int_0^\infty\bigg[\int_0^\infty f(st)^p K(1,t)^p\rmd s\bigg]^{1/p}\rmd t\\
     &=\int_0^\infty\bigg[\int_0^\infty f(s)^p t^{-1}K(1,t)^p\rmd s\bigg]^{1/p}\rmd t\\
     &=\bigg[\int_0^\infty f(s)^p\rmd s\bigg]^{1/p}\bigg[\int_0^\infty t^{-1/p}K(1,t)\rmd t\bigg]\\
     &\leq\gamma\pnorm{f}.
     \qedhere
  \end{align*}
\end{exercise}

\begin{exercise}{9.13}
  For each $x \in (0, 1)$,
  if we let $I_k(x)$ be the unique interval
  among $I_{k,j}$ containing $x$,
  then $\{I_k(x)\}$ shrinks regularly to $x$.
  It follows from Lebesgue Differentiation Theorem
  that $f_k \rightarrow f$ a.e.
  By Exercise 8.12, to show $f_k \rightarrow f$ in $L^p(0,1)$,
  it suffices to show $\|f_k\|_p \rightarrow \|f\|_p$.
  To this end, we compute
  \[
    \begin{aligned}
      \int_0^1 \left|f_k\right|^p
      &= \sum_{j=1}^{2^k} \int_{I_{k,j}} \left|f_k\right|^p \\
      &= \sum_{j=1}^{2^k} 2^{-k} \left| 2^k \int_{I_{k,j}} f \right|^p \\
      &\le 2^{k(p-1)} \sum_{j=1}^{2^k}
      \left( \int_{I_{k,j}} \left|f\right|^p \right)
      \left( \int_{I_{k,j}} 1 \right)^{p/p'}
      \quad &&(\text{H\"{o}lder's inequality}) \\
      &= 2^{k(p-1)} 2^{-k(p-1)}
      \sum_{j=1}^{2^k} \int_{I_{k,j}} \left|f\right|^p
      \quad &&(p/p' = p-1) \\
      &= \int_0^1 \left| f \right|^p.
    \end{aligned}
  \]
  Taking limit superior, we obtain
  $\limsup \left\| f_k \right\|_p \le \left\| f \right\|_p$.
  However by Fatou's Lemma, we have
  $\left\| f \right\|_p \le \liminf \left\| f_k \right\|_p$.
  Combining the inequalities, we find $\lim \|f_k\|_p = \|f\|_p$
  and the proof is completed.
\end{exercise}
